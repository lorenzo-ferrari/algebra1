\documentclass[10pt]{article}

\usepackage{amsmath, amsthm, amssymb, amsfonts}
\usepackage{mathrsfs}
\usepackage{thmtools}
\usepackage{graphicx}
\usepackage{setspace}
\usepackage{geometry}
\usepackage{float}
\usepackage{hyperref}
\usepackage[utf8]{inputenc}
\usepackage{framed}
\usepackage[dvipsnames]{xcolor}
\usepackage{tcolorbox}
\usepackage{tikz}
\usepackage{tikz-cd}
\usepackage{mathtools}
\usepackage{enumitem}
\usepackage[italian]{babel}

\setlength{\parindent}{0pt}

\colorlet{LightGray}{White!90!Periwinkle}
\colorlet{LightOrange}{Orange!15}
\colorlet{LightGreen}{Green!15}

\newcommand{\HRule}[1]{\rule{\linewidth}{#1}}

\newcommand{\Z}{\mathbb{Z}}
\newcommand{\R}{\mathbb{R}}
\newcommand{\Q}{\mathbb{Q}}
\newcommand{\N}{\mathbb{N}}
\newcommand{\C}{\mathbb{C}}
\newcommand{\Zp}{\Z/p\Z}
\newcommand{\Zq}{\Z/q\Z}
\newcommand{\Zn}{\Z/n\Z}
\newcommand{\F}{\mathbb{F}}
\newcommand{\Fp}{\F_p}
\newcommand{\Fq}{\F_q}
\newcommand{\Fpn}{\F_{p^n}}
\newcommand{\Fpm}{\F_{p^m}}

\newcommand{\tri}{\vartriangleleft}
\newcommand{\set}[1]{\left\{ #1 \right\}}
\newcommand{\grp}[1]{\langle #1 \rangle}

\renewcommand{\ker}{\text{Ker}}
\newcommand{\imm}{\text{Im}}
\newcommand{\Inn}{\text{Inn}}
\newcommand{\Aut}{\text{Aut}}
\newcommand{\ord}{\text{ord}}
\renewcommand{\char}{\text{char}}
\newcommand{\sgn}{\text{sgn}}
% ancora da sostituire:
\newcommand{\Gal}{\text{Gal}}
\newcommand{\Hom}{\text{Hom}}
\newcommand{\id}{\text{id}}
\newcommand{\Fix}{\text{Fix}}
\newcommand{\orb}{\text{orb}}
\newcommand{\stab}{\text{stab}}
\newcommand{\GL}{\text{GL}}

% siamo consistenti e usiamo le frecce corte
\renewcommand{\iff}{\Leftrightarrow}
\renewcommand{\implies}{\Rightarrow}
\renewcommand{\impliedby}{\Leftarrow}

% per la divisibilità usiamo | o \mid ?

% c'è un TODO nella proposizione "(*) automorfismi", i.e. Aut(Sn) = Sn

% soluzione ottima per gestire eventuali proposizioni senza titolo (i.e. con argomento opzionale):
% % Define the 'definition' environment
% \newenvironment{definition}[1][Definition] % Optional argument for custom title
% {
%     \par\vspace{1em} % Start with some vertical space
%     \noindent % Suppress indentation
%     \textbf{#1.} \itshape % Bold title and italic content
% }
% {
%     \par\vspace{1em} % End with vertical space
% }
% soluzione che funziona: "proposition2"

\theoremstyle{plain}

\newenvironment{definition}[1]{\par \textbf{Definizione - #1:}}{\par}
\newenvironment{proposition}[1]{\par \textbf{Proposizione - #1:}}{\par}
\newenvironment{proposition2}{\par \textbf{Proposizione:}}{\par}
\newenvironment{theorem}[1]{\par \textbf{Teorema - #1:}}{\par}
\newenvironment{lemma}[1]{\par \textbf{Lemma - #1:}}{\par}
\newenvironment{lemma2}{\par \textbf{Lemma:}}{\par}
\newenvironment{corollary}[1]{\par \textbf{Corollario - #1:}}{\par}
\newenvironment{corollary2}{\par \textbf{Corollario:}}{\par}
\newenvironment{observation}[1]{\par \textbf{Osservazione - #1:}}{\par}
\newenvironment{example}{\par \textbf{Esempio:}}{\par}
\newenvironment{example2}[1]{\par \textbf{Esempio - #1:}}{\par}
\newenvironment{exercise}{\par \textbf{Esercizio:}}{\par}
\newenvironment{solution}{\par \begin{changemargin}{0.5cm}{0cm} \underline{Soluzione:} }{\end{changemargin}}
\renewenvironment{proof}{\par \begin{changemargin}{0.5cm}{0cm} \underline{dim.} }{\end{changemargin}}
\newenvironment{bdef}{\par \begin{changemargin}{0.5cm}{0cm} \underline{buona def.} }{\end{changemargin}}


\newtheorem*{theorem*}{Teorema}


\setstretch{1.2}
\geometry{
    textheight=9in,
    textwidth=6.5in,
    top=1in,
    headheight=12pt,
    headsep=25pt,
    footskip=30pt
}


% ------------------------------------------------------------------------------

\begin{document}
\def\changemargin#1#2{\list{}{\rightmargin#2\leftmargin#1}\item[]}
\let\endchangemargin=\endlist
% ------------------------------------------------------------------------------
% Cover Page and ToC
% ------------------------------------------------------------------------------
\begin{titlepage}
    \begin{center}
        {\bfseries
        \textbf{}
        \vspace{2.0cm}
        
		{ \LARGE
        \HRule{1.5pt} \\
		Dispense di Algebra 1
		\HRule{2.0pt} \\
        }
        
        \large
        \vspace{0.6cm} Università di Pisa \\
        \vspace{0.6cm} Basate sui corsi tenuti dai professori \\ 
        Ilaria Del Corso e Davide Lombardo (a.a 2023/2024)\\
        Ilaria Del Corso e Leonardo Patimo (a.a. 2024/2025)
        }
		 \vspace*{7\baselineskip} 


        \vfill
        { \large \textbf{Daria Pasqualetti, Lorenzo Ferrari} }

        \vspace{0.8cm}

        {\normalsize Le dispense che avete in mano nascono dagli appunti di Daria dell'a.a. 2023/2024, l'anno successivo Lorenzo le ha integrate sulla base del nuovo corso. I due programmi erano di fatto identici.
        Sono inclusi tutti$\pm \varepsilon$ i fatti di teoria, gli esercizi lunghi o interessanti svolti a esercitazione e qualche fatto proveniente da esami o dal libro ``Esercizi scelti di Algebra. Volume 2.'' di  R. Chirivì, I. Del Corso, R. Dvornicich.

        \smallbreak
        Per segnalare typo:
        \href{mailto:d.pasqualetti3@studenti.unipi.it}{d.pasqualetti3@studenti.unipi.it}
        $ \lor $
        \href{mailto:l.ferrari41@studenti.unipi.it}{l.ferrari41@studenti.unipi.it}.}
    \end{center}
\end{titlepage}

\tableofcontents
\newpage

% ------------------------------------------------------------------------------

\section{Teoria dei gruppi}

Dove non diversamente specificato $G$ è un gruppo con identità $e$. Il simbolo dell'operazione verrà omesso.

\subsection{Definizioni e richiami di Aritmetica}

\begin{definition}{centro}
    $Z(G) = \{h \in G : \forall \ g \in G \ gh = hg\}$.
\end{definition}
\begin{proposition}{proprietà del centro}
    $G/Z(G)$ ciclico $\Rightarrow$ $G$ abeliano ($Z(G) = G$).
    
    \textbf{Attenzione!} $G/Z(G)$ abeliano $\Rightarrow$ $G$ abeliano è falsa, un controesempio è $Q_8$.
\end{proposition}
\begin{definition}{centralizzatore di un elemento}
    se $g \in G$,  $Z_G(g) = \{h \in G : gh = hg\}$.
\end{definition}
\begin{definition}{centralizzatore di un sottogruppo}
    se $H < G$, $Z_G(H) = \bigcap_{h \in H} Z_G(h)$.
\end{definition}
\begin{definition}{classe di coniugio di un elemento}
    $Cl(g) = \{hgh^{-1} : h \in G \}$.
\end{definition}
\begin{theorem}{Lagrange}
    sia $G$ finito e $H< G$. Allora $\#H \mid \#G$. Corollario: $g \in G \Rightarrow \text{ord}(g) \mid \#G$.
\end{theorem}
\begin{definition}{prodotto diretto}
    Dati $(H,*_H), (K,*_K)$ gruppi si può dare una struttura di gruppo al loro prodotto cartesiano $H \times K$, prendendo come operazione $*$ tale che $(h_1,k_1)*(h_2,k_2) = (h_1*_Hh_2, k_1*_Kk_2)$. Tale gruppo è detto \textit{prodotto diretto} di $H, K$.
\end{definition}
\begin{proposition}{prodotto di sottogruppi}
    Siano $H,K \leq G$. $HK$ sottogruppo $\iff HK = KH$. Si nota che ciò è certamente verificato se almeno uno tra $H$ e $K$ è normale.
\end{proposition}
\begin{proposition}{cardinalità del prodotto di sottogruppi}
    Siano $H,K \leq G$. A prescindere dal fatto che $HK$ sia o meno un sottogruppo, $\#HK = \frac{\#H \#K}{\#H\cap K} = \#KH$.
\end{proposition} 

\subsection{Teoremi di omomorfismo}
\begin{theorem}{1$^{\circ}$ di omomorfismo}
    \newline
    \begin{minipage}{0.7\textwidth}
    Sia $\varphi : G \rightarrow G'$ un omomorfismo, $N \tri G$ tale che $N \subseteq \ker(\varphi)$. Allora $\exists ! f$ che fa commutare il diagramma a lato. Inoltre $N = \ker(\varphi) \Rightarrow f$ iniettiva.
    \end{minipage}\hfill\hfill
    \begin{minipage}{0.1\textwidth}  
    \end{minipage}\hfill
    \begin{minipage}{0.2\textwidth}  
    \begin{tikzcd}
    G \arrow{r}{\varphi} \arrow{d}{\pi_N} & G'\\
    G/N \arrow[dashed]{ur}[swap]{f}
    \end{tikzcd} 
    \end{minipage}\hfill
\end{theorem}
\begin{proof}
    Considero la funzione $f : G/N \rightarrow G'$ data da $f(gN\mapsto \varphi(g))$.
    \begin{itemize}
        \item buona def.: se scelgo due diversi rappresentanti per la classe in $G/N$  allora si ha $g_1N = g_2N \Rightarrow g_2^{-1}g_1 \in N \Rightarrow g_2^{-1}g_1 \in \ker(\varphi) \Rightarrow \varphi(g_2)^{-1}\varphi(g_1) = e' \Rightarrow \varphi(g_1) = \varphi(g_2)$ quindi non ci sono problemi.
        \item è omomorfismo: usando che $\varphi$ è omomorfismo si ha $gNhN = ghN \Rightarrow f(gNhN) = f(ghN) = \varphi(gh) = \varphi(g)\varphi(h) = f(gN)f(hN)$ 
    \end{itemize}
    Dimostriamo ora che $\ker(f) = \ker(\varphi)/N$. Si nota intanto che l'espressione a destra ha senso perché $N$ è un sottogruppo normale in $G$ contenuto in $\ker(\varphi)$, e quindi è normale in $\ker(\varphi)$. Inoltre $f(gN) = e' \iff \varphi(g) = e' \iff g \in \ker(\varphi) \iff gN \in \ker(\varphi)/N$. Da ciò segue l'iniettività se $\ker(\varphi)= N$.
\end{proof}
\begin{theorem}{2$^{\circ}$ di omomorfismo}
    Siano $H,K \tri G$, $H \leq K$. Allora $\frac{G/H}{K/H} \cong G/K$.
\end{theorem}
\begin{proof}
    Notiamo intanto la buona def. Infatti da $H \tri G$ e $H \leq K$ segue che $H$  normale anche in $K$. Quindi $K/H$ è ben definito. Inoltre da $K \tri G$ segue $K/H \tri G/H$.
    Considero l'omomorfismo $f: G/H \rightarrow G/K$ tale che $f(gH \mapsto gK)$.
    \begin{itemize}
    \item buona def.: se scelgo due diversi rappresentanti per la classe in $G/H$  allora si ha $g_1H = g_2H \Rightarrow g_2^{-1}g_1 \in H \subset K \Rightarrow g_1K = g_2K$ quindi non ci sono problemi
    \item è omomorfismo: chiaro per le proprietà del gruppo quoziente
    \end{itemize}
    Dimostriamo ora che $\ker(f) = K/H$. Infatti $f(gH) = K \iff gK = K \iff g \in K$. Per il $1^{\circ}$ teorema di omomorfismo si ha la tesi.
\end{proof}
\begin{theorem}{3$^{\circ}$ di omomorfismo} Siano $H < G, N \tri G$. Allora $\frac{H}{H \cap N} \cong \frac{HN}{N}$.
\end{theorem}
\begin{proof}
    Per normalità di $N$ si ha $HN$ gruppo e $N \cap H \tri H$, da cui le buone definizioni degli oggetti coinvolti. Considero ora $f: H \rightarrow HN/N$ definita da $f(h \mapsto hN)$. È un omomorfismo per le proprietà del gruppo quoziente. 
    Dimostriamo che $\ker(f) = H \cap N$: si ha $f(h) = N \iff hN = N \iff h \in N$, ma $h\in H$ per definizione, quindi $\ker(f) = H \cap K$. A questo punto la tesi segue dal $1^{\circ}$ teorema di omomorfismo.
\end{proof}

\subsection{Sottogruppi normali e automorfismi interni}
\begin{definition}{normalizzatore}
    Dato $H \leq G$ chiamiamo $N_G(H) := \{g \in G : gHg^{-1} = H\}$.
\end{definition}

Dalla definizione segue che $N_G(H)$ è un gruppo, $H \tri N_G(H)$, $H \tri G \iff N_G(H) = G$ e $Z_G(H) < N_G(H)$. Talvolta si chiama normalizzatore di un elemento il normalizzatore del sottogruppo generato.

\begin{example}
    per $G = S_3$ si ha $N_G((1,2)) = \grp{(1,2)}$, $N_G((1,2,3)) = G$ (e infatti $\grp{(1,2,3)} \tri S_3$); per $G = S_4$ si ha $N_G((1,2)) = \set{id,(1,2),(3,4),(1,2)(3,4)}$.
\end{example}
\begin{definition}{automorfismi interni}
    Sia $g \in G$; denotiamo con $\varphi_g \in \Aut(G)$ il coniugio $\varphi_g(x \mapsto gxg^{-1})$. $\Inn(G) := \{\varphi_g : g \in G\}$ viene chiamato il gruppo degli \textit{automorfismi interni} di $G$.
\end{definition}
Per definizione $H < G$ è normale se e solo se è è invariante per automorfismi interni.
\begin{bdef}
    $\varphi_g \in \Aut(G)$ perché $\varphi_g(h_1)\varphi_g(h_2) = gh_1g^{-1}gh_2g^{-1} = gh_1h_2g^{-1} = \varphi_g(h_1h_2)$. Inoltre è un gruppo perché $\varphi_{g_1}\circ\varphi_{g_2}(h) = g_1g_2hg_2^{-1}g_1^{-1} = g_1g_2h(g_1g_2)^{-1} = \varphi_{g_1g_2}(h)$, ovvero composizione di automorfismi interni resta automorfismo interno. L'identità è $\varphi_e$, l'inverso di $\varphi_g$ è $\varphi_{g^{-1}}$ per quanto appena detto.
\end{bdef}
\begin{proposition}{proprietà degli automorfismi interni}
    (1) $\Inn(G) \lhd \Aut(G)$, (2) $\Inn(G) \cong G/Z(G)$
\end{proposition}
\begin{proof} 
    \begin{enumerate}
    \item[(1)] basta notare che $\forall f \in \Aut(G)$ si ha $f \circ \varphi_g \circ f^{-1} = \varphi_{f(g)}$. Infatti $\forall h \in G$ si ha $f \circ \varphi_g \circ f^{-1}(h) = f(gf^{-1}(h)g^{-1}) = f(g)f(f^{-1}(h))f(g)^{-1} = f(g)hf(g)^{-1} = \varphi_{f(g)}(h)$
    \item[(2)] Consideriamo l'omomorfismo di gruppi $\varphi: G \rightarrow \Aut(G)$, $\varphi(g \mapsto \varphi_g)$ (si vede facilmente che è omomorfismo). Notiamo che $\varphi_g(h) = h \iff gh = hg$ e quindi $g \in \ker(\varphi) \iff g \in Z(G)$ (l'uguaglianza di prima deve valere $\forall h \in G$). Allora per il primo teorema di omomorfismo si ha: $G/Z(G) \cong \Inn(G)$, come voluto. 
    \end{enumerate}
    \end{proof}
\begin{example2}{\normalfont $\Inn(S_3) = \Aut(S_3)$}
    vale in particolare  $\Inn(S_3) = \Aut(S_3) \cong S_3$.

    \hyperlink{InnSn}{Questo risultato si può generalizzare.}
\end{example2}
\begin{proof}
    Innanzitutto $\Inn(S_3) \cong S_3$ per il teorema appena dimostrato (il centro di $S_3$ è banale). Notiamo ora che $\#\Aut(G) \leq 6$. Infatti $S_3$ è generato dalle sue trasposizioni, la cui immagine può essere solo un'altra trasposizione. Quindi un automorfismo di $S_3$ deve permutare le trasposizioni $\Rightarrow$ ci sono (al più) 6 possibilità $\Rightarrow \#\Aut(G) \leq 6$ $\Rightarrow \Inn(S_3) \cong S_3 \cong \Aut(S_3)$.
\end{proof}
\begin{definition}{sottogruppo caratteristico}
    $H < G$ si dice \textit{caratteristico} se $\forall \varphi \in \Aut(G)$ vale $\varphi(H)=H$.
\end{definition}

Un sottogruppo caratteristico è invariante per automorfismi, quindi in particolare per automorfismi interni, ed è quindi normale... ma non vale il viceversa! Infatti $\Z/2\Z \times \Z/2\Z$ è abeliano, quindi tutti i suoi sottogruppi sono normali, tuttavia l'omomorfismo che scambia le componenti dimostra che $\Z/2\Z \times \set{0}$ non è caratteristico.

\begin{proposition}{la normalità non è transitiva}
    Un controesempio sono i sottogruppi $D_2 < D_4 < D_8$: si ha infatti $D_2 \tri D_4$, $D_4 \tri D_8$ ma $D_2 \not \tri D_8$.
\end{proposition}
\begin{proof}
    (si rimanda allo \hyperlink{diedrale}{studio del gruppo diedrale} per dettagli sulle dimostrazioni che seguono.) Si ha $D_2 \tri D_4 \tri D_8$ perché ciascuno ha indice 2 nel successivo, ma $D_2 \not \tri D_8$ perché, chiamando $r,s$ la rotazione e la simmetria standard di $D_8$ si ha $D_2 = <r^4, s> = \{id, r^4, s, sr^4\}$ ma $srD_2(sr)^{-1} = srD_2sr = \{id, srr^4sr, srssr, srsr^4sr\} = \{id, r^4, sr^2, sr^6\}$. 
\end{proof}
Rafforzando leggermente le ipotesi, il fatto vale.
\begin{proposition}{fatto utile}
    se $C < N< G$ con $C$ caratteristico in $N$ e $N \tri G$, allora $C \tri G$.
\end{proposition}
\begin{proof}
    $C \tri G \iff \forall \varphi \in \Inn(G) \ \varphi(C) = C$ ma $N \tri G \Rightarrow \forall F \in \Inn(G) \ F\mid_H \in \Aut(H)$ e $C$ caratteristico in $N \Rightarrow \forall f \in \Aut(H) \ f(C)= C$. Mettendo assieme le due cose: $ \forall \varphi \in \Inn(G) \ \varphi(C)  = \varphi\mid_H(C) = C$. 
\end{proof}
\begin{lemma}{normalizzatore-centralizzatore}
    Sia $H < G$.  Vale che $Z_G(H) \tri N_G(H)$ e inoltre
    \[
        N_G(H)/Z_G(H) \hookrightarrow \Aut(H).
    \]
\end{lemma}
\begin{proof}
    $Z_G(H) \subset N_G(H)$ è chiaro dalla definizione. Per vedere la normalità, basta considerare $g \in N_G(H)$. Per definizione, $\forall h \in H \ \exists h' \in H \ h = gh'g^{-1}$.
    Quindi si ha $\forall z \in Z_G(H) \ gzg^{-1}h = gzh'g^{-1} = gh'zg^{-1} = hgzg^{-1}$,
    e dunque $gzg^{-1} \in Z_G(H) \Rightarrow gZ_G(H) \ g^{-1} \subseteq Z_G  \Rightarrow gZ_G(H)g^{-1} = Z_G$.
    
    Considero ora $\varphi: N_G(H) \rightarrow \Aut(H)$ tale che $\varphi(g \mapsto \varphi_g)$ dove $\varphi_g(h \mapsto ghg^{-1})$ è il coniugio per $g$. È ben definita per definizione di normalizzatore, e si è visto prima che è un omomorfismo. Inoltre è chiaro che $\ker(\varphi) = Z_G(H)$, e quindi per il $1^{\circ}$ teorema di omomorfismo $N_G(H)/Z_G(H)$ è isomorfo a $\imm(\varphi) < \Aut(H)$, come voluto. 
\end{proof}

\subsection{Azioni di gruppo}
\begin{definition}{azione}
    dato $X$ insieme, chiamiamo \emph{azione} di $G$ su $X$ un omomorfismo da $G$ nel gruppo delle permutazioni degli elementi di $X$, i.e. $S(X)$. La permutazione in cui viene mandato $g \in G$ si indica con $\varphi_g$, oppure semplicemente $x \mapsto g\cdot x$ o $x \mapsto x^g$. In questo caso diciamo che $X$ è un $G$-insieme.
\end{definition}

Notiamo ora che un'azione definisce in modo naturale una relazione d'equivalenza, in cui $(x,y \in X)$ $x \sim y \iff \exists g \in G$ $g\cdot x = y$. 
\begin{bdef}
    È riflessiva perché $e\cdot x = x \Rightarrow x \sim x$. È simmetrica perché $x \sim y \Rightarrow \exists g\in G \ g\cdot x = y$ e $g\cdot x = y \Rightarrow g^{-1} \cdot y = x$ (l'azione è un omomorfismo, quindi la permutazione di $g^{-1}$ è l'inversa di quella di $g$). È transitiva perché $x \sim y, y \sim z \Rightarrow \exists g_1,g_2\in G \ g_1\cdot x = y, g_2\cdot y = z$ e quindi $g_2g_1 \cdot x = z$ (l'azione è un omomorfismo, quindi $g_2g_1 \cdot x = g_2 \cdot(g_1 \cdot x)$).
\end{bdef}
\begin{example} Nel caso in cui $X=G$ il coniugio è un'azione di gruppo ($g \cdot x := gxg^{-1}$), così come anche la moltiplicazione a sinistra/destra ($g \cdot x := gx$ oppure $xg$). Un caso in cui $X \neq G$ è dato da $G = K^\times$ con $K$ campo, $X = V$ $K-$ spazio vettoriale, e l'azione di $X$ su $G$ è data da $\lambda \cdot \underline{v} :=  \lambda \underline{v} $.

Le verifiche sono lasciate per esercizio.
\end{example}

\begin{definition}{orbita}
    data un'azione di $G$ su $X$  $\orb(x) := \{g \cdot x : g\in G\} \subseteq X$.
\end{definition} 

Le orbite sono per definizione le classi della relazione di equivalenza definita dall'azione, quindi partizionano $X$.
\begin{itemize}
    \item Nel caso del coniugio per $g \in G$, l'orbita di un elemento $x \in G$ è la sua classe di coniugio $Cl(x)$;
    \item nel caso della moltiplicazione a sinistra (o a destra), l'orbita di un elemento $x \in G$ è tutto $G$;
    \item nel caso $G = K^\times, X = V$ $K$-spazio vettoriale e $\lambda \cdot \underline{v} = \lambda\underline{v}$, vale che se $\underline{v} \neq \underline{0}$ $\text{orb}(\underline{v}) = \text{Span}(\underline{v}) \setminus \set{\underline{0}}$, mentre $\orb(\underline{0}) = \set{\underline{0}}$.
    \end{itemize}
\begin{definition}{stabilizzatore}
    data un'azione di $G$ su $X$ $\stab(x) := \set{g \in G : g \cdot x = x} $
\end{definition} 
Si verifica facilmente che $\stab(x)$ è un sottogruppo.
\begin{itemize}
    \item Nel caso del coniugio per $g \in G$, lo stabilizzatore di un elemento $x \in G$ è il suo centralizzatore $Z_G(g)$ (basta notare che $gxg^{-1} = x \Longleftrightarrow gx = xg$);
    \item nel caso della moltiplicazione a sinistra (e a destra) l'orbita di un elemento $x \in G$ è solo $\set{e}$ ($yx = x \Rightarrow y = e$);
    \item nel caso $G = K^\times$ e $X = V$ $K$-spazio vettoriale e $\lambda \cdot \underline{v} = \underline{v}$, vale che se $\underline{v} \neq \underline{0}$ $\stab(\underline{v}) = \set{1}$, mentre $\text{stab}(\underline{0}) = X$.
    \end{itemize}
\begin{lemma}{orbita-stabilizzatore}
    se $G$ è finito vale,
    \[
    \forall x \in X \quad \#\text{orb}(x) \#\text{stab}(x) = \#G.
    \]
\end{lemma}
\begin{proof}
    Poiché lo stabilizzatore è un sottogruppo di $G$, per il teorema di Lagrange si ha $\#G = [G : \text{stab}(x) ]\#\text{stab}(x)$. Basta quindi dimostrare $[G : \text{stab}(x) ] = \# \text{orb}(x)$. Consideriamo la mappa \\ $F : \text{orb}(x) \rightarrow G/\text{stab}(x)$ tale che se $y = g\cdot x$ $F(y) = g \ \text{stab}(x)$.
    \begin{itemize}
        \item buona def: se $y = g_1\cdot x = g_2\cdot x$ allora $g_2^{-1}g_1 \cdot x = x \Rightarrow g_2^{-1}g_1 \in \text{stab}(x) \Rightarrow g_1\text{stab}(x) = g_2\text{stab}(x)$ quindi la scelta è indipendente dal rappresentante di $g$.
        \item iniettività: si procede al contrario $g_1\text{stab}(x) = g_2\text{stab}(x) \Rightarrow g_2^{-1}g_1 \in \text{stab}(x) \Rightarrow g_2^{-1}g_1 \cdot x = x \Rightarrow g_1\cdot x = g_2\cdot x$
        \item suriettività: segue dal fatto che $\text{orb}(x)$ contiene tutti gli $g\cdot x$ al variare di $g \in G$ e quindi la sua immagine secondo $F$ contiene tutti i $g \text{stab}(x)$ al variare di $g \in G$.
    \end{itemize}
\end{proof}

Osserviamo che preso $H < G$, se si considera l'azione di $G$ su $G/H$ data sulla moltiplicazione a sinistra si ottiene il teorema di Lagrange: $\#G/H = \#\text{orb}(H) = \frac{\#G}{\#\text{stab}(H)} = \frac{\#G}{\#H}$; poiché $G/H$ è un intero, $\#H \mid \#G$.

\begin{example2}{cardinalità classi di coniugio}
    $\forall x \in G \ \ \#Cl(x) = [G : Z_G(x)]$; equivalentemente, se $G$ finito, si ha $\forall x \in G$ $\#G = \#Cl(x) \#Z_G(x)$.
\end{example2}
\begin{proof}
    considerando come azione quella di coniugio (azione di $G$ su $G$) si ha $\forall x \in G$ $\text{stab}(x) = Z_G(x)$, $\text{orb}(x) = Cl(x)$ e si ottiene la tesi per il lemma orbita-stabilizzatore.   
\end{proof}
\begin{corollary}{gli stabilizzatori sono tutti coniugati}
    se l'azione agisce transitivamente sull'insieme (ovvero $\forall x,y \in X \ \exists g \in G \ \ g \cdot x = y$).
\end{corollary}
\begin{proof} $\forall x,y \in X $ se $g_0 \cdot x = y$ si ha 
    \begin{align*}
        \text{stab}(y) & = \set{g \in G : g \cdot y = y} = \\
         & = \set{g \in G : gg_0 \cdot x = g_0 \cdot x} = \\
         & = \set{g \in G : g_0^{-1}gg_0 \cdot x = x} = \\
         & = \set{g \in G : g_0^{-1}gg_0 \in \text{stab}(x)} = \\
         & = g_0\text{stab}(x)g_0^{-1}
    \end{align*} 
\end{proof}
\begin{example2}{esiste azione senza punti fissi}
    se $\#X \geq 2$ e l'azione agisce transitivamente sull'insieme allora $\exists g \in G$ tale che la sua permutazione associata $\varphi_g$ non abbia punti fissi (ovvero $g\cdot x \neq x \ \forall x \in X$).
\end{example2}
\begin{proof}
    \begin{align*}
        g \text{ agisce senza punti fissi } & \iff \forall x \in X g \not \in \text{stab}(x) \iff \\
        & \iff g \not \in \bigcup_{x \in X} \text{stab}(x) \iff  \qquad \text{ (fissiamo $x_0 \in X$ e usiamo transitività)} \\
        & \iff g \not \in \bigcup_{h \in G} \text{stab}(h \cdot x_0) \iff \qquad \text{ usiamo il corollario di sopra} \\
        & \iff g \not \in \bigcup_{h \in G} h\text{stab}(x_0) h^{-1} 
    \end{align*}
    Usando che \hyperlink{es1}{$G$ non è unione di sottogruppi coniugati} si ha che un tale $g$ esiste sempre se $\text{stab}(x_0) \neq G$. Ma $\text{stab}(x_0) = G$ e $\text{orb}(x_0) = X$ (azione transitiva) dà un assurdo per il lemma orbita-stabilizzatore, quindi $\text{stab}(x_0) \neq G$ da cui la tesi. 
\end{proof}
\begin{theorem}{formula delle classi}
    Sia $G$ finito e $R$ insieme di rappresentanti per le classi di coniugio di $G$. Vale allora:
    \[
    \#G = \#Z(G) + \sum_{g \in R \setminus Z(G)} \frac{\#G}{\#Z_G(g)}
    \]
\end{theorem}
\begin{proof}
    Notiamo intanto che l'omomorfismo che manda un elemento nel coniugio per quell'elemento è un'azione di un gruppo su sé stesso (omettiamo le verifiche). Consideriamo quindi l'azione data dal coniugio. Chiaramente in questa azione l'orbita di un elemento è la sua classe di coniugio, mentre lo stabilizzatore è il centralizzatore. Per il lemma orbita-stabilizzatore allora si ha $\forall g \in G \ \#Cl(g)\#Z_G(g) = \#G$.  Poiché $G$ può essere partizionato nelle sue classi di coniugio, scegliendo un insieme $R$ di rappresentanti per esse, vale $\#G = \sum_{g \in R} \#Cl(g) = \sum_{g \in R} \frac{\#G}{\#Z_G(g)}$. Notiamo ora che se $g \in Z(G)$ allora $Cl(g) = \{g\}$ e $Z_G(g) = G$, e quindi $\frac{\#G}{\#Z_G(g)} = 1$. Notiamo anche che ciò implica che in $R$ si trovano tutti gli elementi di $Z(G)$, visto che ciascuno è l'unico possibile rappresentante della propria classe di coniugio. Isolando questi termini nella sommatoria si ottiene la formula dell'enunciato.
\end{proof}
\begin{corollary}{formula delle classi per sottogruppi normali}
    Se $H \tri G$ allora vale una formula delle classi ``modificata'':
    \[
    \#H = \#(Z(G)\cap H) + \sum_{g \in (R \setminus Z(G)) \cap H} \frac{\#G}{\#Z_G(g)}.
    \]
\end{corollary}
\begin{proof}
    Basta notare che $H \tri G \Rightarrow H$ è unione di classi di coniugio e procedere come nella dimostrazione del teorema.
\end{proof}
Alcune applicazioni interessanti della formula delle classi si trovano nella sezione \hyperlink{gruppi finiti}{``Fatti utili sui gruppi finiti''}.

\begin{definition}{(*) omomorfismi di azioni}
    Siano $X$ e $Y$ rispettivamente un $G$-insieme e un $H$-insieme. Si chiama omomorfismo di azioni una coppia $(\psi, \sigma)$ con $\psi : G \to H$ omomorfismo e $\sigma : X \to Y$ tale per cui $\forall g \in G, x \in X \ \sigma(g \cdot x) = \psi(g) \cdot \sigma(x)$. Se $\psi$ e $\sigma$ sono bigettive si dice che le due azioni sono isomorfe.
\end{definition}
\begin{proposition2}
    Sia $\varphi : G \to S(X)$ un'azione transitiva del gruppo $G$ sull'insieme $X$. Sia $H < G$ uno stabilizzatore. Allora, in modo del tutto analogo alla dimostrazione di orbita-stabilizzatore, $\varphi$ è isomorfa all'azione di moltiplicazione di $G$ su $G/H$.
\end{proposition2}
\begin{exercise}
    Sia $\varphi : G \to S(X)$ un'azione 2-transitiva, cioè tale per cui ogni coppia ordinata di elementi distinti $(w, x)$ può essere mappata in ogni altra coppia ordinata di elementi distinti $(y, z)$ attraverso un opportuno elemento di $G$: $y = g\cdot w, z = g\cdot x$. Allora gli stabilizzatori di $\varphi$ sono massimali in $G$.
\end{exercise}
\begin{theorem}{Poincaré}
    Sia $H < G$ con $[G : H] = n$. Allora esiste un sottogruppo normale $N \tri G$, $N \subseteq H$ con $[G : N] \mid n!$.
\end{theorem}
\begin{proof}
    Consideriamo l'azione $\varphi : G \to S(G / H)$ di moltiplicazione di $G$ sui laterali di $H$. Sia $N = \ker(\varphi) \tri G$. La moltiplicazione per elementi di $N$ fissa il laterale $eH$, dunque $N \subseteq H$. Inoltre $G / N \cong \imm(\varphi) < S(G / H)$, da cui $[G : N] \mid n!$.
\end{proof}

\subsection{Teoremi fondamentali}
\begin{theorem}{Cauchy}
    Sia $G$ finito, e sia $p$ primo che divide $\#G$. Allora $\exists g\in G \ \text{ord}(g) = p$.
\end{theorem}
\begin{proof}
    Notiamo intanto che è sufficiente dimostrare che esiste un elemento $x \in G$ tale che $p \mid \text{ord}(x)$. Infatti $y := x^{\frac{\text{ord}(x)}{p}}$ avrebbe ordine $p$, come voluto.
    
    Trattiamo prima il caso in cui $G$ è abeliano. Scriviamo $\#G = pn$ e procediamo per induzione estesa su $n$.
    
    Passo base: $n = 1 \ \Rightarrow \#G = p \Rightarrow G \cong \Zp$ e la tesi è ovvia.
    
    Passo induttivo: prendiamo ora $x \in G \setminus \{e\}$. Poichè $G$ è abeliano tutti i suoi sottogruppi sono normali in $G$ e quindi $\grp{x} \tri G$. Distinguiamo ora due casi: 
    \begin{itemize}
        \item $p \mid \text{ord}(x)$ allora si ha già la tesi.
        \item $p \nmid \text{ord}(x)$ allora $p \mid \#(G/\grp{x})$. $G/\grp{x}$ ha cardinalità minore di $G$ perché $x \neq e$. Allora per ipotesi induttiva nel gruppo quoziente $\exists \overline y$ tale che $p \mid \text{ord}(\overline y)$. Notiamo però che considerando l'omomorfismo di proiezione $\pi : G \rightarrow G/\grp{x}$ poiché l'ordine in arrivo è un divisore dell'ordine in partenza, si ha che, se $\overline y = \pi(y)$, $p \mid \text{ord}(\overline y) \mid \text{ord}(y)$, come voluto.
    \end{itemize}
    Trattiamo ora il caso generale, di nuovo per induzione. Il passo base è analogo a prima.
    
    Passo induttivo: distinguiamo due casi:
    \begin{itemize}
        \item $\exists H < G, H \neq G, \ p \mid \#H$. Allora per ipotesi induttiva si avrebbe la tesi. 
        \item $\forall H < G, \ p \nmid \#H$. In particolare allora $\forall g \in G \setminus Z(G) \ p \nmid Z_G(g)$. Guardiamo allora la formula delle classi modulo $p$. $p \mid \#G, \frac{\#G}{\#Z_G(g)}$ e quindi $p \mid \#Z(G)$, da cui $Z(G) = G$, cioè $G$ abeliano e ci riconduciamo al caso precedente.
    \end{itemize}

\underline{seconda dim.}
    Consideriamo un'azione di $\Zp$ su $X = \{ (h_1,h_2,\dots,h_p) : h_1h_2\dots h_p = e \}$ ($p$-uple di $G$ con prodotto $e$). Notiamo intanto che $X$ è non vuoto perché $(e,\dots, e) \in X$. L'azione è da $k \mapsto f_k$ tale che $f_k(\ (h_1,\dots,h_p) \mapsto (h_{1+k},\dots,h_{p+k}) \ )$, dove nella seconda espressione gli indici sono intesi modulo $p$ (è chiaramente ben definita).
    
    $\forall \bar h = (h_1,\dots,h_p) \in X$ per il lemma orbita-stabilizzatore $\#\Zp = p = \#\text{orb}(\bar h) \#\text{stab}(\bar h)$, e quindi abbiamo due scelte: $ \#\text{orb}(\bar h) = 1$ oppure $p$. Notiamo che se $\#\text{orb}( \bar h ) = 1$ allora è formata da $p$ elementi uguali. Infatti $f_k( (h_1,\dots,h_p) ) = (h_1,\dots,h_p) \Rightarrow h_1 = h_{1+k}$ e quindi se vale $\forall k$ tutti gli elementi sono uguali a $h_1$. In particolare, o $h_1 = e$ o $\text{ord}(h_1) = p$, poiché per definizione $h_1^p = e$ e $p$ è primo. Sia $Y$ l'insieme delle $p$-uple formate da $p$ elementi uguali, ossia con orbita banale, e $Z$ un insieme di rappresentanti per gli elementi di $X$ con orbita di cardinalità $p$. Vorremmo dimostrare che $\#Y \geq 2$, perché così avremmo una $p$-upla di tutti elementi uguali e diversi da $e$.
    
    $\#X = \sum_{x \in Y \cup Z} \#\text{orb}(x) = \#Y + p\#Z \cong \#Y \pmod{p}$ perché $X$ è partizionato nelle sue orbite. Studiamo ora $\#X$. Fissati in un qualunque modo i primi $p-1$ elementi della $p$-upla l'ultimo è univocamente determinato da $h_p = (h_1\dots h_{p-1})^{-1}$. Quindi $\# X = (\#G)^{p-1}$. In particolare $p \mid \#G \Rightarrow p \mid \#X \Rightarrow p \mid \#Y$ e quindi, poiché abbiamo dimostrato che $Y$ è non vuoto, $\#Y \geq p \geq 2$.
\end{proof}
\begin{theorem}{Cayley}
    Sia $G$ finito, $\#G = n$. Allora $G \hookrightarrow S_n$ ($G$ ``si immerge'' in $S_n$, ovvero esiste una copia di $G$ in $S_n$).
\end{theorem}
\begin{proof}
    Consideriamo l'azione $g \mapsto \psi_g \in S(G) \cong S_n$ tale che $\psi_g(h \mapsto gh)$.
    \begin{itemize}
        \item buona def.: la moltiplicazione a sinistra appartiene a $S(G)$ perché $gh_1 = gh_2 \iff h_1 = h_2$ (legge di cancellazione), quindi $\psi_g$ è iniettiva $\Rightarrow$ è bigettiva ($G$ è finito).
        \item è omomorfismo: si vede chiaramente $\psi_{g_1} \circ \psi_{g_2} = \psi_{g_1g_2}$
        \item iniettività: guardiamo il nucleo dell'azione, corrispondente a $\{g \in G : \psi_g = \text{id}\} = \{g \in G \mid \forall h \in G gh = h \} = \{e\}$. Questo basta per dimostrare l'iniettività.
    \end{itemize}
\end{proof}
\begin{observation}{embedding di Cayley}
    guardiamo l'immagine di un elemento di $g$ secondo quell'omomorfismo. Se $\text{ord}(g) = d$ (divisore di $n$) allora ``seguendo'' un elemento $h_0 \in G$ otteniamo $h_0, \ gh_0=h_1, \ g^2h_0 = gh_1=h_2, \ \dots , \ g^{d-1}h_0 = h_{d-1}, \ g^dh_0 = h_0$ (e da qui in poi si ripete il ciclo). Gli $h_i$ sono necessariamente tutti distinti ($h_i = h_j \iff g^ih_0= g^jh_0 \iff g^{i-j} = e \iff i-j \cong 0 \pmod{d} \iff i-j = 0$ visto che $1 \leq i,j\leq d-1$).
    Quindi $\psi_g$ nell'isomorfismo con $S_n$ è una permutazione con $\frac{n}{d}$ $d$-cicli (abbiamo appena dimostrato che ogni elemento che viene permutato sta in un $d$-ciclo).
\end{observation}
\begin{theorem}{di corrispondenza}
    Sia $N \tri G$ e consideriamo $\pi : G \rightarrow G/N = G'$ la proiezione al quoziente. Allora c'è una corrispondenza biunivoca tra i sottogruppi di $G$ che contengono $N$ e i sottogruppi di $G'$. Inoltre tale corrispondenza preserva ordinamento, indice e normalità. 
\end{theorem}
\begin{proof}
    Siano $\pi_N: G \rightarrow G/N$ la proiezione al quoziente, $X  = \{ H \leq G \mid N \subseteq H \}$, $Y = \{ \overline H \leq G/N \}$. Consideriamo le due funzioni $\alpha: X \rightarrow Y$ tale che $\alpha(H \mapsto \pi_N(H))$ e $\beta: Y \rightarrow X$ tale che $\beta( \overline H \mapsto \pi_N^{-1}(\overline H) )$. 
    \begin{itemize}
        \item buona def. di $\alpha$: $N \tri G, N \subseteq H \Rightarrow N \tri H$ e quindi è ben definito $H/N = \pi_N(H)$, che è chiaramente un sottogruppo
        \item buona def. di $\beta$: $\pi_N^{-1}(\overline H)$ è un sottogruppo perché controimmagine di sottogruppo. Inoltre da $N \in \overline H$ segue $N \subseteq \pi_N^{-1}(\overline H)$
    \end{itemize}
    Notiamo ora che: $\alpha \circ \beta$ manda $H \mapsto  \pi_N(\pi_N^{-1}(\overline H)) = H$ dove l'ultima uguaglianza segue dal fatto che $\pi_N$ è suriettiva. Invece $ \beta \circ \alpha$ manda 
    $H \mapsto  \pi_N(\pi_N(H))^{-1} = \pi_N(H/N)^{-1} = \{g \in G : gN \in H/N \} =  H$
    dove l'ultima uguaglianza segue da $N \subseteq H$ (quindi $\exists h \in H \ gN = hN \iff \exists h \in H, n \in N \  g = hn \iff g \in H$ dove $\Rightarrow$ segue da $N \subseteq H$, mentre per $\Leftarrow$ basta scegliere, per ogni $g \in H$, $h = g, n = e$).
    
    Quindi $\alpha$ e $\beta$ sono entrambe bigezioni, una inversa dell'altra. Dimostriamo le proprietà della tesi per $\alpha$.
    \begin{itemize}
        \item Preserva il contenimento: segue dalla definizione
        \item Preserva la normalità: segue dal fatto che $\pi_N$ è un omomorfismo, e quindi controimmagine di sottogruppi normali è normale, e inoltre è suriettiva, e quindi manda sottogruppi normali in sottogruppi normali
        \item Preserva gli indici:  occorre dimostrare che $\forall H \leq G , N \subseteq H \ [G:H] = [G' : \alpha(H)] = [G' : H/N]$. Consideriamo la funzione $T: G/H \rightarrow \frac{G'/N}{H/N}$ (nota: poiché non è assunta la normalità, non stiamo intendendo il gruppo quoziente ma solo l'insieme delle classi laterali) data da $T(xH \mapsto (xN)H/N)$. $T$ è ben definita perché $xH = yH \Rightarrow \exists h \in H \ x = yh \Rightarrow (xN)H/N = (yhN)H/N = (yNh)H/N = (yN)H/N$ dove la penultima uguaglianza segue dalla normalità di $N$. Inoltre $T$ è suriettiva per definizione ed è iniettiva perché se si ha $(xN)H/N = (yN)H/N$ allora $\exists h \in H \ xN = (yN) (hN) \Rightarrow x \in yNh \Rightarrow$ (usando che $N\subseteq H$) $\exists h' \in H \ x = yh' \Rightarrow xH = yH$.
        
        (Si noti che se il sottogruppo in questione fosse stato abeliano, per dimostrare che l'indice viene preservato sarebbe bastato il $2^{\circ}$ teorema di omomorfismo.)
    \end{itemize}
\end{proof}
\begin{theorem}{decomposizione in prodotto diretto}
    Siano $H,K \tri G$ tali che $H \cap K = \{e\}$ e $HK = G$. Allora $G \cong H \times K$.
\end{theorem}
\begin{proof}
    Mostriamo prima un lemma, ossia che nelle ipotesi del teorema si ha $hk = kh \ \forall h\in H, k \in K$. Fissiamo tali $h,k$. Per normalità di $H$ in G si ha $khk^{-1} = \tilde h \in H$. Quindi $khk^{-1}h^{-1} = \tilde h h^{-1} \in H$. In modo analogo si mostra  $khk^{-1}h^{-1} \in K$. Quindi $khk^{-1}h^{-1} \in H \cap H = \{ e\} \Rightarrow kh= hk$. \\Consideriamo ora $\varphi: H \times K \to G$ $\varphi(\ (h,k) \mapsto hk)$. $\varphi$ è omomorfismo perché per il lemma appena dimostrato $h_1h_2k_1k_2 =  h_1k_1h_2k_2$ $\Rightarrow$ $\varphi((h_1h_2, k_1k_2)) =\varphi((h_1, k_1)) \varphi((h_2k_2))$.  Da $HK = G$ segue $\varphi$ suriettiva. Inoltre $\varphi$ è iniettiva perché $(h,k) \in \ker(\varphi) \iff hk = e \iff H \ni h = k^{-1} \in K \Rightarrow h,k \in H\cap K \Rightarrow h=k=e$.
\end{proof}
\begin{definition}{prodotto semidiretto}
    Siano $H, K$ due gruppi, e $\psi: K \rightarrow \Aut(H)$ un omomorfismo. Indichiamo con $\psi_g$ l'immagine di $g\in K$ secondo $\psi$ e definiamo l'operazione $*$ sul prodotto cartesiano $H \times K$ data da $(h_1,k_1)*(h_2,k_2) = (h_1\psi_{k_1}(h_2), k_1k_2)$. Indichiamo con $H \rtimes_{\psi} K$ questo nuovo gruppo, che chiamiamo un \textit{prodotto semidiretto} di $H$ e $K$.
\end{definition}

Si noti che se $\psi \equiv id_{K}$ si ottiene il prodotto diretto (e viceversa). D'ora in poi il segno ``$*$'' verrà omesso, come le altre operazioni.

\begin{bdef} Verifichiamo che è un gruppo:
    \begin{itemize}
        \item elemento neutro: $(h,k)*(h',k') = (h',k') \iff (h\phi_k(h'),kk') = (h',k') \iff h\phi_k(h') = h' \text{ e } k' = e_K \iff h\phi_{e_K}(h') = hh' = h' \text{ e } k' = e_K \iff (h,k) = (e_H,e_K) $ e analogamente $(h',k')*(h,k) = (h',k') \iff (h'\phi_{k'}(h),k'k) = (h',k') \iff \phi_{k'}(h) = e_H \text{ e } k' = e_k \iff (h,k) = (e_H,e_K)$.
        \item associatività: siano $h,h',h'' \in H$, $k,k',k'' \in K$.
        \begin{align}
         ((h,k)*(h',k'))*(h'',k'') & = (h\phi_k(h'),kk')*(h'',k'') = \nonumber \\
         & = (h\phi_k(h')\phi_{kk'}(h''),kk'k'') = \nonumber \\
         & = (h\phi_k(h')\phi_h(h'')\phi_{k'}(h''),kk'k'') \nonumber ;
         \end{align} 
         \begin{align}
         (h,k)*((h',k')*(h'',k'')) & = (h,k)*(h'\phi_{k'}(h''),k'k'') = \nonumber \\
         & = (h\phi_k(h'\phi_{k'}(h'')),kk'k'') = \nonumber \\
         & = (h\phi_k(h') \phi_k\circ\phi_{k'}(h'')),kk'k'') \nonumber.
         \end{align}
        Quindi le due espressioni sono uguali e si ha associatività.
        \item inverso: detto $(h,k)^{-1} = (\phi_{k^{-1}}(h^{-1}), k^{-1})$, vale $(h, k)(h, k)^{-1} = (h, k)^{-1}(h, k) = e$.
    \end{itemize}
\end{bdef}
\begin{observation}{asimmetria del prodotto semidiretto}
    $H \times \{e_K\} \tri H \rtimes K$; \textbf{non} vale con $\{e_H\} \times K$.
\end{observation}
\begin{proof}
    Basta considerare la proiezione sul secondo fattore, e notare che $H \times \{e_K\}$ ne è il nucleo. Non si può fare lo stesso con la proiezione sul primo fattore per quella che è la definizione del gruppo, e dopo il teorema successivo sarà chiaro come costruire un controesempio.
\end{proof}
\begin{theorem}{decomposizione in prodotto semidiretto}
    Siano $H,K \leq G$ tali che $H \tri G$, $H \cap K = \{e\}$ e $HK = G$. Allora $G \cong H \rtimes_{\varphi} K$, dove $\varphi(k \mapsto \varphi_k)$ è la mappa che manda $k \in K$ nel coniugio per $k$ ristretto ad $H$.
\end{theorem}
\begin{proof}
    Si nota intanto che $\forall k \in K \ \varphi_k \in \Aut(H)$ per normalità di $H$.
    
    Consideriamo $F: H \rtimes_{\varphi} K \rightarrow G$ tale che $F((h,k) \mapsto hk)$.
    \begin{itemize}
        \item è omomorfismo: $F((h,k)(h',k')) = F(hkh'k^{-1},kk') = hkh'k^{-1}kk' = hkh'k' = F((h,k))F((h',k'))$
        \item è bigettiva: il ragionamento è analogo a quello per i prodotti diretti
    \end{itemize}
\end{proof}
\begin{proposition}{isomorfismo di prodotti semidiretti}
    siano $H,K$ gruppi, $\varphi, \psi: K \rightarrow \Aut(H)$ omomorfismi. Se $\exists \alpha \in \Aut(H), \ \beta \in \Aut(K)$ tali che $\forall k \in K \ \alpha \circ \varphi_k \circ \alpha^{-1} = \psi_{\beta(k)}$ allora $H \rtimes_{\varphi} K \cong  H \rtimes_{\psi} K$.
\end{proposition}
\begin{proof}
    Consideriamo $F: H \rtimes_{\varphi} K \to  H \rtimes_{\psi} K$ tale che $F((h,k) \mapsto (\alpha(k), \beta(h))$.
    \begin{itemize}
        \item è omomorfismo: $F((h,k)*_{\varphi}(h',k')) = F(h\varphi_{k}(h'),kk') =  (\alpha(h)\alpha\circ\varphi_{k}(h'),\beta(k)\beta(k')) = (\alpha(h)\psi_{\beta(k)}(\alpha(h')),\beta(k)\beta(k')) =(\alpha(h), \beta(k))*_{\psi} (\alpha(h'),\beta(k')) =  F((h,k))*_{\psi}F((h',k'))$
        \item è bigettiva perché lo sono $\alpha, \beta$
    \end{itemize}
\end{proof}
\begin{definition}{$p$-Sylow}
    Sia $G$ un gruppo finito, $\#G = p^n m$ con $p$ primo, $n \geq 1$, $p \nmid m$. Chiamiamo $p$-Sylow un sottogruppo di $G$ di cardinalità $p^n$.
\end{definition}
\begin{theorem}{Sylow}
    Sia $G$ come prima. Valgono i seguenti quattro enunciati:
    \begin{itemize}
        \item \textbf{esistenza}: $\forall \alpha \in \N, \ 1 \leq \alpha \leq n$ $\exists  H \leq G, \#H = p^{\alpha}$;
        \item \textbf{inclusione}: $\forall \alpha \in \N, \ 1 \leq \alpha \leq n-1$ $\forall  H \leq G$ tali che $\#H = p^{\alpha}$ $\exists  K \leq G, \#K = p^{\alpha+1}$ e $ H \leq K$;
        \item \textbf{coniugio}: i $p$-Sylow sono tutti coniugati;
        \item \textbf{numero}: detto $n_p$ il numero di $p$-Sylow, valgono $n_p \equiv 1 \pmod{p}$ e $n_p \mid \#G$, da cui $n_p \mid m$.
    \end{itemize}
\end{theorem}

\begin{proof}
    Sia $M = \{ X \subseteq G : \#X = p^{\alpha}\}$.  Calcoliamo innanzitutto la potenza di $p$ che divide $\#M = \binom{\#G}{p^{\alpha}} = \frac{p^nm \cdot (p^nm-1) \cdot \dots \cdot (p^nm - p^{\alpha} +1)}{p^{\alpha}\cdot(p^{\alpha}-1)\cdot \dots \cdot 1} = p^{n-\alpha}m \prod_{i=1}^{p^{\alpha}-1} \frac{p^nm-i}{p^{\alpha}-i}$. Da $p^n \geq p^{\alpha} > i$ segue che nella produttoria la valutazione $p$-adica di numeratore e denominatore è sempre la stessa, e quindi non sopravvive nessun fattore $p$ nel prodotto. Quindi la valutazione $p$-adica di $\#M$ è $n-\alpha$.
    
    Consideriamo l'azione di $G$ su $M$ data da $g\mapsto \psi_g(X \mapsto gX)$ (chiaramente è ben definita). Poiché $p^{n-\alpha +1} \nmid \#M$ e $\#M$ è somma delle cardinalità delle orbite dell'azione, deve necessariamente esistere $Y \in M$ tale che $p^{n-\alpha +1} \nmid \#\text{orb}(Y) = \frac{p^nm}{\#\text{stab}(Y)}$ per il lemma orbita-stabilizzatore. Ma allora necessariamente $p^{\alpha} \mid \#\text{stab}(Y)$.
    
    Fissiamo ora $y_0 \in Y$ e consideriamo $j: \text{stab}(Y) \rightarrow Y$ data da $j(x \mapsto y_0x)$. È ben definita per definizione di stabilizzatore ed è iniettiva per la legge di cancellazione, quindi $p^{\alpha} = \#Y \geq \text{stab}(Y)$, da cui segue $\#\text{stab}(Y) = p^{\alpha}$. Abbiamo così \textbf{esistenza}.
    
    Sia ora $S$ un $p$-Sylow di $G$ e $H \leq G$ tale che $\#H = p^{\alpha}$, $0 \leq \alpha \leq n$. Consideriamo l'insieme $G/S$ delle classi laterali di $S$ e consideriamo l'azione (di moltiplicazione a sinistra) di $H$ su $G/S$ che manda $h \in H$ in $\theta_h(gS \mapsto (hg)S)$ (chiaramente è ben definita). Per il lemma orbita-stabilizzatore e usando che le orbite partizionano $G/S$ si ha che, scelto un insieme $R$ di rappresentanti per le orbite $m = \#(G/S) = \sum_{g \in R} \text{orb}(gS) = \sum_{g\in R} \frac{\#H}{\#\text{stab}(gS)} = \sum_{g\in R} \frac{p^{\alpha}}{\#\text{stab}(gS)}$. Da ciò segue, poiché $p \nmid m$, che $\exists g_0 \in R$ tale che $p^{\alpha} = \#\text{stab}(g_0S)$ e quindi $H = \text{stab}(g_0S)$. Ma allora si ha $\forall h \in H \ hg_0S = g_0S \Rightarrow h \in g_0 S g_0^{-1}$ e quindi $H \subseteq g_0Sg_0^{-1}$, che è un $p$-Sylow. Ciò prova \textbf{coniugio} se si pone $\alpha = n$ (l'uguaglianza segue per cardinalità).
    Se $\alpha < n$ si ha solo che $H$ è contenuto in un $p$-Sylow, ma tanto basta per restringerci al caso di un $p$-gruppo.
    \begin{lemma2}
        in un $p$-gruppo $G$, $\forall H \lneq G \ H \lneq N_G(H)$.
    \end{lemma2}
    \begin{proof}
        Procediamo per induzione su $n= v_p(\#G)$. Il passo base $n = 1$ è ovvio perché il gruppo è isomorfo a $\Zp$, che ha solo i due sottogruppi banali per cui la tesi è ovvia. Nel passo induttivo distinguiamo due casi: 
        \begin{itemize}
        \item $Z(G) \nsubseteq H$: basta allora notare  $Z(G) \subseteq N_G(H)$
        \item $Z(G) \subseteq H$: allora quozientiamo per $Z(G)$ e concludiamo usiamo l'ipotesi induttiva, unita al teorema di corrispondenza. 
        \end{itemize}
    \end{proof}
    Consideriamo la proiezione: $\pi: N_G(H) \rightarrow N_G(H)/H$. Poiché $N_G(H)/H$ è un $p$-gruppo non banale, per il teorema di Cauchy $\exists \overline x \in N_G(H)/H \ \text{ord}(\overline x) = p$. Allora $H \subseteq \pi^{-1}(\langle \overline x \rangle)$ e $\#\pi^{-1}(\langle \overline x \rangle) = p^{\alpha+1}$. Infatti se $\overline x = xH$, si ha $\pi(H)\cup \pi(xH)\cup\dots\cup\pi(x^{p-1}H) =  \langle \overline x \rangle$ e per costruzione sono tutti disgiunti ($\overline x = xH$ ha ordine $p$). Allora, poiché le classi laterali hanno tutte la stessa cardinalità, pari a $\#H = p^{\alpha}$, si ha $\#\pi^{-1}(\langle \overline x \rangle) = \#\bigcup_{i=0}^{p-1} x^iH = p\cdot p^{\alpha}$. Questo dimostra \textbf{inclusione}.
    
    Passiamo ora al numero di $p$-Sylow. Sia $X = \set{ p\text{-Sylow}}$ e come prima fissiamo $S$ un $p$-Sylow. Il coniugato di un $p$-Sylow, avendo la stessa cardinalità, è a sua volta un $p$-Sylow e abbiamo precedentemente dimostrato che i $p$-Sylow sono tutti coniugati, quindi $n_p = \#X$ è la cardinalità della classe di coniugio di $S$. Consideriamo l'azione di $G$ per coniugio sull'insieme $X$, ben definita per quanto detto. Chiaramente $\text{orb}(S) = X$, quindi per il lemma orbita-stabilizzatore $n_p = \#\text{orb}(S) \mid \#G$. Restringiamo ora questa azione a $S$ (ossia, consideriamo solo il coniugio per elementi di $S$, come azione di $S$ su $X$). Notiamo che $S$ è l'unico elemento con orbita banale: per ogni $p$-Sylow $S'$ con orbita banale si ha $S \subseteq \text{stab}(S') = N_G(S') \Rightarrow SS'$ sottogruppo e $\#(SS') = \frac{\#S \#S'}{\#(S \cap S')} = \frac{p^n \cdot p^n}{\#(S \cap S')}$, ma $SS'$ sottogruppo di $G$ e $p$-gruppo $\Rightarrow \#(SS') = p^k$ con $k\leq n \Rightarrow \#(S\cap S') \geq p^n \Rightarrow \#(S\cap S') = p^n$ e quindi $S = S'$. Poiché le orbite partizionano l'insieme che subisce l'azione, dato $R$ insieme di rappresentanti, si ha $n_p = \sum_{S' \in R} \#\text{orb}(S') = 1 + \sum_{S' \in R\setminus\{S\}} \#\text{orb}(S') \equiv 1 \pmod{p}$. Questo conclude \textbf{numero}.
\end{proof}

\begin{exercise}
    Sia $G$ un gruppo finito. Si mostri che le seguenti proprietà sono equivalenti:
    \begin{enumerate}[label=$(\roman*)$]
        \item $G$ è \emph{nilpotente}, cioè, detta $f$ la funzione $G \overset{f}{\mapsto} \frac{G}{Z(G)}$ e $f^{k}$ la sua composizione $k$ volte, esiste $k \in \N$ tale che $f^k(G) \cong \{e\}$;
        \item $\forall H \lneq G \ H \subsetneq N_G(H)$;
        \item $G$ è isomorfo al prodotto diretto dei suoi Sylow.
    \end{enumerate}
    \tiny{HINT: dato $P$ $p$-Sylow, $P \tri N_G(N_G(P))$.}
\end{exercise}

\subsection{Teorema di struttura dei gruppi abeliani finiti}
La sezione tratta gruppi abeliani, quindi si userà principalmente la notazione additiva (vale a dire $g+h$ invece di $gh$ e $0$ invece di $e$).

\textbf{\ldots in generale:} I seguenti risultati sono conseguenze del teorema di decomposizione ciclica primaria dei moduli di torsione su PID (Capitolo 6 di Advanced Linear Algebra, Roman)! In particolare, ogni gruppo abeliano è uno $\mathbb{Z}$-modulo di torsione (per $n = \#G$ si ha $n G = \{e\}$) e $\mathbb{Z}$ è un PID. Il teorema di struttura dei gruppi abeliani finiti corrisponde alla più generale ``decomposizione in fattori invarianti''. Queste generalizzazioni non fanno parte del programma di Algebra 1.

\begin{definition}{componente di $p$-torsione}
    Dato $p$ primo e $G$ abeliano, chiamiamo così il gruppo $G(p) = \{g \in G : \exists k \in \N\ \ \text{ord}(x) = p^k\}$. (Se $G$ non è abeliano, in generale $G(p)$ non è un sottogruppo: un $3$-ciclo è prodotto di due trasposizioni.)
\end{definition}
\begin{theorem}{1}
    Se $G$ abeliano finito, $\#G = \prod_{i = 1}^s p_i^{\alpha_i}$ con $p_i$ primi,  $G \cong G(p_1) \times \dots \times G(p_s)$.
\end{theorem}
\begin{proof}
    Procediamo per induzione su $s$. Nel passo base la tesi è ovvia. Scriviamo ora $\#G = n = mm'$ con $m,m' \neq 1$ e coprimi. Consideriamo i sottogruppi $mG$ e $m'G$. Essi sono chiaramente sottogruppi di $G$ e inoltre si ha:
    \begin{itemize}
        \item $mG + m'G = G$: il contenimento $\subseteq$ è ovvio, mentre $\supseteq$ utilizza il teorema di Bezout: poiché $m,m'$ sono coprimi, esistono $a,b\in \Z$ tali che $am +bm' = 1$ e quindi vale $\forall g \in G \ m(ag) + m'(bg) = g$
        \item $mG \cap m'G = \{0\}$: infatti se $g$ appartiene all'intersezione, allora necessariamente si ha $g = mx = m'y$ per degli opportuni $x,y \in G$. Allora $m'g = m'mx = nx = 0$, $mg = mm'y = ny = 0$ $\Rightarrow \text{ord}(g) \mid m,m'$ e, poiché $m,m'$ coprimi, $\text{ord}(g) = 1 \Rightarrow g = 0$.
    \end{itemize}
    Quindi per il teorema di decomposizione in prodotto diretto $G \cong mG \times m'G$. Noto ora che $mG = G_{m'} = \{x \in G : m'x = 0\}$. Infatti il contenimento $\subseteq$ è ovvio, mentre $\supseteq$ utilizza ancora $a,b$ dati da Bezout: se $x \in G_{m'}$ $x = amx + bm'x = amx = m(ax)$. Analogamente $m'G = G_m$.
    
    Da $G \cong mG \times m'G = G_{m'} \times G_m$ segue allora $\#G_m \#G_{m'} = mm'$. Poiché l'ordine di un elemento deve dividere l'ordine del sottogruppo, guardando i primi che possono dividere $\#G_m, \#G_{m'}$, necessariamente si deve avere $\#G_m=m, \#G_{m'}=m'$. Per lo stesso motivo e usando le cardinalità $\forall p_i$ divisore di $m$ si ha anche $G(p_i) = G_m(p_i)$ e analogamente con $m'$. Quindi le componenti di $p$-torsione si partizionano tra $G_m, G_{m'}$. Poiché $m,m' < \#G$, usiamo l'ipotesi induttiva e concludiamo:
    
    $G \cong G_{m'}\times G_m \cong \prod_{p_i \mid m'} G_{m'}(p_i) \times \prod_{p_i \mid m} G_{m}(p_i) = \prod_{p_i \mid m'} G(p_i) \times \prod_{p_i \mid m} G(p_i)$.
\end{proof}
\begin{corollary}{1}
    Sia $G$ abeliano finito, allora $\forall p$ primo $\#G(p) = p^r$ con $r = v_p(\#G)$.
\end{corollary}
\begin{proof}
    Basta guardare le cardinalità dei fattori nella scomposizione data dal teorema, notando che in $\#G(p)$ può e deve comparire solo $p$.
\end{proof}
\begin{corollary}{2}
    Sia $G$ abeliano finito, allora la decomposizione di $G$ come prodotto diretto di $p$-gruppi esiste ed è unica (ed è quella data dall'enunciato del teorema). 
\end{corollary}
\begin{proof}
    Le cardinalità dei $p$-gruppi sono fissate dal fatto che il loro prodotto deve essere $\#G$. A questo punto l'unica possibilità nel caso di due diverse scomposizioni (isomorfe, perché isomorfe a $G$) è che i $p$-gruppi corrispondenti allo stesso primo siano a due a due isomorfi, il che implica che sono la stessa composizione. Siano infatti $H_1, H_2$ i due $p$-gruppi in questione, relativi al primo $p$, e sia $n = \#G = p^km$ con $(m,p)=1$. Allora $H_1 \cong mG \cong H_2$ (moltiplicando ogni elementi del prodotto diretto per $m$ tutte le componenti relative ai primi divisori di $m$ diventano 0).
\end{proof}
\begin{theorem}{2}
    Sia $G$ $p$-gruppo abeliano. Allora esistono $r_1 \geq \dots \geq r_t$ univocamente determinati tali che $G \cong \Z/p^{r_1}\Z \times \dots \times \Z/p^{r_t}\Z$.
\end{theorem}
\begin{proof}
    Sia $\#G = p^n$ e procediamo per induzione su $n$. Il passo base è chiaro.
    Per il passo induttivo, sia $ x_1 \in G$ di ordine massimo, $\text{ord}(x_1) = p^{r_1}$. Se $r_1 = n$ $G \cong \Z/p^n\Z$ e si ha la tesi. Altrimenti consideriamo $G/\langle x_1 \rangle$. Esso è un $p$-gruppo non banale di cardinalità strettamente inferiore a $p^n$, e posso dunque applicargli l'ipotesi induttiva, ottenendo (prendo i generatori dei gruppi ciclici) $\Z/p^{r_2}\Z \times \dots \times \Z/p^{r_t}\Z \cong G/\langle x_1 \rangle = \langle \overline x_2 \rangle \langle \overline x_3 \rangle \dots  \langle \overline x_t \rangle $ con $\text{ord}(x_i) = p^{r_i}$ e $r_2 \geq r_3 \geq \dots \geq r_t$.

    \textbf{lemma:} $\forall \overline x \in G/\langle x_1 \rangle$ $\exists x \in \pi^{-1}(\overline x)$ tale che $\text{ord}(x) = \text{ord}(\overline x)$.
    \begin{proof}
        Sia $y \in \pi^{-1}(\overline x)$: cerchiamo tale elemento in $y + \langle x_1 \rangle = \{y + a x_1 \mid a \in \mathbb{Z}\} = \pi^{-1}(\overline x)$. Sia $p^r = \text{ord}(\overline x)$, $r \leq r_1$ perché per omomorfismo e per massimalità di $r_1$ si ha $p^r = \text{ord}(\overline x) \mid \text{ord}(y) \mid p^{r_1}$. Analogamente anche $\forall a \in \mathbb{Z} \ p^r = \text{ord}(\overline x) \mid \text{ord}(y + a x_1)$, dunque $\text{ord}(y + a x_1) = p^r \Leftrightarrow p^r(y + a x_1) = 0$. Abbiamo $0 = \pi(p^r y) \Rightarrow p^r y \in \ker(\pi) = \langle \overline x_1 \rangle$, dunque $\exists b \in \Z \ p^ry = bx_1$. Sappiamo inoltre $0 = p^{r_1}y = p^{r_1-r}(p^r y) = p^{r_1-r}(b x_1) \Rightarrow p^{r_1} = \text{ord}(x_1) \mid p^{r_1-r}b \Rightarrow \exists c \in \Z \ b = p^rc$. Allora $y-cx_1$ è l'elemento cercato, infatti $p^r(y-cx_1) = p^ry - p^rcx_1 = b x_1 - bx_1 = 0$.
     \end{proof}
    Consideriamo $x_2, \dots, x_t$ dati dal lemma per $\overline x_2, \dots , \overline x_t$. Sia $H = \langle x_2,\dots, x_t \rangle$. Dimostriamo $\pi|_H$ isomorfismo. È chiaramente suriettiva perché $\pi|_H(x_i) = \overline x_i$ per $i=2,\dots,t$. Inoltre:
    \begin{align}
    \ker(\pi|_H) & = \{h \in H : \pi(h) = 0 \} =  \nonumber \\
    & = \{a_2 x_2+\dots+a_t x_t \in H :  \overline 0 = \pi(h) = a_2\overline x_2+\dots+a_t\overline x_t \}  =  \nonumber \\
    & =  \{a_2 x_2+\dots+a_t x_t \in H : a_i\overline x_i = \overline 0 \ \forall i = 2,\dots, t\}  = \nonumber \\
    & = \{a_2 x_2+\dots+a_t x_t \in H : p^{r_i} \mid a_i \ \forall i = 2,\dots, t\}  = \nonumber \\
    & = \{a_2 x_2+\dots+a_t x_t \in H : a_i x_i =  0 \ \forall i = 2,\dots, t\} = \{0\}. \nonumber 
    \end{align}
    Quindi $H \cong \langle \overline x_2 \rangle \dots \langle \overline x_t \rangle \cong \langle x_2 \rangle \dots  \langle  x_t \rangle$. Se dimostriamo $G \cong \langle x_1 \rangle \times H$, segue la tesi. Verifichiamo che sono soddisfatte le ipotesi del teorema di decomposizione in prodotto diretto: 
    \begin{itemize}
    \item $\langle x_1 \rangle + H  = G$: $\forall x \in G$ vale che $\pi(x) = a_2\overline x_2 + \dots + a_t \overline x_t \Rightarrow \pi(x - (a_2 x_2 + \dots + a_t  x_t)) = 0 \Rightarrow x - (a_2 x_2 + \dots + a_t  x_t) \in \langle x_1 \rangle$;
    \item $\langle x_1 \rangle \cap H = \{e\}$: basta notare che $\{0\} = \ker(\pi|_H) = \ker(\pi) \cap H = \langle x_1 \rangle \cap H.$
    \end{itemize}
    Dunque $G \cong \langle x_1 \rangle \times \dots \langle x_t \rangle$.

    Per l'unicità procediamo sempre per induzione. Nel passo base è ovvia, perché può essere solo isomorfo a $\Zp$. Per il passo induttivo supponiamo che esistano due scritture $\Z/p^{r_1}\Z \times \dots \times \Z/p^{r_t} \cong \Z/p^{q_1}\Z \times \dots \times \Z/p^{q_s}$,($r_i$ e $q_i$ crescenti). Poiché i campi sono isomorfi l'ordine massimo di un loro elemento deve essere lo stesso, e quindi  $p^{r_t} = p^{q_s} \Rightarrow r_t = q_s$. Quozientiamo allora i due gruppi per $\Z/p^{r_t}$. Otteniamo due scritture  $\Z/p^{r_1}\Z \times \dots \times \Z/p^{r_{t-1}} \cong \Z/p^{q_1}\Z \times \dots \times \Z/p^{q_{s-1}}$ che per ipotesi induttiva sono la stessa, e quindi $t = s$ e $r_i = q_i \forall i = 1,\dots,t$.
\end{proof}

\begin{theorem}{struttura dei gruppi abeliani finiti}
    Sia $G$ abeliano finito. Allora esistono univocamente determinati $n_1, \dots, n_t$ tali che $G \cong \Z/n_1\Z \times \dots \times \Z/n_t\Z$ e $n_t \mid n_{t-1} \mid \dots \mid n_1$.
\end{theorem}
\begin{proof}
    Sia $\#G = \prod_{i = 1}^s p_i^{\alpha_i}$ con $p_i$ primi. Mettiamo insieme i due teoremi appena visti.
    \begin{align}
        G & \cong G(p_1) \times \dots \times G(p_s) \cong \\
        & \cong \Z/p_1^{r_{1,1}}\Z \times \dots \times \Z/p_1^{r_{1,t_1}}\Z \times \dots \times \Z/p_s^{r_{s,1}}\Z \times \dots \times \Z/p_s^{r_{s,t_s}}\Z \cong \\
        & \cong \Z/p_1^{r_{1,1}}\Z \times \dots \times \Z/p_1^{r_{1,t}}\Z \times \dots \times \Z/p_s^{r_{s,1}}\Z \times \dots \times \Z/p_s^{r_{s,t}}\Z \cong  \\
        & \cong \Z/p_1^{r_{1,1}}\Z \times \dots \times \Z/p_s^{r_{s,1}}\Z \times \dots \times \Z/p_1^{r_{1,t}}\Z \times \dots \times \Z/p_s^{r_{s,t}}\Z \cong  \\
        & \cong \Z/n_1\Z \times \times \dots \times \Z/n_t\Z 
        \end{align}
    Motivazioni dei vari passaggi:
    \begin{itemize}
        \item[(1)] applichiamo il teorema 1
        \item[(2)] applichiamo il teorema 2, e chiamiamo $r_{i,j}$ gli esponenti ottenuti per il primo $p_i$, $1 \leq j \leq t_i$; si ha quindi $r_{i,1} \geq \dots \geq r_{i,t_i} \forall i = 1,\dots, s$
        \item[(3)] imponiamo wlog che le scritture abbiano tutte la stessa lunghezza, estendendole eventualmente alla massima, che indichiamo con $t$, con dei gruppi banali ($\forall i = 1,\dots,s \ r_{i,j} = 0 \ \forall j > t_i$)
        \item[(4)] riarrangiamo i termini
        \item[(5)] applichiamo TCR raggruppando blocchi di termini coprimi: $n_k = \prod_{i = 1}^s p_i^{r_{i,k}}$ e quindi, dal fatto che gli $r_{i,k}$ sono ordinati (rispetto a $k$) in senso decrescente, si ha $n_t \mid n_{t-1},  \dots, n_2 \mid n_1$
    \end{itemize}
    Per l'unicità ripercorriamo i passaggi al contrario, sfruttando il fatto che si ha unicità nei teoremi 1 e 2.
\end{proof} 

\hypertarget{gruppi finiti}{
\subsection{Fatti utili sui gruppi finiti}
}
\hypertarget{es1}{
\begin{proposition}{gruppo finito non è unione di sottogruppi coniugati}
    Sia $G$ finito e $H <G$ sottogruppo. Allora $\bigcup_{g \in G} gHg^{-1} = G \iff H = G$
\end{proposition}
}
\begin{proof}
    La freccia ``$\Leftarrow$'' è ovvia. Si nota che l'unione è un'unione di $\# G $ gruppi, che però non sono necessariamente tutti distinti. In particolare:
    \[
    g_1 H g_1^{-1} = g_2 H g_2^{-1} \iff g_2^{-1}g_1 H g_1^{-1}g_2 = H \iff g_2^{-1}g_1 \in N_G(H).
    \]
    Quindi ogni gruppo compare $\# N_G(H)$ volte, da cui segue che i gruppi distinti sono $n = \#G / \#N_G(H)$. Potremmo notare che se vale $N_G(H) \gneq H $ abbiamo già chiuso per cardinalità. Tuttavia, possiamo raffinare la stima: ciascun gruppo $gHg^{-1}$ è in bigezione naturale con $H$ e quindi ha $\#H$ elementi. Ma essendo gruppi, certamente tutti contengono l'identità. Quindi gli elementi distinti nell'unione $\bigcup_{g \in G} gHg^{-1}$ sono al più $n\cdot(\#H -1) + 1$, ovvero:
    \[
    \#\bigcup_{g \in G} gHg^{-1} \leq \#G / \#N_G(H) \cdot (\#H -1) + 1 \leq \#G / \#H \cdot (\#H -1) + 1 = \#G - \#G / \#H + 1,
    \]
    che è minore di $\#G$ se $H \neq G$. 
\end{proof}
\begin{proposition}{centro di un $p$-gruppo}
    Sia $G$ tale che $\#G = p^n$, con $p$ primo. Allora $Z(G) \neq \{e\}$.
\end{proposition}
\begin{proof}
    Consideriamo la formula delle classi modulo $p$. Se fosse $\#Z(G) = 1$ allora $\sum \frac{\#G}{\#Z_G(g)} = \sum \frac{p^n}{\#Z_G(g)}$ non sarebbe divisibile per $p$, e quindi almeno uno dei termini dovrebbe non essere divisibile per $p$. L'unica possibilità è che si abbia $\frac{\#G}{\#Z_G(g)} = 1$ per un qualche $g$, ossia $Z_G(g) = G$, assurdo perché $g \notin Z(G)$.
\end{proof}
\begin{proposition}{gruppi di ordine $p^2$}
    Sia $G$ con $\#G = p^2$, con $p$ primo. Allora $G \cong \Z/p^2\Z$ oppure $G \cong \Zp \times \Zp$. In particolare $G$ è abeliano.
\end{proposition}
\begin{proof}
    Dimostriamo innanzitutto che $G$ è abeliano. Le possibili cardinalità di $Z(G)$ sono solo $1, p, p^2$. $\#Z(G) = 1$ è esclusa dalla proposizione precedente. Non può essere neanche $\#Z(G) = p$, infatti si avrebbe $[G : Z(G)] = p \Rightarrow G/Z(G)$ ciclico $\Rightarrow$ $G$ abeliano $\Rightarrow$ $\#Z(G) = p^2$, contro l'ipotesi $\#Z(G) = p$. Necessariamente allora $\#Z(G) = p^2$, i.e. $G$ è abeliano.
    
    Distinguiamo ora due casi: se esiste un elemento di ordine $p^2$ allora $G$ è ciclico, se invece un tale elemento non esiste, per il teorema di Lagrange tutti gli elementi eccetto il neutro hanno ordine $p$. Sia $x$ un tale elemento e $y \in G \setminus \langle x \rangle$. $\langle x \rangle\cap \langle y \rangle = \{e\}$ perché se avessero in comune un elemento $z \neq e$ di ordine $p$ si avrebbe l'assurdo $\langle x \rangle = \langle z \rangle = \langle y \rangle$. Poiché $G$ è abeliano, sia $\langle x \rangle$ che $\langle y \rangle$ sono normali in $G$. Dal teorema di decomposizione in prodotto diretto segue $G \cong \langle x \rangle \times \langle y \rangle \cong \Zp \times \Zp$.
\end{proof}
\begin{proposition}{gruppi di ordine $pq$}
    Sia $G$ tale che $\#G = pq$, con $p < q$ primi. Se $p \nmid q-1$ allora necessariamente $G \cong \Z/pq\Z$, altrimenti a questa possibilità si aggiunge $G \cong  \Zq \rtimes \Zp$.
\end{proposition}
\begin{proof}
    Per il teorema di Cauchy $\exists x,y \in G \ \text{ord}(x) = q, \text{ord}(y) = p$. Se per assurdo esistesse $z \in G \setminus \grp{x}$ di ordine $q$, allora avrei $\grp{x} \cap \grp{z} = \{e\}$ e quindi l'assurdo $\# \grp{x}\grp{z} = q^2 > pq$. Dunque $\grp{x}$ è l'unico sottogruppo di ordine $q$, quindi è caratteristico e in particolare normale.
    $\grp{x} \cap \grp{y} = \{e\}$ perché l'ordine di un elemento nell'intersezione divide sia $p$ che $q$, che sono coprimi. Segue $\#\grp{x}\grp{y} = pq$, cioè $G = \grp{x}\grp{y}$. Per il teorema di decomposizione in prodotto semidiretto si ha allora $G \cong \langle x \rangle \rtimes_{\varphi} \langle y \rangle$ dove $\varphi$ è l'azione per coniugio di $\langle y \rangle$ su $\langle x \rangle$,
    quindi un omomorfismo $\varphi : \langle y \rangle \rightarrow \Aut(\grp{x})$ che corrisponde a un omomorfismo $f : \Zp \to \Z/(q-1)\Z \ (\cong \Aut(\Zq))$.
    Le possibili cardinalità dell'immagine di $\varphi$ sono $1$ o $p$, ma per Lagrange l'immagine può avere cardinalità $p$ solo se $p \mid q-1$. Quindi:
    \begin{itemize}
        \item se $p \nmid q-1$ l'unica possibilità è l'omomorfismo banale, nel cui caso il prodotto è diretto;
        \item se $p \mid q-1$ esistono anche omomorfismi $f$ non banali, ognuno univocamente determinato dall'immagine di $1$. $\text{ord}f(1) = p \Rightarrow f(1) = k \frac{q-1}{p}$ per un $k \in \{1, \dots, p-1\}$. Quindi, detta $f^{(k)}$ la funzione $f^{(k)}(1 \mapsto k \frac{q-1}{p})$, vale $f = f^{(k)}$ per qualche $k$. Data $\beta \in \Aut(\Zp)$ definita da $\beta(1 \mapsto k^{-1})$ vale $\forall i \in \Zp \ (f^{(k)} \circ \beta)(i) = f^{(k)}(i \beta(1)) = i\beta(1)f^{(k)}(1) = i\beta(1)k\frac{q-1}{p} = i (ap + 1) \frac{q-1}{p} = i \frac{q-1}{q} = f^{(1)}(i)$, dunque $f^{(k)} \circ \beta = f^{(1)}$. Per il lemma sull'isomorfismo di prodotti semidiretti si ha $\Zq \rtimes_{f^{(1)}} \Zp \cong \Zq \rtimes_{f^{(k)}} \Zp$, dunque il prodotto semidiretto non banale $\langle x \rangle \rtimes_\varphi \langle y \rangle$ è unico a meno di isomorfismo.
    \end{itemize}
\end{proof}
\begin{proposition}{gruppi di ordine $2d$}
    Sia $G$ tale che $\#G = 2d$, con $d$ dispari. Allora $G$ ha un sottogruppo di indice 2 (che quindi è normale in $G$).
\end{proposition}
\begin{proof}
    Per il teorema di Cauchy $\exists x \in G \ \text{ord}(x) = 2$. Consideriamo l'immersione del teorema di Cayley, $f: G \hookrightarrow S_{2d}$. Sia $H = f^{-1}(A_{2d}) = f^{-1}(A_{2d} \cap f(G))$. Per il teorema di corrispondenza, $[G : H] = [f(G) : A_{2d} \cap f(G)]$, dove il secondo termine può essere solo 1 o 2. Infatti $A_{2d} \cap f(G) = \ker(\sgn\mid_{f(G)})$ e quindi $f(G)/(A_{2d} \cap f(G)) \cong \imm(\sgn\mid_{f(G)}) \subseteq\{\pm 1\}$, da cui segue $[f(G) : A_{2d} \cap f(G)] \leq 2$.
    Segue dalle proprietà viste dell'embedding di Cayley che se $\text{ord}(x) = 2$ allora $f(x)$ è una permutazione formata da $d$ 2-cicli, quindi in particolare $f(x)$ ha segno dispari e $f(x) \not \in A_{2d}$. Quindi $H \neq G \Rightarrow [G:H] = 2$.
    Alternativamente, per ogni $H < S_n$ se esiste $\sigma \in H \ \sgn(\sigma) = -1$, allora $\tau \mapsto \sigma \circ \tau$ è una bigezione tra gli elementi pari e gli elementi dispari di $H$.
\end{proof}
\begin{proposition}{condizione sufficiente per normalità}
    Sia $G$ finito. Se $H < G$ ha indice il più piccolo primo che divide $\#G$, allora $H$ è normale.
\end{proposition}
\begin{proof}
    Considero l'azione $\varphi$ di $G$ sull'insieme $X = \set{gH : g \in G}$ data dalla moltiplicazione a sinistra. Per definizione si ha $\# X = p $. Ricordiamo che un'azione è definita come un omomorfismo $\varphi: G \rightarrow S(X) \cong S_p$. Sia $K$ il suo nucleo, voglio dire $K = H$. Per il primo teorema di omomorfismo, $G/K \cong \imm(\phi) < S_p$, quindi $\#(G/K) \mid p!$ e chiaramente $\#(G/K) \mid \#G$. Segue $\#(G/K) \mid MCD(\# G, p!) = p$, dove l'ultima uguaglianza segue dal fatto che $p$ è il \textit{minimo} primo che divide $\#G$. Non può essere $K = G$ poiché $\forall g \in G \ gH = H \Rightarrow H = G$, contro l'ipotesi. Necessariamente allora $\#(G/K) = p$. $H \leq K$ implica $p = \#(G/K) \leq \#(G/H) = p$, da cui $K = H$.

    \underline{seconda dim:}
    Vale $H \tri G \Leftrightarrow \forall g \in G, h \in H \ hgH = gH$. Considero l'azione $\varphi$ di $H$ sull'insieme $X = \set{gH : g \in G} \setminus \{ eH \}$ data dalla moltiplicazione a sinistra, per definizione si ha $\# X = p - 1$.
    La moltiplicazione per elementi di $H$ è una bigezione di $G/H$ che fissa $eH$, quindi l'azione è ben definita. Ricordiamo che un'azione è un omomorfismo $\varphi: H \rightarrow S(X) \cong S_{p-1}$. Per quanto detto, $H \tri G \Leftrightarrow \imm(\varphi) = \{ id_X \}$. Si ha $\# \imm(\varphi) \mid \#S_{p-1} = (p-1)!$ e $\# \imm(\varphi) \mid \# H$, ma allora $\#\imm(\varphi) \mid MCD(\#H, (p-1)!) = 1$, cioè $\imm(\varphi) = \{id_X\}$.
\end{proof}
\begin{proposition}{sottogruppo normale contenuto nel centro}
    Sia $G$ finito. Se $H \tri G$ ha ordine $p$ il più piccolo primo che divide $\#G$, allora $H$ è contenuto nel centro.
\end{proposition}
\begin{proof}
    Nello stesso spirito della dimostrazione precedente, considero l'azione per coniugio di $G$ su $X = H \setminus \{ e \}$, ben definita per normalità di $H$ e poiché $ghg^{-1} = e \Leftrightarrow h = e$. L'azione è un omomorfismo $\varphi : G \to S(X) \cong S_{p-1}$. $H \subset Z(G)$ se e solo se l'immagine di $\varphi$ è banale, ma ciò è sicuramente verificato poiché $\#\imm(\varphi) \mid MCD(\#G, \#S_{p-1}) = 1$.
\end{proof}
\begin{definition}{sottogruppo derivato/dei commutatori}
    il commutatore di due elementi $x,y \in G$ si indica con $[x,y]:= xyx^{-1}y^{-1}$. Il sottogruppo generato da tutti i commutatori si indica con $G'$. \\
\end{definition}
\begin{proposition}{proprietà del sottogruppo derivato}
    Valgono le seguenti:
    \begin{enumerate}
        \item è caratteristico (e quindi anche normale)
        \begin{proof}
            Basta osservare che per ogni omomorfismo $f$ con dominio $G$ si ha $f([x,y]) = [f(x),f(y)] \ \forall x,y \in G$. Quindi un qualsiasi automorfismo manda l'insieme dei commutatori in sé stesso, e di conseguenza $G'$ in sé stesso.
        \end{proof}
        
        \item $G/G'$ è abeliano (tale gruppo è detto l'\textit{abelianizzato} di $G$).
        \begin{proof}
            $\forall g,h \in G$ si ha $gG' \cdot hG' =  hG' \cdot gG' \iff gh G' = hgG' \iff ghg^{-1}h^{-1} \in G'$ che è chiaro perché è un commutatore.
        \end{proof}
        
        \item $\varphi: G \rightarrow H$ omomorfismo con $H$ abeliano $\Rightarrow G' \subset \ker(\varphi)$
        \begin{proof}
            Basta osservare che se $H$ abeliano
            \[
                f([x,y]) = [f(x),f(y)] = f(x)f(y)f(x)^{-1}f(y)^{-1} = f(x)f(x)^{-1}f(y)f(y)^{-1} = e_H \quad \forall x,y \in G.
            \]
            Quindi i commutatori sono tutti nel nucleo e di conseguenza anche $G'$.
        \end{proof}
        
        \item $H$ abeliano $\Rightarrow$ $\Hom(G,H)$ e $\Hom(G/G',H)$ sono in bigezione.
        \begin{proof}
            Costruiamo le sue corrispondenze come segue. 
            \begin{itemize}
            \item $\varphi \in \Hom(G,H)$ la mandiamo in $\tilde \varphi \in \Hom(G/G',H)$ data dal $1^{\circ}$ teorema di omomorfismo (visto che $G' \subseteq \ker(\varphi)$). Segue dall'unicità nel teorema che sono tutte distinte e quindi questa mappa è iniettiva.
            \item $\varphi \in \Hom(G,H)$ la mandiamo in $\varphi \circ \pi \in \Hom(G/G',H)$ dove $\pi$ è la proiezione $\pi: G \rightarrow G/G'$. Anche questa mappa è chiaramente iniettiva, visto che $G' \subseteq \ker(\varphi)$ e quindi se due mappe vengono mandate nella stessa allora coincidono anche su tutto $G$.
            \end{itemize}
        \end{proof}
        \end{enumerate}    
\end{proposition}

\begin{proposition}{prodotti diretti belli}
    Se $G \cong H \times K$ con $H,K$ finiti tali che $(\#H, \#K) = 1$ allora $\{e_H\} \times K$ e $H \times \{e_K\}$ sono caratteristici in $G$.
\end{proposition}
\begin{proof}
    Dimostriamo che $\{e_H\} \times K$ e $H \times \{e_K\}$ sono gli unici sottogruppi delle rispettive cardinalità, quindi caratteristici. Basta dire che nelle ipotesi per ogni sottogruppo $L \le G$ vale $L = \grp{\pi_H(L) \times \{e_K\}, \{e_H\} \times \pi_K(L)}$ e quindi $\#L = \#\pi_H(L) \#\pi_K(L)$. Vale sempre $L \subseteq \grp{\pi_H(L) \times \{e_K\}, \{e_H\} \times \pi_K(L)}$. L'altro contenimento segue da $(\#H, \#K) = 1$, infatti per Bezout $\forall (h, k) \in L \ \exists a, b \in \N \ (h, k)^a = (h, e_K)$ e $(h, k)^b = (e_H, k)$.
\end{proof}
\begin{proposition}{automorfismi in un prodotto diretto}
    Se $G \cong H \times K$ e $\{e_H\} \times K$ e $H \times \{e_K\}$ sono caratteristici in $G$, allora $\Aut(G) \cong \Aut(H) \times \Aut(K)$. 
\end{proposition}
\begin{proof}
    Per ogni $\varphi \in \Aut(G)$ sia $\varphi_H \in \Aut(H)$ definito mediante $\varphi_H(x) := \pi_H \circ \varphi (x, e_K)$ (chiaramente è un automorfismo) e analogamente $\varphi_K \in \Aut(K)$. \\
    Consideriamo allora $\Phi: \Aut(G) \rightarrow \Aut(H) \times \Aut(K)$ dato da $\Phi(\varphi) = (\varphi_H, \varphi_K)$. 
\begin{itemize}
    \item è omomorfismo: basta notare che $(\psi \circ \varphi)_H = \psi_H \circ \varphi_H$;
    \item è iniettivo: $\Phi(\varphi)=(id_H,id_K) \Rightarrow \forall (h, k) \in G \ \varphi((h, k)) = (h, k) \Rightarrow \varphi = id_G$.
\end{itemize}
\end{proof} 
\begin{proposition}{(*) centro di un prodotto semidiretto}
    Sia $G\cong H \rtimes_{\varphi} K$ con $H$ abeliano. Allora
    \[
    Z(G) \cong \left( \bigcap_{k \in K} Fix(\varphi_k) \right) \times (\ker(\varphi) \cap Z(K)).
    \]
\end{proposition}
\begin{proof}
    Un elemento $(a, b)$ sta nel centro di $G$ se e solo se commuta con gli elementi dell'insieme di generatori $H \times \{e_K\} \cup \{e_H\} \times K$:
    \begin{itemize}
        \item $(a, b)(e_H, b') = (a \varphi_b(e_H), b b') = (a, b b')$ e $(e_H, b')(a, b) = (\varphi_{b'}(a), b' b)$ coincidono per ogni $b' \in K$ se e solo se $b \in Z(K)$ e $a \in Fix(\varphi_{b'})$;
        \item $(a, b)(a', e_K) = (a \varphi_b(a'), b)$ e $(a', e_K)(a, b) = (a' a, b) = (a a', b)$ coincidono per ogni $a' \in H$ se e solo se $b \in \ker(\varphi)$.
    \end{itemize}
    Quindi $(h,k) \in Z(G) \Leftrightarrow h \in \bigcap_{k \in K} Fix(\varphi_k) \land k \in \ker(\varphi) \cap Z(K)$, come voluto.
\end{proof}
\begin{proposition}{(*) intersezione dei $p$-Sylow con il centro}
    L'intersezione di un $p$-Sylow con il centro non dipende dal $p$-Sylow scelto.
\end{proposition}
\begin{proof}
    Sia $S$ un fissato $p$-Sylow. Ricordiamo che i $p$-Sylow sono tutti coniugati, quindi dato $Q$ un qualsiasi $p$-Sylow $\exists g \in G$ $Q = gSg^{-1}$. Poiché ogni elemento del centro è invariante per qualsiasi coniugio si ha: $Q \cap Z(G) = gSg^{-1} \cap Z(G) = gSg^{-1} \cap gZ(G)g^{-1} = g(S \cap Z(G))g^{-1} = S \cap Z(G)$. 
\end{proof} 

\hypertarget{diedrale}{\subsection{Il gruppo diedrale}}
    Il gruppo diedrale $D_n$ è il gruppo delle isometrie di un fissato $n$-agono regolare -- diciamo quello inscritto nella circonferenza unitaria e con un vertice in $(1,0)$ -- con l'operazione di composizione. Chiamiamo $r$ la rotazione di $2\pi/n$ in senso antiorario e $s$ la simmetria rispetto all'asse $x$. Le $n$ rotazioni sono multiple di $r$ ed elementi di $D_n$, così come sono elementi di $D_n$ anche le $n$ simmetrie relative all'asse origine-vertice al variare dei vertici. Questi sono $2n$ elementi distinti, da cui $\#D_n \geq 2n$. Assicuriamoci siano tutti e soli gli elementi del diedrale mostrando $\#D_n = 2n$.
\begin{proof}
    Chiamiamo i vertici, in senso antiorario a partire da $(1,0)$, $V_1, V_2, \dots, V_n$ e sia $\sigma \in D_n$. Poiché $\sigma$ è isometria, manda vertici in vertici, e inoltre una volta fissata l'immagine di $V_1,V_2$, è tutto univocamente determinato. Per $V_1$ abbiamo $n$ scelte (tutti i vertici), per $V_2$ ne abbiamo 2 (i due vicini di $\sigma(V_1)$), da cui $\#D_n = 2n$. 
\end{proof}
Notiamo ora che $sr^k$ è un'altra simmetria.
\begin{proof}
    Un'isometria è in particolare un'applicazione lineare. Consideriamo le matrici $2\times 2$ $R$ relativa a $r$ e $S$ relativa a $s$. Si nota che un'isometria in $D_n$ è una simmetria assiale se e solo se la sua matrice ha determinante $-1$ e una rotazione se e solo se la sua matrice ha determinante $1$. La tesi segue allora dalla moltiplicatività del determinante. 
\end{proof}
    Allora $s, sr, sr^2, \dots sr^{n-1}$ sono $n$ simmetrie distinte, quindi tutte (e sole) quelle descritte prima. Segue
    \[
        D_n = \langle r, s \rangle.
    \]
    Per lavorare con il diedrale è fondamentale la seguente relazione: $srs = r^{-1}$
\begin{proof}
    $r$ manda $V_i$ in $V_{i+1}$, $s$ manda $V_i$ in $V_{n+1-i}$ (guardando gli indici modulo $n$) e quindi $srs$ manda $V_1 \mapsto V_1 \mapsto V_2 \mapsto V_{n-1}$, $V_2 \mapsto V_{n-1} \mapsto V_1 \mapsto V_1$,  da cui la tesi.
\end{proof}
    La presentazione di $D_n$ è $\langle x,y \mid x^n = id, y^2 = id, yxyx = id\rangle$.
    
    Da quella relazione, con manipolazioni algebriche, si ricava 
    \[ s^ar^bs^cr^d= s^{a+c} r^{(-1)^cb + d}. \]
    Questa relazione ci fa già intuire la struttura di prodotto semidiretto. Si dimostra infatti
    \[ D_n \cong \Zn \rtimes \Z/2\Z \]
\begin{proof}
    Il sottogruppo delle rotazioni $R=\langle r \rangle$ è normale perché ha indice 2, è chiaramente disgiunto da $\langle s \rangle$ per quanto detto precedentemente sui determinanti, e si è visto prima che $\langle r \rangle \langle s \rangle = \langle r, s \rangle = D_n$. La tesi segue allora dal teorema di decomposizione in prodotto semidiretto: $D_n \cong \langle r \rangle \rtimes \langle s\rangle \cong \Zn \rtimes \Z/2\Z$.
\end{proof}
Studiamone infine i sottogruppi. Come detto sopra, il sottogruppo delle rotazioni $R = \langle r \rangle$ è normale perché ha indice 2. Inoltre è ciclico, quindi ha esattamente un sottogruppo di ordine $d$ per ogni $d$ divisore di $n$. Tutti questi sottogruppi sono caratteristici in $R$ perché sono gli unici del proprio ordine, quindi sono normali in $D_n$. Perciò i sottogruppi generati da un qualsiasi multiplo della rotazione $r$ sono tutti normali.

Notiamo ora che se in un gruppo $H$ ci sono due simmetrie distinte, allora c'è anche una rotazione: $sr^a, sr^b \in H \Rightarrow sr^asr^b = r^{b-a} \in H$. Tutti i gruppi restanti sono allora quelli generati da una rotazione (quella di ordine massimo nel sottogruppo) e una simmetria. Sia $H = \langle r^d, sr^h \rangle$. Senza perdita di generalità possiamo supporre:
\begin{itemize}
    \item $0 \leq h < d$: se $h = md + x$ con $0 \leq x < d$ (resto) e $r^d \in H$ si ha $sr^x\in H \iff sr^h = sr^x(r^d)^m \in H$;
    \item $0 \leq d < n$, $d \mid n$: basta notare che $\grp{r^d} = \grp{r^{xd}}$ per ogni $x$ coprimo con $n$ e $d$ divisore.
\end{itemize}
Imponendo queste condizioni su $d,h$ notiamo che i gruppi trovati sono tutti distinti. In altre parole, $\langle r^{d_1}, sr^{h_1} \rangle = \langle r^{d_2}, sr^{h_2} \rangle \iff (d_1,h_1) = (d_2,h_2)$ (se $d_1,h_1$ sono come su e anche $d_2,h_2$).
\begin{proof}
    Sia $H_i = \langle r^{d_i}, sr^{h_i} \rangle$ per $i = 1,2$. Notiamo che $H_1 = H_2 \Rightarrow \grp{r^{d_1}} = H_1 \cap R = H_2 \cap R = \grp{r^{d_2}}$. Per quanto osservato prima, ciò implica $d_1 = d_2$. Si può ora notare che $H_i = \grp{sr^{h_i}}\grp{r^{d_i}}$ (basta osservare che $r^{d_i}sr^{h_i}=sr^{h_i-d} \in \grp{sr^{h_i}}\grp{r^{d_i}}$). Allora $sr^{h_2} \in H_1  \Rightarrow sr^{h_2} = (sr^{h_1})^a(r^{d_1})^b = s^ar^{ah_1+d_1}$ da cui segue che $a$ è dispari (o il membro di destra non sarebbe una simmetria) e quindi senza perdita di generalità $a =1$ (le simmetrie hanno ordine 2). Ma allora $h_1 + d_1 = ah_1 + d_1 \equiv h_2 \pmod{n}$. In particolare $h_1 \equiv h_2 \pmod{d_1}$. Ma $d_1 = d_2$ e l'ipotesi implicano che sono entrambi ridotti modulo $d_1$ e quindi non si ha solo congruenza ma uguaglianza.
\end{proof}
Ci chiediamo quando un gruppo nella forma sopra è normale. Verificando la normalità rispetto ai generatori $r,s$ si verifica che gli unici sottogruppi normali che contengono una simmetria sono $\grp{r^2, s}$, $\grp{r^2, sr}$, che sono sottogruppi propri solo nel caso in cui $n$ sia pari.

Ricapitolando, i sottogruppi di $D_n$ sono tutti e soli i seguenti:
\begin{itemize}
    \item $R = \grp{r}$ delle rotazioni, che è normale e caratteristico (non ci elementi di ordine $n$ fuori da $R\cong \Z/n\Z$);
    \item $\grp{r^d}$ con $d\mid n$, che sono caratteristici in $R$, quindi normali in $D_n$;
    \item $\grp{r^d, sr^h}$ con $d \mid n, 0\leq h < n$, che sono normali in $D_n$ solo nel caso $\grp{r^2, s}$ e $\grp{r^2, sr}$.
\end{itemize}

\begin{exercise}
    Un gruppo si dice \emph{decomponibile} se è isomorfo al prodotto diretto di gruppi non banali. Mostrare che $D_n$ è decomponibile sse $n = 2d$ con $d$ dispari.
\end{exercise}

\subsection{Il gruppo simmetrico}

Ogni permutazione $\sigma \in S_n$ si può scrivere come composizione di trasposizioni. Infatti un ciclo si può scrivere nella forma $(a_1, \dots, a_k) = (a_1, a_2)(a_2,a_3)\dots(a_{k-1},a_k)$ e ogni permutazione può essere rappresentata come prodotto di cicli disgiunti. Dunque le trasposizioni generano $S_n$.

\begin{proposition}{classi di coniugio}
    Data $\sigma \in S_n$, $Cl(\sigma)$ è l'insieme delle permutazioni di $S_n$ con la stessa struttura ciclica di $\sigma$.
    
    Ad esempio, $Cl((1,2)) = \set{\text{trasposizioni}}$ e $Cl((1,2)(3,4,5)) = \set{\text{2+3-cicli}}$.
\end{proposition}
\begin{proof}
    Fissiamo $\sigma = (a^1_1, \dots, a^1_{k_1})(a^2_1, \dots, a^2_{k_2}) \dots (a^c_1, \dots, a^c_{k_c})\in S_n$ e $X$ insieme delle permutazioni di $S_n$ con la stessa struttura in cicli di $\sigma$. \\
    Si mostra facilmente che $\forall \tau \in S_n$, $\tau\circ\sigma\tau^{-1} = (\tau{a^1_1}, \dots, \tau{a^1_{k_1}})(\tau{a^2_1}, \dots, \tau{a^2_{k_2}}) \dots (\tau{a^c_1}, \dots, \tau{a^c_{k_c}}) \in X$. Quindi $Cl(\sigma)\subseteq X$. \\
    Sia $\sigma ' = (b^1_1, \dots, b^1_{k_1})(b^2_1, \dots, b^2_{k_2}) \dots (b^c_1, \dots, b^c_{k_c}) \in X$; notiamo che $\tau = (a^1_1,b^1_1)\circ\dots \circ (a^c_{k_c},b^c_{k_c})$ è tale che $\tau\circ \sigma\circ \tau^{-1} = \sigma'$. Quindi $\sigma' \subseteq Cl(\sigma)$. Facendo variare $\sigma'$, $X \subseteq Cl(\sigma)$. 
\end{proof}

\begin{proposition}{cardinalità di un centralizzatore}
    Se $\sigma \in S_n$ è formata da $a_i \geq 0$ $i$-cicli $\forall i = 1,\dots,n$, vale che \[ \#Z_{S_n}(\sigma) = \prod_{i=1}^n (a_i!\cdot i^{a_i}). \]
    Ciò ci può aiutare a capire come sono fatti dei centralizzatori semplici oppure, combinato con il lemma normalizzatore-centralizzatore, dei centralizzatori.
    
    Generalmente si cerca di costruire ``indovinando'' il centralizzatore e poi si dice che è quello per cardinalità. Ad esempio il centralizzatore di un $k$-ciclo ($a_i = 1$ se $i = k$, $a_i = n-k$ se $i = 1$, $a_i = 0$ altrimenti) secondo la formula ha cardinalità $k \cdot (n-k)!$. Poiché sia il sottogruppo generato dal $k$-ciclo che il sottogruppo di $S_n$ che permuta gli elementi che non compaiono nel $k$-ciclo sono banalmente nel centralizzatore (e l'intersezione di questi due sottogruppi è banale) per cardinalità si conclude che il centralizzatore del ciclo è proprio il prodotto di questi due sottogruppi. 
\end{proposition}
\begin{proof}
    Abbiamo infatti visto che per il lemma orbita-stabilizzatore vale $\#Z_{S_n} = \frac{\#S_n}{Cl(\sigma)}$. Sappiamo che $Cl(\sigma)$ è l'insieme di tutte e sole le permutazioni con $a_i$ (fissato da $\sigma$) $i$-cicli. Dobbiamo contare quante sono tali permutazioni. Il conto è puramente combinatorico e si può fare in vari modi, eccone uno: 
    \begin{enumerate}
        \item mettiamo in fila in ordine: il primo blocco da 1 casella, il secondo blocco da 1 casella, $\dots$ il $a_1$-esimo blocco da 1 casella, il primo blocco da 2 caselle, $\dots$, il $a_2$-esimo blocco da 2 caselle, $\dots$, il $a_k$-esimo blocco da $k$-caselle; chiaramente in totale ci sono in fila $n$ caselle;
        \item riempiamo tali caselle con i numeri da 1 a $n$ in qualche ordine ; questa operazione ci dà una permutazione in $Cl(\sigma)$ se trasformiamo i blocchi di caselle in cicli;
        \item ci sono delle ripetizioni, che si manifestano in 2 modi:
        \begin{itemize}
            \item possiamo scambiare due blocchi da $i$ caselle ottenendo la stessa scrittura in cicli; essendoci $a_i$ $i$-cicli per ogni $i$ per togliere queste ripetizioni dobbiamo dividere per $\prod_{i=1}^n a_i!$;
            \item possiamo far ciclare (attenzione: non permutare!) il contenuto di un blocco da $i$ fissato; per tale blocco i modi di ciclare sono $i$ quindi per togliere queste ripetizioni dobbiamo dividere per $\prod_{i=1}^n i^{a_i}$ (per ciascun blocco divido per $i$).
        \end{itemize}
    \end{enumerate}
    Si ha allora la tesi in quanto \[ \#Cl(\sigma) = \frac{n!}{\prod_{i=1}^n (a_i!\cdot i^{a_i})}. \]
\end{proof}
\begin{definition}{segno di una permutazione}
    $\forall \sigma \in S_n$ definiamo \[ \sgn(\sigma) := \prod_{1\leq i<j\leq n} \frac{\sigma(j)-\sigma(i)}{j-i}. \]
\end{definition}
Le permutazioni con segno 1 sono dette \emph{pari}, le altre \emph{dispari}. Si mostra che se $\tau$ è una trasposizione, allora $\sgn(\tau) = -1$ e più in generale se $\sigma$ è un $k$-ciclo, $\sgn(\sigma) = (-1)^{k+1}$.

Vale inoltre $\sgn (\sigma \circ \tau) = \sgn (\sigma)\sgn(\tau)$.
    \begin{proof}
        \begin{align*}
            \sgn(\sigma \circ \tau) & =  \prod_{1\leq i<j\leq n} \frac{\sigma \circ \tau(j)-\sigma \circ \tau(i)}{j-i} = \qquad\qquad\qquad\qquad\qquad\text{applicando $\tau^{-1}$}\\
            & = \prod_{1\leq i<j\leq n} \frac{\sigma(j)-\sigma(i)}{\tau^{-1}(j)-\tau^{-1}(i)} = \\
            & = \prod_{1\leq i<j\leq n} \frac{\sigma(j)-\sigma(i)}{j-i}\cdot \frac{j-i}{\tau^{-1}(j)-\tau^{-1}(i)} = \\
            & = \prod_{1\leq i<j\leq n} \frac{\sigma(j)-\sigma(i)}{j-i} \prod_{1\leq i<j\leq n}  \frac{j-i}{\tau^{-1}(j)-\tau^{-1}(i)} = \qquad \text{applicando $\tau$ nel secondo fattore}\\
            & = \prod_{1\leq i<j\leq n} \frac{\sigma(j)-\sigma(i)}{j-i} \prod_{1\leq i<j\leq n}  \frac{\tau(j)-\tau(i)}{j-i} = \\
            & = \sgn(\sigma)\sgn(\tau)
        \end{align*}
    \end{proof}
    
    In altre parole, $\sgn: S_n \rightarrow \set{\pm 1} \cong \Z/2\Z$ è un omomorfismo.
\begin{definition}{$A_n$}
    Chiamiamo \emph{sottogruppo alternante} il sottogruppo $A_n = \{\sigma \in S_n : \sgn (\sigma) = 1\} = \ker(\sgn) \tri S_n$.
\end{definition}
\begin{proposition}{intersezioni con $A_n$}
    Se $H < S_n$ vale $[H : H \cap A_n] \in \{1, 2\}$.
\end{proposition}
\begin{proof}
    Se $H \subseteq A_n$ è chiaro che $[H : H \cap A_n] = 1$. Supponiamo allora che $H$ contenga un ciclo dispari $\sigma$. 
    Considero l'azione per moltiplicazione a sinistra di $H$ sulle classi laterali di $H \cap A_n$. Per quanto già visto $[H : H \cap A_n] = \#\text{orb}(H \cap A_n) = \#\set{g(H \cap A_n)  \mid  g \in H}$.

    Per ciascuna permutazione dispari $\sigma \in H$ $\sigma(H \cap A_n)$ contiene solo permutazioni dispari ed è quindi disgiunta da $H \cap A_n$, da cui segue che l'orbita ha cardinalità almeno 2. Ma due permutazioni $\sigma_1,\sigma_2$ con lo stesso segno sono tali che $\sigma_1(H \cap A_n) = \sigma_2(H \cap A_n)$. Infatti ciò vale $\iff \sigma_2^{-1}\sigma_1 \in H \cap A_n$ banalmente vero. Quindi l'orbita ha cardinalità 2 da cui la tesi.
\end{proof}
\begin{proposition}{classi di coniugio in $A_n$}
    Sia $\sigma \in A_n$ e $Cl_{A_n}(\sigma)$ la sua classe di coniugio in $A_n$. Allora $\#Cl_{A_n}(\sigma)$ è uguale a $\#Cl_{S_n}(\sigma)$ oppure a $\frac{1}{2}\#Cl_{S_n}(\sigma)$. Vale inoltre $\#Cl_{S_n}(\sigma) = \#Cl_{A_n}(\sigma) \iff$ $\sigma$ ha almeno un ciclo pari oppure 2 cicli dispari di uguale lunghezza (gli elementi fissati sono cicli di lunghezza 1).
\end{proposition}
\begin{proof}
    Sia $j=[Z_{S_n}(\sigma): (Z_{S_n}(\sigma) \cap A_n)] \in \{1, 2\}$. Un'applicazione diretta del lemma orbita-stabilizzatore mostra $\#Cl_{A_n}(\sigma) = \#Cl_{S_n}(\sigma)$ se $j = 2$, $\#Cl_{A_n}(\sigma) = \frac12\#Cl_{S_n}(\sigma)$ altrimenti. 
    
    Dimostriamo la seconda parte dell'enunciato. Ci basta dire $Z_{S_n}(\sigma) \subseteq A_n \iff \sigma$ è prodotto di cicli disgiunti, tutti dispari e di lunghezze diverse. $(\implies)$ per contronominale. Se $\sigma$ ha un ciclo pari $\tau$, allora $\tau \in Z_{S_n}(\sigma)$ e $\tau \notin A_n$. Se invece $\sigma$ ha due cicli dispari $(a, b, \dots, k), (\alpha, \beta, \dots, \kappa)$ di uguale lunghezza, allora commuta con la permutazione $(a, \alpha)(b, \beta) \dots (k, \kappa)$ che li scambia, prodotto di un numero dispari di trasposizioni, quindi non un elemento di $A_n$. In entrambi i casi $Z_{S_n}(\sigma) \not\subseteq A_n$.
    $(\impliedby)$ Se $\sigma = \tau_1 \tau_2 \dots \tau_k$ è prodotto di cicli dispari di lunghezze $d_1, d_2, \dots, d_k$ distinte, allora il coniugio per una permutazione che commuta con $\sigma$ coniuga ogni ciclo in se stesso, quindi elementi in cicli distinti ``non si mescolano''. Ma allora (dopo opportuna identificazione degli elementi di $\tau_i$ con $\{1, \dots, d_i\}$) il centralizzatore di $\sigma$ in $S_n$ è il generato dai centralizzatori $\{ Z_{S_{d_i}}(\tau_i) \}_{i=1\dots k}$. Per orbita-stabilizzatore il centralizzatore di un $d$-diclo in $S_d$ è il suo generato, quindi per $d$ dispari gli elementi del centralizzatore sono permutazioni pari. Nel nostro caso quindi $Z_{S_n}(\sigma) \subseteq A_n$.
\end{proof}
\begin{proposition}{generatori di $A_n$}
    se $n\geq 3$ i $3$-cicli generano $A_n$; se $n\geq 5$ i $2+2$-cicli generano $A_n$.
\end{proposition}
\begin{proof}
    Ogni permutazione $\sigma$ si scrive come prodotto di trasposizioni, se inoltre $\sigma \in A_n$, allora $\sigma$ si esprime come prodotto di un numero \emph{pari} di trasposizioni. Basta allora mostrare che i 3-cicli generano $\set{(a,b)(c,d) : a \neq b, c\neq d, 1\leq a,b,c,d \leq n}$ visto che ogni elemento di $A_n$ si scrive come composizione di un certo numero di elementi di questo tipo. Basta notare $(a, c, d)(a, b, d) = (a,b)(c,d)$.
    
    Se $n \ge 5$, allora i 2+2-cicli generano i 3-cicli e quindi tutto $A_n$. Si nota infatti $(a, b, c) = (a, b)(b, c) = (a b)(x y) \circ (x y)(b, c)$ per $x, y \neq a, b, c$ (da cui la necessità di $n \ge 5$).
\end{proof}
Con tutti i lemmi appena visti, possiamo caratterizzare i sottogruppi normali di $S_n$.
\begin{example2}{sottogruppi normali di $S_3$}
    In $S_3$ i sottogruppi normali sono $\set{id}, A_3 = \grp{(1,2,3)}, S_3$.
\end{example2}
\begin{proof}
    In $S_3$ la cardinalità di un sottogruppo $H$ può essere solo $1,2,3,6$. Se $\#H = 1$ il sottogruppo è banale e se $\#H = 6$ allora $H = S_3$. Se $\#H = 2$ o $3$, essendo primi, il gruppo è ciclico (generato da un 2-ciclo oppure 3-ciclo). Se $H \cong \Z/3\Z$ allora $H$ è normale perché ha indice 2 (minimo primo che divide 6). $H \cong \Z/3\Z$ non può essere normale perché altrimenti $S_3$ sarebbe abeliano. Quindi l'unico sottogruppo normale non banale è $\grp{(1,2,3)} = A_3$\\
    Nota: si poteva anche usare che $\#S_3 = pq$ con $p,q$ primi, per dire che $S_3 \cong \Z/3\Z \rtimes \Z/2\Z$.
\end{proof}
\begin{definition}{sottogruppo di Klein}
    $K_4 := \set{id, (1,2)(3,4), (1,3)(2,4), (1,4)(2,3)}$. Si verifica facilmente che è sottogruppo normale di $S_4$ e che $K_4 \cong \Z/2\Z \times \Z/2\Z$.
\end{definition}
\begin{example2}{sottogruppi normali di $S_4$}
    In $S_4$ gli unici sottogruppi normali sono $\set{id}, K_4, A_4, S_4$, dove 
\end{example2}
\begin{proof}
    Abbiamo già visto che tutti quei sottogruppi sono normali. Sia $H \neq \set{id}$ un sottogruppo normale di $S_4$. Facciamo i casi. Si ricorda che per normalità $H$ è unione di classi di coniugio, e che le classi di coniugio in $S_4$ sono: trasposizioni, 3-cicli, 2+2-cicli, 4-cicli.
    \begin{enumerate}
        \item $H$ contiene una trasposizione: allora contiene tutte le trasposizioni che però sono generatori di $S_4 \Rightarrow H = S_4$. 
        \item $H$ contiene un 3-ciclo: allora contiene tutti i 3-cicli, che però sono generatori di $A_4$, quindi $A_4 \leq H \leq S_4$. Ma $A_4$ ha indice 2 $\Rightarrow$ $H = A_4$ oppure $S_4$ (non esistono possibilità intermedie).
        \item $H$ contiene un 2+2-ciclo: allora contiene il sottogruppo di Klein $K_4$. Se $H$ contiene un 4-ciclo allora li contiene tutti e si ha in particolare $(1,2)(3,4) \in H$, $(1,2,3,4)$ $\in H \Rightarrow (1,2)(3,4)\circ(1,2,3,4) = (2,4) \in H$ e quindi siamo nel primo caso. Se $H$ non contiene 4-cicli allora o $H = K_4$ o comunque ci riconduciamo a uno dei casi precedenti.
        \item $H$ contiene un 4-ciclo: sia esso $(a_1,a_2,a_3,a_4)$. Ma allora contiene anche $(a_1,a_2,a_3,a_4)^2 = (a_1,a_3)(a_2,a_4)$ e ci riconduciamo al caso precedente.
    \end{enumerate}
\end{proof}

\begin{proposition}{$A_n$ è semplice}
    Se $n \ge 5$ gli unici sottogruppi normali di $A_n$ sono $\{id\}$ e $A_n$.
\end{proposition}
\begin{proof}
    Dimostriamo l'enunciato per induzione. Se $n = 5$ (caso base), l'ordine di un sottogruppo normale proprio $N \tri A_5$ divide $60$. Le classi di coniugio dei 3-cicli e dei 2+2-cicli non si spezzano in $A_5$, quindi $N$ normale proprio non contiene 3-cicli e 2+2-cicli, cioè (in $A_5$) elementi di ordine $3$ o di ordine $2$, altrimenti suoi elementi genererebbero $A_5$. Per Cauchy allora l'unica possibilità è $\#N = 5$. Preso però 5-ciclo, diciamo $(1, 2, 3, 4, 5) \in N$ e il suo coniugato (in $A_n$) $(1, 5, 3, 2, 4) \in N$ si ha $(1, 2, 3, 4, 5)(1, 5, 3, 2, 4) = (1, 5, 3) \in N$: assurdo. Alternativamente avremmo potuto guardare le cardinalità delle classi di coniugio.

    \begin{lemma2}
        Sia $\sigma \in A_n$ con $n \ge 6$. Allora $\exists i \in \{1, \dots, n\}, \tilde \sigma \in A_n$ coniugata non banale di $\sigma$ tale che $\sigma(i) = \tilde \sigma(i)$.
    \end{lemma2}
    \begin{proof}
        $\sigma$ non è una trasposizione, quindi muove almeno tre elementi. Scriviamo (wlog) $\sigma = (1, 2, \dots)\dots$. Se la classe di coniugio di $\sigma$ non si spezza in $A_n$, allora si può scegliere una sua coniugata non banale in $S_n$ (quindi in $A_n$) permutando opportunamente gli elementi $3, \dots, n$. Se invece $\sigma$ è prodotto di cicli dispari di lunghezze distinte, allora necessariamente il ciclo di lunghezza massima è almeno un 5-ciclo, quindi $\sigma = (1, 2, 3, 4, 5, \dots)\dots$, nel cui caso scegliamo $\tilde \sigma = (3, 4, 5) \sigma (3, 4, 5)^{-1} = (1, 2, 4, 5, 3, \dots)\dots$.
    \end{proof}
    Sia ora $n \ge 6$ e $N \tri A_n$. Poniamo $K_i = \{\sigma \in A_n : \sigma(i) = i\} \cong A_{n - 1}$, semplice per ipotesi induttiva. Notiamo che per $n \ge 4$ ciascun $K_i$ contiene un 3-ciclo. Siano $\sigma \in N$ e $i, \tilde \sigma$ come nel lemma, per normalità di $N$ $\tilde \sigma \in N$. Ma allora $id \neq \sigma {\tilde \sigma}^{-1} \in N \cap K_i \tri K_i$. Per semplicità di $K_i$ allora $N \cap K_i = K_i$, quindi $N$ contiene un 3-ciclo, da cui $N = A_n$.
\end{proof}
% \begin{proof}
%     Useremo come lemma che se $H \tri A_n$ contiene un 3-ciclo allora $H = A_n$. Poiché i 3-cicli generano ciò è dimostrato se si mostra che $Cl_{A_n}(\text{3-ciclo}) = Cl_{S_n}(\text{3-ciclo}) = \set{3\text{-cicli}}$. Il centralizzatore di un $3$-ciclo in $S_n$ è il sottogruppo da esso generato. Essendo contenuto in $A_n$ è anche il suo centralizzatore in $A_n$. Ma allora per quanto visto sulle classi di coniugio in $A_n$ si ha la tesi. 
    
%     Mostriamo ora la semplicità per induzione su $n$. I passi base $n = 5,6$ si trattano vedendo le cardinalità delle classi di coniugio delle permutazioni pari e provando a farle sommare ad $\#A_n$ (occhio all'identità).
    
%     Per il passo induttivo supponiamo la tesi dimostrata per $n-1$ e mostriamola per $n$. Sia $H \tri A_n$. Sia per ogni $i = 1, \dots, n$ $K_i := \set{\sigma \in A_n : \sigma(i) = i} \cong A_{n-1}$. Poiché $H$ è normale in $A_n$, $H \cap K_i$ è normale in $K_i$ per ogni $i$. Ma $K_i \cong A_{n-1}$ è semplice, quindi $\forall i$ si ha $K_i \cap H = \set{id}$ oppure $K_i$.
    
%     Notiamo che ciascun $K_i$ contiene un 3-ciclo (basta $n \geq 4$), quindi se per qualche $i_0$ $K_{i_0} \cap H = K_{i_0} \Rightarrow K_{i_0} \subseteq H$ si ha necessariamente $H = A_n$ per il lemma iniziale.

%     Si allora $H \cap K_i = \set{id}$ per ogni $i$ e per assurdo $id \neq \sigma \in H$. Prendiamo un qualsiasi 3-ciclo $\tau \in A_n$. Per normalità di $H$ $\tau \sigma^{-1} \tau^{-1} \in H$, e quindi $H \ni \sigma \tau \sigma^{-1} \tau^{-1} = (\sigma \tau \sigma^{-1}) \tau^{-1}$ dove i due fattori sono 3-cicli. Questa permutazione muove allora al più 6 elementi, e quindi o è l'identità o ci dà un assurdo se $n \geq 7$ perché si dovrebbe trovare in uno dei $K_i$. Facendo variare $\tau$ possiamo sceglierla in modo che non sia l'identità e ottenere l'assurdo (basta ad esempio che $\tau$ muova $\sigma(1)$). 
% \end{proof}

\begin{proposition}{sottogruppi normali di $S_n$}
    Se $n \geq 5$ gli unici sottogruppi normali di $S_n$ sono $\{id\}, A_n, S_n$.
\end{proposition}
\begin{proof}
    È noto $A_n \tri S_n$. Sia ora $N \tri S_n$ un sottogruppo normale, allora $N \cap A_n \tri A_n$, da cui $N \cap A_n \in \{ \{id\}, A_n \}$. Se $A_n \subseteq N$ allora $N = A_n$ o $N = S_n$. Se fosse $N \cap A_n = \{id\}$, allora $N = \grp{\sigma}$ con $\sigma$ dispari di ordine $2$, ma tale $N$ sarebbe troppo piccolo per contenere tutti i coniugati di $\sigma$.
\end{proof}
\begin{corollary2}
    $A_n$ è caratteristico in $S_n$.
\end{corollary2}

\begin{proposition}{derivato di $S_n$}
    $S_n' = A_n$ per ogni $n$.    
\end{proposition}
\begin{proof}
    Per omomorfismo $\sgn([\sigma, \tau]) = \sgn(\sigma)^2\sgn(\tau)^2 = 1$ e quindi $S_n' \subseteq A_n$. \\ Inoltre il sottogruppo $S_n'$ è caratteristico, e dunque normale, da cui segue $S_n' = \set{id}$ oppure $A_n$. La prima si esclude perché $S_n$ non è abeliano. 
\end{proof}
\begin{proposition}{derivato di $A_n$}
    se $n \geq 5$ $A_n' = A_n$.
    
    Domanda: chi è $A_4'$?
\end{proposition}
\begin{proof}
    Il sottogruppo derivato è caratteristico, e quindi normale. Ma $A_n$ è semplice per $n $, quindi restano solo le possibilità $A_n' = A_n$ e $A_n' = \{ id\}$. Ma quest'ultima implicherebbe che $A_n$ sia abeliano, assurdo. 
\end{proof}
\hypertarget{InnSn}{\begin{proposition}{(*) automorfismi interni}
    $\forall n$ vale $Z(S_n) = \set{id}$, da cui segue $\Inn(S_n) \cong S_n$. 
\end{proposition}}
\begin{proof}
    I casi $n = 2,3$ si trattano a mano. Sia $id \neq \sigma \in S_n$. Costruiamo $\tau \in S_n$ che non commuti con $\sigma$. $\sigma \neq id \Rightarrow \exists k \in \set{1, \dots, n}$ $\sigma(k) \neq k$. Se prendo $j \neq k,\sigma(k),\sigma(\sigma(k))$ e considero $\tau = (k, \sigma(k), j)$ si ha per costruzione $\tau\circ\sigma(k) = j \neq \sigma(\sigma(k)) = \sigma\circ\tau(k)$, e quindi $\tau\sigma \neq \sigma\tau $ come voluto.
\end{proof}
\begin{proposition}{(*) Automorfismi di $S_n$}
    $\Aut(S_n) = S_n$ $\forall n \in \N \setminus \{2, 6\}.$
\end{proposition}
\begin{proof}
    La dimostrazione si divide in due parti: 
    \begin{enumerate}
    \item se un automorfismo manda trasposizioni in trasposizioni, allora è interno;
    \item un automorfismo manda trasposizioni in trasposizioni se $n \neq 6$;
    \end{enumerate}
    (1.) Sia $\varphi \in \Aut(S_6)$ che mappa trasposizioni in trasposizioni e consideriamo le immagini tramite $\varphi$ $\{ (a_k,b_k) : k = 2 \dots n \}$ dell'insieme di generatori $\{ (1,k) : k = 2 \dots n \}$. Dati $k \neq k'$ si ha $[(1, k), (1, k')] \neq \id$, quindi anche $[(a_k,b_k),(a_{k'},b_{k'})] \neq \id$, da cui $\lvert \{a_k,b_k\} \cap \{a_{k'},b_{k'}\} \rvert = 1$. (Wlog) $a_k = a_{k'} = a$. Poniamo $k=2,k'=3$, $a$ come sopra e mostriamo che anche $\forall k > 3 \ a \in \{a_k, b_k\}$. $[(1,k),(1,2)],[(1,k),(1,3)] \neq \id$, quindi se per assurdo fosse $a \notin \{a_k,b_k\}$ si avrebbe $\{a_k,b_k\} = \{b_2,b_3\}$, contro l'iniettività di $\varphi$: $\varphi((2,3)) = \varphi((1,3)(1,2)(1,3)) = (a,b_3)(a,b_2)(a,b_3) = (b_2,b_3) = \varphi((1,k))$. (Wlog) $a = a_k$ per ogni $k = 2\dots n$. Sia allora $\sigma \in S_n$ tale che $\sigma: 1 \mapsto a, k \mapsto b_k$ e $\varphi_\sigma \in \Inn(S_n)$ il coniugio per $\sigma$. Allora $\varphi$ e $\varphi_\sigma$ coincidono su un insieme di generatori, cioè $\varphi = \varphi_\sigma \in \Inn(S_n)$.
    
    $(2.)$ Un automorfismo preserva gli ordini degli elementi e le classi di coniugio. Preso $f \in \Aut(S_n)$ si ha che $f((1,2))$ ha ordine 2 (quindi è il prodotto di $k$ 2-cicli) e che $f(Cl((1,2))) = Cl(f((1,2)) \Rightarrow \#Cl((1,2)) = \#Cl(f((1,2))$. Usando la formula per la cardinalità delle classi di coniugio vista a inizio paragrafo, si ha:
    \[
    \#Cl((1,2)) = \frac{n!}{2\cdot (n-2)!}
    \qquad
    \#Cl(f((1,2)) = \frac{n!}{2^k\cdot k!(n-2k)!}
    \]
    Uguagliando le due espressioni, $(n-2)! = 2^{k-1}\cdot k!(n-2k)!$, le cui uniche soluzioni per $(n,k)$ sono $(n,1)$, $(6,3)$. Quindi, a meno che $n = 6$, si ha che $k = 1$, cioè ogni trasposizione viene mappata in una trasposizione.
\end{proof}

\begin{proposition}{Automorfismo esotico di $S_6$}
    Esiste un automorfismo di $S_6$ \emph{non} interno.
\end{proposition}
\begin{proof}
    Consideriamo l'azione per coniugio di $S_5$ sull'insieme dei suoi sei 5-Sylow. Tale azione è transitiva e ha kernel banale (il kernel è normale e non contiene $A_5$), dunque è un'immersione di $S_5$ in $S_6$. Sia $H \le S_6$ l'immagine di $S_5$ tramite questa immersione. $H$ è un sottogruppo transitivo, quindi non una delle copie canoniche di $S_5$ in $S_6$. Consideriamo l'azione per moltiplicazione a sinistra di $S_6$ sull'insieme dei sei laterali $S_6/H$. Tale azione ha kernel banale, quindi corrisponde a un automorfismo di $S_6$. La moltiplicazione per elementi del gruppo (transitivo) $H$ fissa il laterale $eH$, quindi la sua immagine tramite l'automorfismo non è transitiva. Segue che l'automorfismo così trovato non è un automorfismo interno, infatti mappa un sottogruppo transitivo in un sottogruppo non transitivo.
\end{proof}

\subsection{Presentazione di gruppo}

\begin{definition}{gruppo libero}
    chiamiamo gruppo libero su $n$ generatori l'insieme
    \[ F_n = \grp{x_1, \dots, x_n} = \set{s = x_{i_1}^{\alpha_1} x_{i_2}^{\alpha_2} \dots x_{i_k}^{\alpha_k} : k \in \N, i_j \in \set{1,\dots,n} \text{ e } \alpha_j \in \{\pm 1\} \ \forall \  1 \leq j \leq k, \ s \text{ ridotta}} \]
    Ignorando l'ultima condizione esso è l'insieme delle stringhe (scritture formali) di lunghezza finita ottenute concatenando come caratteri degli $x_i$ oppure dei loro inversi (definizione solo formale). Diciamo che una tale stringa è \textit{ridotta} se non esiste $i$ tale che esistano un $x_i$ e un $x_i ^{-1}$ consecutivi. Chiaramente per ciascuna scritta esiste la stringa ridotta, ottenibile in un numero finito di passi. Talvolta si dice che due stringhe a cui è associata la stessa stringa ridotta sono \textit{equivalenti}.
    
    Ciò ci permette di dare a $F_n$ struttura di gruppo: l'operazione è quella di concatenare le stringhe una accanto all'altra, l'elemento neutro è la stringa vuota e l'inverso di $s$ è la stringa ottenuta riscrivendo $s$ da destra a sinistra al contrario e cambiando segno agli esponenti (è già ridotta e chiaramente concatenandola a $s$ e riducendo si cancella tutto).
    
    Notare che la definizione funziona ugualmente anche se il gruppo dei generatori è infinito (continuiamo a considerare le stringhe finite). In tal caso chiameremo $\grp{X}$ il gruppo libero generato da $X$. 
\end{definition}
\begin{example}
    $F_1 = \grp{x} = \set{x^k : k \in \Z} \cong \Z$.
\end{example}
\begin{proposition}{proprietà universale del gruppo libero}
    per ogni gruppo $G$ si ha \[ \forall 1 \leq i \leq n \ \Hom(F_n, G) \leftrightarrow G^n \cong \{(g_1, \dots, g_n) : g_i \in G\}, \] ovvero sono in bigezione.
\end{proposition}
\begin{proof}
    verifiche formali (lasciate per esercizio), la bigezione è quella naturale. 
\end{proof}
\begin{definition}{presentazione di gruppo}
    Dato $G$ gruppo, $\grp{S \mid R}$ è una presentazione di $G$ se:
    \begin{itemize}
        \item $S\subseteq G$ è un insieme di \emph{generatori} di $G$
        \item $R \subseteq \grp{S}$ (gruppo libero generato da $S$) è un insieme di \emph{relazioni}, tale che $\exists \varphi: \grp{S} \rightarrow G$ surgettivo con $\ker(\varphi) = N_R$ più piccolo sottogruppo normale di $\grp{S}$ che contiene $\grp{R}$.

        Intuitivamente, sono scritture formali tali che una volta valutate nel gruppo $G$ siano l'elemento neutro. Ciò sarà chiarito dall'esempio.
    \end{itemize} 
    Notare che, nella notazione della definizione e usando $\varphi$ vale $G \cong \grp{S}/N_R$.
\end{definition}
\begin{itemize}
    \item la presentazione è un invariante per isomorfismo
    \item non tutti i gruppi ammettono presentazione finita (sia come generatori che come relazioni)
    \item la presentazione non è unica (ci possono essere vari insiemi di generatori o varie relazioni ammissibili)
    \item a ogni presentazione è associato un unico gruppo (a meno di isomorfismo)
    \end{itemize}
\begin{example}
    $\Z/n\Z = \grp{x \mid x^n = 1}$ (di solito invece di indicare le relazioni si scrive ``stringa'' = el. neutro).
\end{example}
\begin{example}
    il gruppo libero generato da $S$ ha presentazione $\grp{S} = \grp{S \mid \emptyset}$.
\end{example}
\begin{example}
    Per dire che due elementi commutano basta scrivere $[a,b] = 1$ (si ricorda che $[a,b] = aba^{-1}b^{-1}$ è il commutatore). Quindi ad esempio $\Z/2\Z \times \Z/2\Z = \grp{x,y \mid x^2 = y^2 = xyx^-1y^-1 = 1}$ e l'informazione "il gruppo è abeliano" può essere espressa mediante le relazioni $\set{[g,h] = 1 \mid g,h \in G}$.
\end{example}
È un utile esercizio mostrare che le seguenti due sono presentazioni usando la definizione data sopra:
\begin{example}
    $D_n = \grp{ r, s \mid r^n = id, s^2 = id, srsr = id }$ (da queste relazioni si ricava $s^ar^bs^cr^d = s^{a+c}r^{(-1)^cb+d}$).
\end{example}
\begin{example}
    $Q_8 = \grp{i,j \mid i^4 = 1, i^2j^{-2} = 1, j^{-1}iji = 1} = \grp{i,j \mid i^4 = 1, i^2 = j^2, ji = i^{-1}j}$.
\end{example}
A cosa ci serve una presentazione? Intuitivamente, è l'insieme dei check minimali da fare quando si costruisce un omomorfismo definendolo prima sui generatori e poi volendolo estendere. Nella pratica, è molto difficile lavorarci e costruirle, nel corso si usano principalmente per ``riconoscere'' $D_n$ oppure $Q_8$ mentre si studia un gruppo. 


\subsection{Invertibili modulo $n$}

\begin{proposition}{invertibili modulo $n$}
    $(\Zn)^\times$ (come gruppo moltiplicativo) è ciclico se e solo se $n = 2,4,p^k$ o $2p^k$ con $p$ primo dispari, $k \geq 1$.
\end{proposition}
\begin{proof}
    La dimostrazione per $n = p$ è data ad aritmetica. Si verifica manualmente che $(\Zn)^\times$ è ciclico per $n=2, 4$. Mostriamo che $(\Zn)^\times$ è ciclico anche per $n = p^k$, il caso $n = 2p^k$ segue dal teorema cinese del resto:
    \[
        (\Zn)^\times \cong (\Z/2\Z)^\times \times (\Z/p^k\Z)^\times \cong (\Z/p^k\Z)^\times.
    \]
    
    Sia in seguito $r$ un generatore modulo $p$.
    
    \begin{lemma2}
        uno tra $r$ e $r+p$ è generatore $\pmod{p^2}$.
    \end{lemma2}
    \begin{proof}
        Poiché $\varphi(p^2) = p(p-1)$ e $r,r+p$ sono coprimi con $p^2$ si ha $\ord_{p^2}(r), \ord_{p^2}(r+p) \mid p(p-1)$. Ma poiché entrambi sono per definizione generatori $\pmod{p}$ deve necessariamente valere $p-1  \mid  \ord_{p^2}(r), \ord_{p^2}(r+p)$. Quindi per entrambi gli ordini abbiamo due possibilità: $p-1$ e $p(p-1)$. Supponiamo per assurdo che entrambi gli ordini siano $p-1$. Sviluppando il binomio di Newton si ha:  \[ 1 \equiv (r + p)^{p-1} = \sum_{i=0}{p-1} \binom{p-1}{i}p^ir^{p-1-i} \equiv r^{p-1} + (p-1)pr^{p-2} \equiv 1 - pr^{p-2} \pmod{p^2}, \]
        dove la penultima congruenza segue dal fatto che per $i\geq 2$ l'addendo nella sommatoria è divisibile per $p^2$ e quindi $\equiv 0 \pmod{p^2}$. Si ha quindi $0 \equiv pr^{p-2} \pmod{p^2}$ che è assurdo essendo $r,p$ coprimi. 
    \end{proof}
    Sia ora $s$ il generatore modulo $p^2$ trovato grazie al lemma. Ci occorre prima di tutto un lemma.
    
    \textbf{lemma:} se $a,m \geq 1$ si ha $(1+mp)^{p^a} \equiv 1 + mp^{a+1} \pmod{p^{a+2}}$.
    \begin{proof}
        Per induzione su $a$. 
        \begin{itemize}
            \item caso base, $a=1$. Sviluppiamo il binomio di Newton:
                \[
                    (1+mp)^{p} \equiv \sum_{i=0}^{p} \binom{p}{i}(mp)^i \equiv 1 + mp^2 + m^2p^3 \frac{p-1}{2} \equiv 1 + mp^2 \pmod{p^3}
                \]
                dove nella congruenza centrale abbiamo eliminato i termini con $i \geq 3$ in quanto chiaramente divisibili per $p^3$.
            \item Per il passo induttivo $a \mapsto a+1$ scriviamo $(1+mp)^{p^a} = 1 + mp^{a+1} + m' p^{a+2}$ (è l'ipotesi induttiva) e procediamo in modo analogo:
            \begin{align*}
                (1+mp)^{p^{a+1}} &= (1 + p^{a+1}(m+m'p))^p \\
                &\equiv \sum_{i=0}^{p} \binom{p}{i} p^{i(a+1)}(m+m'p)^i \\
                &\equiv 1 + p^{a+2}(m+m'p) \\
                &\equiv 1 + mp^{a+2} \pmod{p^{a+3}}.
            \end{align*}
        \end{itemize}
    \end{proof}
    \textbf{lemma:} $s$ è generatore modulo $p^k$ per ogni $k \geq 2$.
    \begin{proof}
        Usando che $s$ è generatore modulo $p^2$ e Lagrange si ha $p(p-1)  \mid  \text{ord}_{p^k}(s)  \mid  \varphi(p^k) = p^{k-1}(p-1)$, da cui segue che $\text{ord}_{p^k}(s) = p^h(p-1)$ per qualche $1 \leq h \leq k-1$. Supponiamo per assurdo che $h < k-1$ e quindi che $\text{ord}_{p^k}(s)  \mid  p^{k-2}(p-1)$. Poiché $s$ era per costruzione anche un generatore modulo $p$ si ha $s^{p-1} = 1 + mp $ per qualche $m \in \Z$. Usando il lemma appena mostrato si ha allora \[ 1 \equiv s^{p^{k-2}(p-1)} =\equiv (1 + mp)^{p^{k-2}} \equiv 1 + mp^{k-1} \pmod{p^k}, \]
        da cui segue che $ mp^{k-1} \equiv 0\pmod{p^k}$, che implica $p  \mid  m$. Ciò è però impossibile visto che si avrebbe $s^{p-1} \equiv 1 \pmod{p^2}$, ma $s$ è un generatore modulo $p^2$. 
    \end{proof}
    Questo conclude la dimostrazione della ciclicità per il caso $n = p^k$. Per $n = 2p^k$ si ha per il teorema cinese del resto: \[(\Zn)^\times \cong (\Z/2\Z)^\times \times (\Z/p^k\Z)^\times \cong (\Z/p^k\Z)^\times,\]
    dove l'ultima uguaglianza segue dal fatto che $(\Z/2\Z)^\times$ è il gruppo banale.

    Lasciamo le ultime verifiche per esercizio. Sapendo che $(\Z/8\Z)^\times \cong (\Z/2\Z)^2$, dimostrare che $(\Z/2^k\Z)^\times$ non è ciclico. Usando il TCR, mostrare che se la fattorizzazione di $m$ non ha la forma sopra, allora $(\Z/m\Z)^\times$ non è ciclico.
\end{proof}


\subsection{Esercizi di classificazione}
\begin{example2}{classificazione dei gruppi di ordine 12}
    Gli unici gruppi di ordine $12$ sono, a meno di isomorfismo: $\Z/12\Z, \Z/2\Z\times \Z/6\Z, D_6, A_4,\Z/3\Z \rtimes \Z/4\Z$.
\end{example2}
\begin{proof}
    Notiamo $12 = 2^2 \cdot 3$. Siano $P_2,P_3$ rispettivamente un $2$-Sylow e un $3$-Sylow di $12$. Notiamo che $P_2 \cap P_3 = \{e\}$ (le cardinalità sono coprime) e quindi $P_2 P_3 = G$ (per cardinalità). Dal teorema di Sylow si vede che $n_2 = 1, 3$ e $n_3 = 1, 4$.
    Guardando le possibilità e le cardinalità, necessariamente uno tra $P_2$ e $P_3$ è normale in $G$ { \tiny (due $3$-Sylow distinti devono necessariamente avere intersezione banale, quindi se $n_3 = 4$ si ha che i tre $3$-Sylow hanno unione di cardinalità $4\cdot (3-1) + 1 = 9$; un 2-Sylow ha chiaramente intersezione banale con questa unione, quindi deve essere contenuto nei $12 - 9 + 1 = 4$ elementi rimanenti, da cui segue che ce ne può essere uno solo) }, quindi $G \cong P_2 \rtimes P_3$ oppure $G \cong P_2 \rtimes P_3$. Poiché $P_3 \cong \Z / 3 \Z$ e $P_2 \cong \Z / 4 \Z$ oppure $P_2 \cong \Z / 2\Z \times \Z / 2\Z$ facciamo i casi: 
    \begin{itemize}
        \item $G \cong P_2 \rtimes P_3$, $P_2 \cong \Z / 4 \Z$. Quindi $G \cong  \Z / 4 \Z \rtimes_{\varphi} \Z / 3 \Z$. $\varphi : \  \Z / 3 \Z \rightarrow \text{Aut}(\Z / 4 \Z) \cong \Z / 2 \Z$ e quindi l'unica scelta per $\varphi$ è l'identità (i due gruppi hanno ordini coprimi). Quindi  $G \cong \Z / 4 \Z \times \Z / 3 \Z \cong \Z / 12 \Z$.
        \item $G \cong P_2 \rtimes P_3$, $P_2 \cong \Z/2\Z \times \Z/2\Z$. Quindi $G \cong  (\Z/2\Z)^2 \rtimes_{\varphi} \Z / 3 \Z$. $\varphi :  \Z / 3 \Z \rightarrow \text{Aut}((\Z/2\Z)^2) \cong S_3$ e quindi abbiamo due scelte.
        \begin{itemize}
            \item $\varphi = id$ e quindi  $G \cong (\Z / 2 \Z)^2 \times \Z / 3 \Z \cong \Z / 2 \Z \times \Z / 6 \Z$.
            \item $\varphi \neq id$ che ci dà $G \cong A_4$, con l'isomorfismo che manda $(\Z / 2 \Z)^2$ nel Klein e $\Z / 3 \Z$ nel gruppo generato da un tre ciclo.
        \end{itemize} 
        \item $G \cong P_3 \rtimes P_2$, $P_2 \cong \Z / 4 \Z$. Quindi $G \cong  \Z / 3 \Z \rtimes_{\varphi} \Z / 4 \Z$. $\varphi : \  \Z / 4 \Z \rightarrow \text{Aut}(\Z / 3 \Z) \cong \Z / 2 \Z$ e quindi abbiamo due scelte:
        \begin{itemize}
            \item $\varphi = id$ e quindi  $G \cong \Z / 4 \Z \times \Z / 3 \Z \cong \Z / 12 \Z$ (già visto)
            \item $\varphi \neq id$ che non dà "gruppi noti"; scriviamo solo $G \cong \Z/3\rtimes \Z/4\Z$
        \end{itemize} 
        \item $G \cong P_3 \rtimes P_2$, $P_2 \cong \Z / 2 \Z \times \Z/2\Z$. Quindi $G \cong \Z / 3 \Z  \rtimes_{\varphi} (\Z/2\Z)^2$. $\varphi : \  (\Z/2\Z)^2 \Z \rightarrow \text{Aut}(\Z / 3 \Z) \cong \Z / 2 \Z$ e quindi abbiamo due scelte (in realtà le scelte sono 4, se consideriamo i due generatori canonici, ma tre di esse sono chiaramente isomorfe):
        \begin{itemize}
            \item $\varphi = id$ e quindi $G \cong (\Z / 2 \Z)^2 \times \Z / 3 \Z \cong  \Z / 2 \Z \times \Z / 6 \Z$ già visto. 
            \item $\varphi \neq id$, e quindi $\ker(\varphi) \cong \Z/2\Z$. Sia $\ker(\varphi) = \grp{x}$ con $x \in G$, e siano $y,z \in G$ tali che $\grp{z} \cong \Z/3\Z$ e $\grp{x}\grp{y} \cong (\Z / 2 \Z)^2 \Rightarrow G \cong \grp{z} \rtimes \grp{x}\grp{y}$. Allora per definizione (essendo $\varphi_z$ il coniugio per $z$) si ha $zxz^{-1} = x, zyz^{-1} = xy = yx \Rightarrow $
        Si può allora mostrare che $Z(G) \cong \Z/6\Z$. Poiché
        \end{itemize} 
    \end{itemize}
\end{proof}
\begin{example2}{classificazione dei gruppi di ordine 8}
    gli unici gruppi di ordine $8$ sono, a meno di isomorfismo: $\Z/8\Z, \Z/2\Z\times \Z/4\Z, \big(\Z/2\Z\big)^3, D_4,Q_8$.
\end{example2}
\begin{proof}
    Facciamo qualche considerazione sugli ordini degli elementi. Detto $G$ un gruppo di ordine 8 e preso $g \in G$ per Lagrange $\ord(g) \in \set{1,2,4,8}$. 
    
    Se tutti gli elementi hanno ordine 2 $G$ è abeliano; basta infatti notare che in tal caso si ha $g^2 = e$ e quindi $g^{-1} = g$ $\forall g \in G$. Da ciò segue $gh = g^{-1}h^{-1} = (hg)^{-1} = hg$ $\forall g,h\in G$. Usando il teorema di struttura, questo caso ci dà $G \cong (\Z/2\Z)^3$
    
    Se un elemento ha ordine 8 allora $G$ è ciclico e $G \cong \Z/8\Z$

    Se nessuna delle precedenti condizioni è verificata $\exists a \in G$ tale che $\ord(a) = 4$. $[G : \grp{a}] = 2 \Rightarrow \grp{a} \tri G \Rightarrow$ preso $b \in G \setminus \grp{a}$ si ha $G = \grp{a} \cup b\grp{a}$ ($G$ è unione delle classi laterali), ovvero $G = \set{e,a,a^2,a^3,b,ba,ba^2,ba^3}$. Notiamo che, essendo $\grp{a}$ normale, necessariamente $bab^{-1} \in \grp{a}$. Facciamo i casi:
    \begin{itemize}
        \item $bab^{-1} = e \Rightarrow a =e$ assurdo
        \item $bab^{-1} = a^2 \Rightarrow (bab^{-1})^2 = a^4 = e \Rightarrow ba^2b^{-1} = e \Rightarrow a^2 = e$ assurdo
        \item $bab^{-1} = a \Rightarrow ab = ba$ da cui si verifica facilmente che $G$ è abeliano. I casi con $G$ abeliano sono dati dal teorema di struttura, e sono quelli nel testo. 
        \item $bab^{-1} = a^3 = a^{-1}$. In questo caso distinguiamo due casi in base a $\ord(b)$, notando che le possibilità (per come abbiamo fatto i casi) sono solo 2 e 4:
        \begin{itemize}
            \item se $\ord(b) = 2$ si può verificare che $G \cong D_4$, mediante l'isomorfismo che manda $a \mapsto r$, $b \mapsto s$; la presentazione di gruppo in questo caso è $\grp{a,b \mid a^4 = 1, b^2= 1, ba = a^3b}$ (notare che è la stessa di $D_4$)
            \item se $\ord(b) = 4$ si può verificare che $G \cong Q_8$, mediante l'isomorfismo che manda $a \mapsto i$, $b \mapsto j$; la presentazione di gruppo in questo caso è $\grp{a,b \mid a^4 = 1, b^4= 1, ba = a^3b}$. Si noti che è la stessa di $Q_8$.
        \end{itemize}
    \end{itemize}
\end{proof}
\begin{example2}{classificazione dei gruppi di ordine 30}
    Gli unici gruppi di ordine $30$ sono, a meno di isomorfismo: $\Z/30\Z, D_{15}, D_5 \times \Z/3\Z,  D_3 \times \Z/5\Z$.
\end{example2}
\begin{proof}
    Notiamo innanzitutto che $30 = 2\cdot15$ e quindi è della forma $2d$ con $d$ dispari. Pertanto, detto $G$ un generico gruppo di ordine 30, esiste $H \tri G$ con $\#H = 15$. $15 = 3 \cdot 5$, quindi è della forma $pq$ con $p,q$ primi. Per quanto visto a teoria, poiché $3 \nmid 5-1 = 4$ si ha $H \cong \Z/3\Z \times \Z/5\Z \cong\Z/15\Z$ e quindi $H = \grp{x}$ (è ciclico). Sia ora $y \in G$ di ordine $2$, che esiste per Cauchy. Chiaramente $y \not \in H$ per una questione di ordini. Quindi $\grp{x}\cap\grp{y} = \set{e}$ e per cardinalità $\grp{x}\grp{y} = G$, da cui segue dal teorema di decomposizione in prodotto semidiretto (ricordando che $\grp{x}$ è normale) che $G \cong \grp{x} \rtimes_{\varphi} \grp{y} \cong  \Z/15\Z \rtimes_{\psi} \Z /2\Z$. Sempre dal teorema segue che  $\varphi: \grp{y} \rightarrow \text{Aut}(\grp{x})$ è il coniugio, ossia $\varphi_y(x) = yxy^{-1}$.
    
    Notiamo che $\text{Aut}(\grp{x}) \cong \text{Aut}(\Z/15\Z) \cong (\Z/15\Z)^\times \cong (\Z/3\Z)^\times \times (\Z/5\Z)^\times \cong \Z/2\Z \times \Z/4\Z$. Poiché $y$ ha ordine 2 necessariamente $\varphi_y$ ha ordine 1 (quindi è $id$) oppure 2. In entrambi i casi $\varphi_y(\varphi_y(x)) = x$.
    
    Notiamo intanto che per normalità di $\grp{x}$ si ha $\varphi_y(x) \in \grp{x}$ e quindi $yxy^{-1} = \varphi_y(x) = x^{\ell}$.
    $(\ell,15) = 1$, perché altrimenti $\varphi_y$ non sarebbe automorfismo, e che $x = \varphi_y(\varphi_y(x)) = x^{2\ell} \Rightarrow 2 \ell \equiv 1 \pmod{15}$. Ciò implica
    $\begin{cases}
    \ell \equiv \pm 1 \pmod{3} \\
    \ell \equiv \pm 1 \pmod{5} 
    \end{cases}$
    e ci dà perciò quattro casi: 
    \begin{enumerate}
        \item
        $\begin{cases}
        \ell \equiv  1 \pmod{3} \\
        \ell \equiv  1 \pmod{5} 
        \end{cases}$ $\Rightarrow \ell \cong 1 \pmod{15} \Rightarrow yxy^{-1} = x \Rightarrow yx = xy$.
        
        Quindi il gruppo è abeliano, ovvero $G \cong \grp{x} \times \grp{y} \cong \Z/2\Z \times \Z/15\Z \cong \Z/30\Z$.
        
        \item
        $\begin{cases}
        \ell \equiv  1 \pmod{3} \\
        \ell \equiv  -1 \pmod{5} 
        \end{cases}$  $\Rightarrow \ell \cong 4 \pmod{15} \Rightarrow yxy^{-1} = x^4$ \\
        Si può osservare che ciò implica $yx^5y^{-1} = x^5  \Rightarrow x^5 \in Z_G$. 
        A questo punto si verifica manualmente $G \cong D_5 \times \Z/3\Z$ con l'isomorfismo che manda $x \mapsto (r,\overline{1})$, $y \mapsto (s,\overline{0})$.
        
        \item
        $\begin{cases}
        \ell \equiv  -1 \pmod{3} \\
        \ell \equiv  1 \pmod{5} 
        \end{cases}$  $\Rightarrow \ell \cong -4 \pmod{15} \Rightarrow yxy^{-1} = x^{-4}$
        
        In modo analogo a prima $x^3 \in Z_G$ e si verifica manualmente $G \cong D_3 \times \Z/5\Z$.
        
        \item
        $\begin{cases}
        \ell \equiv  -1 \pmod{3} \\
        \ell \equiv  -1 \pmod{5} 
        \end{cases}$ $\Rightarrow \ell \cong -1 \pmod{15} \Rightarrow yxy^{-1} = x^{-1}$ \\ Riconosciamo adesso la presentazione di $D_{15}$. Si ha allora $G \cong D_{15}$. 
    \end{enumerate}
\end{proof}
\begin{example2}{(*) gruppi semplici piccoli}
    Quali sono i possibili ordini $\leq 100$ di un gruppo semplice?

    Ricordiamo che un gruppo $G$ si dice semplice se i suoi unici sottogruppi normali sono $\set{e}$ e $G$. 
\end{example2}
\begin{proof}
    Si raccomanda di provare prima l'esercizio per conto proprio: per quanto noioso è utile per ripassare tutte le tecniche viste sulla classificazione gruppi. Si lascia qui uno sketch di svolgimento. Sia $n$ l'ordine del gruppo $G$ da studiare.

    Casi noti dalla teoria:
    \begin{itemize}
        \item $n = p$ primo $\Rightarrow$ $G$ semplice (perché?) 
        \item $n = p^k$ con $p$ primo $\Rightarrow$ $G$ NON semplice (perché?)
        \item $n = pq$ con $p,q$ primi $\Rightarrow$ $G$ NON semplice (perché?)
        \item $n = 2d$ con $d$ dispari $\Rightarrow$ $G$ NON semplice (perché?)
    \end{itemize}
    Vediamo ora dei casi in base a come si scompone in fattori primi $n \leq 100$. I casi a mano generalmente si trattano con Sylow. Se si mostra $n_p = 1$ per qualche primo $p  \mid  n$ allora il Sylow è normale, quindi il gruppo non è semplice. Potrebbero funzionare approcci che usino cardinalità cominciando col supporre $n_p > 1 \ \forall p  \mid  n$ per ottenere assurdi.
    
    Notiamo che ogni $n \le 100$ è prodotto di potenze di al più tre primi distinti, infatti $2 \cdot 3 \cdot 5 \cdot 7 = 210 > 100$.

    Caso in cui $n$ è nella forma $n = p^aq^br^c$ con $p < q < r$ primi e $a, b, c \geq 1$. Allora:
    \begin{itemize}
        \item se $c \geq 2$ allora $n \geq 2 \cdot 3 \cdot 5^2 > 100$. Quindi $c =1$.
        \item se $b \geq 3$ $n \geq 2 \cdot 3^3 \cdot 5 > 100$. Quindi $b = 1,2$
        \item se $a=b=1$, $n = pqr \Rightarrow G$ NON semplice (esercizio)
        \item se $b = 2$, allora necessariamente per $n \leq 100$ $p=2, q = 3, r = 5$ e $n = 90$ già trattato 
        \item resta $n = p^a qr$ con $a >1$, che dà i casi $n = 60,84$. Per $n = 60$ esiste $A_5$ che è semplice. Per $n = 84$ si mostra $G$ NON semplice.
    \end{itemize}
    
    Caso in cui $n$ è nella forma $n = p^aq^b$ con $p,q$ primi.
    \begin{itemize}
        \item i casi $a = b = 1$ sono già stati trattati.
        \item $n = 2^a \cdot 3^b$ che possono essere trattati a mano. Un modo veloce di gestirli è notare che $n_3 = 4$ oppure $16$. 
        \begin{itemize}
        \item se $n_3 = 4$ l'azione di coniugio sull'insieme dei $3$-Sylow dà un'immersione di $G$ in $S_4$ (il nucleo dell'azione è normale in $G$ quindi deve essere banale se $G$ semplice), che ci dà pochi casi tutti NON semplici.
        \item se $n_3 = 16$ abbiamo per cardinalità solo $n = 48,96$ e si potrebbero trattare a mano. Si può però notare che $n_2  \mid  3$ in entrambi i casi e quindi $n_2 = 3$ e in modo analogo a prima se $G$ fosse semplice dovremmo poterlo immergere in $S_3$, assurdo. Quindi NON sono semplice.
    \end{itemize} 
        \item $n = 2^kp$ con $p$ primo. Il caso $k=1$ è stato già trattato, $k = 2$ si può trattare a parte, gli altri ($n = 40,56,80,88$) vanno fatti a mano e NON sono mai semplici.
        \item rimangono infine i casi $n = 45,63,75,99,100$, che si trattano a mano e NON sono mai semplici
    \end{itemize}
    
    Gli unici gruppi semplici di ordine fino a $100$ sono quindi $A_5$ e gli $\Zp$.
\end{proof}


\subsection{Consigli e reminder per risolvere gli esercizi}
\begin{itemize}
    \item spesso vanno usati i punti precedenti;
    \item vale la pena conoscere (bene) la struttura di $D_n$ e $S_n$;
    \item i sottogruppi di $S_n$ in generale fanno un po' schifo. Alla richiesta di trovare sottogruppi di ordine dato in $S_n$, quasi sempre si risponde con centralizzatori o normalizzatori, eventualmente intersecati con $A_n$;
    \item i gruppi normali sono i $\ker$ di omomorfismi, quindi di omomorfismi \textit{da} gruppi semplici ce ne sono ben pochi;
    \item talvolta aiuta contare gli elementi di un certo ordine;
    \item commutare/essere normale equivale a commutare con/essere normale a un insieme di generatori;
    \item i prodotti $A \rtimes B$ sono generati da $A \times \{e\}$ e $\{e\} \times B$;
    \item $Z_G(H) \tri N_G(H)$, inoltre $N_G(H) / Z_G(H) \hookrightarrow \Aut(H)$;
    \item nel dubbio fai agire $G$ su $H < G$ per coniugio;
    \item il numero $n_p$ di $p$-Sylow è l'indice in $G$ del normalizzatore di un $p$-Sylow. Segue che il numero di $p$-Sylow di un prodotto diretto è il prodotto dei numeri di $p$-Sylow;
    \item se ho tanti $p$-Sylow, i normalizzatori sono piccoli;
    \item il centro sta quantomeno nell'intersezione di tutti i normalizzatori, quindi se i $p$-Sylow sono tanti e i normalizzatori sono piccoli, anche il centro non è troppo grande;
    \item qualche $p$-Sylow di solito è caratteristico;
    \item il generato da una classe di coniugio (o da un insieme di classi di coniugio) è normale. Quindi se $G$ è semplice ogni classe di coniugio genera;
    \item i $p$-Sylow di un sottogruppo sono tutti coniugati tramite elementi \textit{del sottogruppo}. Se $P$ è un $p$-Sylow di $G$ ed è contenuto in un sottogruppo $H$, allora $P$ è un $p$-Sylow di $H$;
    \item l'insieme dei $p$-Sylow genera;
    \item un'azione transitiva è isomorfa all'azione del gruppo sulle classi laterali di uno stabilizzatore;
    \item per il teorema di Poincaré $A_n$, $S_n$ non hanno sottogruppi di indice piccolo (i.e. $3, \dots, n-1$);
    \item per immergere gruppi in $S_n$ considera azioni il cui kernel è banale, per esempio $\GL_n(\Fp) \hookrightarrow S_{p^n - 1}$ tramite l'azione su $\Fp^n$;
    \item un Sylow è normale sse è caratteristico. Se il normalizzatore di un Sylow avvesse indice $>1$ minimo, allora sarebbe normale in $G$, quindi il Sylow sarebbe normale in $G$: assurdo;
    \item appena trovi un gruppo normale ha senso quozientare e usare corrispondenza.
\end{itemize}
\newpage
\section{Teoria degli anelli}

Dove non diversamente specificato, $A$ è un anello commutativo con unità con operazioni $+$ e $\cdot$ (il cui simbolo verrà omesso). L'elemento neutro della somma sarà indicato con $0$, quello del prodotto con $1$.

\subsection{Definizioni e richiami di Aritmetica}

\begin{definition}{ideale}
    un ideale $I$ di $A$ è un sottogruppo additivo di $A$ dotato della proprietà di assorbimento, ossia tale che $\forall a \in A \, aI \subseteq I$.
\end{definition}
\begin{definition}{ideale generato}
    dato un sottoinsieme $S \subseteq A$ l'ideale generato da $S$ è $(S) = \bigcap_{S \subseteq I \tri A} I$.
\end{definition}
\begin{definition}{ideale principale}
    un ideale $I$ di $A$ è detto principale se $\exists x \in A$ tale che $I = (x)$.
\end{definition}
\begin{definition}{ideale primo}
    $P \tri A$ è detto primo se $\forall x,y \in A$ $xy \in P \Rightarrow x \in P$ o $y\in P$.
\end{definition}
\begin{proposition}{ideali principali primi}
    $P = (x)$ è primo $\iff x$ è un elemento primo di $A$.
\end{proposition}
\begin{definition}{ideale massimale}
    $M \tri A$ proprio è detto massimale se $\forall I \tri A$ $M \subseteq I\subseteq A \Rightarrow I = M$ o $I = A$.
\end{definition}

\begin{definition}{elemento irriducibile}
    Sia $A$ un dominio. $x \in A$ si dice irriducibile se $\forall a, b \in A \ x = ab \Rightarrow a \in A^\times \lor b \in A^\times$.
\end{definition}
\begin{definition}{elemento primo}
    Sia $A$ un dominio. $p \in A \setminus (A^\times \cup \{ 0 \})$ si dice primo se $\forall a, b \in A \ p \mid ab \Rightarrow p \mid a \lor p \mid b$.
\end{definition}
\begin{definition}{elementi associati}
    Sia $A$ un dominio. $x,y \in A$ si dicono associati ``$x \sim y$'' se vale una delle seguenti condizioni equivalenti:
    \begin{enumerate}[label=(\roman*)]
        \item $\exists u \in A^\times \ x = uv$;
        \item $x \mid y \land y \mid x$;
        \item $(x) = (y)$.
    \end{enumerate}
\end{definition}
\begin{proposition}{primi e irriducibili}
    Sia $A$ un dominio. Valgono le seguenti implicazioni:
    \begin{enumerate}[label=(\roman*)]
        \item $x$ primo $\Rightarrow$ $x$ irriducibile;
        \item $x$ primo $\Leftrightarrow$ $(x)$ primo;
        \item $x$ irriducibile $\Leftrightarrow$ $(x)$ massimale nella classe degli ideali principali di $A$.
    \end{enumerate}
\end{proposition}
\begin{proof}
    dimostriamo solamente (iii). $(\Rightarrow)$, dato $x$ irriducibile $\forall y \in A \ (x) \subseteq (y) \subsetneq A \Rightarrow \exists a \in A x = ya$, ma poiché $x$ irriducibile e $(y) \neq A$ necessariamente $a \in A^\times$, cioè $x \sim y$ e quindi $(x) = (y)$, cioè $(x)$ massimale tra gli ideali principali. $(\Leftarrow)$, sia $x = ab$ con $a \notin A^\times$, per massimalità $(x) = (a) \subsetneq A$, da cui $\exists c \ a = xc$. Segue $x(1 - bc) = 0$ e quindi $b \in A^\times$, cioè $x$ irriducibile.
\end{proof}

\subsection{Ideali e proprietà}

Gli ideali di un anello possono in un certo senso essere pensati come l'analogo dei sottogruppi normali in un gruppo. Essi sono infatti i nuclei degli omomorfismi, e valgono per essi teoremi analoghi a quelli visti in teoria dei gruppi, come vedremo adesso. Per questo motivo scegliamo la stessa notazione per indicarli: $I \tri A$.

\vspace{0.5cm}

\begin{minipage}{0.7\textwidth}
\begin{theorem}{1$^{\circ}$ di omomorfismo (per anelli)}
    Siano $\varphi : A \rightarrow A'$ un omomorfismo di anelli e $I \tri A$ tale che $I \subseteq \ker(\varphi)$. Allora $\exists ! f$ che fa commutare il diagramma a lato.
    
    Inoltre $\imm(f) = \imm(\varphi)$ e $f$ iniettiva $\iff I = \ker(\varphi)$.
\end{theorem}
\end{minipage}
\hfill
\begin{minipage}{0.2\textwidth}  
\begin{tikzcd}
    A \arrow{r}{\varphi} \arrow{d}{\pi_I} & A'\\
    A/I \arrow[dashed]{ur}[swap]{f}
\end{tikzcd}
\end{minipage}
\begin{proof}
    Applichiamo il $1^{\circ}$ teorema di omomorfismo per gruppi (considerando $A, A'$ come gruppi additivi) e notiamo che la funzione ottenuta è un omomorfismo di anelli, per le proprietà del quoziente.
\end{proof}
\begin{theorem}{di corrispondenza (per anelli)}
    Sia $I \tri A$ e $\pi_I: A \rightarrow A/I$ la proiezione. Allora $\pi_I$ induce una corrispondenza tra gli ideali di $A/I$ e gli ideali di $A$ che contengono $I$. Tale corrispondenza preserva: ordinamento per inclusione, indice, ideali primi, ideali massimali. 
\end{theorem}\begin{proof}
    Per il teorema di corrispondenza per gruppi (un anello è in particolare un gruppo additivo abeliano) si ha che, detti $X = \{ H \leq A : I \subseteq H\}$ e $Y = \{ \overline{H} \leq A/I\}$, la funzione $\alpha: X \rightarrow Y$ che mappa $H \overset{\alpha}{\mapsto} \pi_I(H)$ è una bigezione tra i sottogruppi additivi di $A$ e i sottogruppi additivi di $A/I$. Restringiamo ora $\alpha$ a $X' = \{ H \tri A : I \subseteq H\}$ e definiamo analogamente $Y' = \{ \overline{H} \tri A/I \}$. Sia $\tilde \alpha$ la funzione ristretta. Poiché $\pi_I$ è omomorfismo di anelli suriettivo, immagine di ideali è ideale (esercizio) e quindi $\tilde \alpha$ manda ideali in ideali. Sempre per omomorfismo, controimmagine di ideali è ideali, quindi $\tilde \alpha : X' \rightarrow Y'$ è anche suriettiva. Ma allora per iniettività $\alpha$, $\tilde \alpha$ è una bigezione. $\tilde \alpha$ Preserva indice e contenimenti perché $\alpha$ lo faceva. Questa corrispondenza preserva inoltre primalità e massimalità perché $\pi_I$ è un omomorfismo suriettivo il cui kernel ($I$) è contenuto in tutti gli elementi di $X'$.
\end{proof}
\begin{theorem}{cinese del resto per anelli}
    Siano $I,J \tri A$ e $f: A \rightarrow A/I \times A/J$ tale che $f(a \mapsto (a+I,a+J))$. Allora $f$ è omomorfismo di anelli, $\ker(f) = I\cap J$ e $(I,J) = A \iff f$ suriettiva. 
\end{theorem}
\begin{proof}
    Che sia omomorfismo di anelli segue dal fatto che sia la proiezione su $I$ che quella su $J$ lo sono.
    $\ker(f) = \{a \in A : a+I = I \text{ e } a+J = J\} = \{a \in A : a \in I \text{ e } a \in J\} = I \cap J$. Per la suriettività, $(I,J)= A \Rightarrow 1 \in (I,J) \Rightarrow \exists x \in I, y \in J \ x+y = 1$. Allora $\forall a,b \in A$ dalla proprietà di assorbimento segue che $f(ax+by) = (ax+by+I, ax+by+J) = (by + I, ax + J) = (b(1-x) + I, a(1-y) +J) = (b+I, a+J)$. Nel verso opposto, se $f$ è suriettiva allora esiste $x \in A$ tale che $(x+I, x+J) = (I,1+J)$, quindi $x \in I$ e $(1 - x) \in J$. Ma allora $1 = x + (1 - x) \in I + J$, vale a dire $(I, J) = A$.
\end{proof}
\begin{theorem}{lemma di Zorn}
    Sia $(X, \leq)$ un insieme non vuoto parzialmente ordinato (poset). Esso si dice induttivo se ogni catena (sottoinsieme di $X$ totalmente ordinato) ammette maggiorante. Se $X$ è induttivo, allora esiste un elemento massimale.
\end{theorem}
\begin{proof}
    È equivalente sotto ZF all'assioma di scelta. Si rimanda al corso ``Elementi di Teoria degli Insiemi''.
\end{proof}

\begin{proposition}{ideali massimali}
    Sia $\mathscr{F} = \{ I \tri A\}$ l'insieme degli ideali propri di $A$. Allora $(\mathscr{F}, \subseteq)$ è un insieme induttivo, ossia verifica le ipotesi del lemma di Zorn. Valgono le seguenti:
    \begin{itemize}
        \item ogni anello possiede ideali massimali (basta applicare Zorn a $\mathscr{F}$);
        \item ogni elemento non invertibile di $A$ è contenuto in un ideale massimale (basta applicare Zorn a $\mathscr{F} \cap \{\text{ideali contenenti l'elemento}\}$);
        \item ogni ideale proprio è contenuto in un ideale massimale (basta applicare Zorn a $\mathscr{F} \cap \{\text{ideali contenenti l'ideale fissato}\}$).
    \end{itemize}
\end{proposition}
\begin{theorem}{caratterizzazione ideali primi e massimali}
    $I$ ideale. Allora valgono le seguenti:
    \begin{itemize}
        \item $I$ è primo $\iff A/I$ dominio;
        \item $I$ è massimale $\iff A/I$ campo.
    \end{itemize}
    Come corollario, $I$ massimale implica $I$ primo.
\end{theorem}
\begin{proof}
    $A/I$ dominio $\iff \forall x,y \in A \ ((x+I)(y+I) = I \iff x+I = I \lor y+I = I) \iff \forall x,y \in A \ (xy+I = I \iff x+I = I \lor y+I = I) \iff \forall x,y \in A \ (xy \in I \iff x\in I \lor y\in I) \iff I$ è primo.
    $A/I$ campo $\iff$ ogni suo elemento è invertibile $\iff$ i suoi unici ideali sono $0$ e $A/I$ $\iff$ (per corrispondenza) gli unici ideali contenenti $I$ sono $I$ e $A$ $\iff$ $I$ è massimale
\end{proof}
\begin{theorem}{ideali e omomorfismi}
    Sia $f: A \rightarrow B$ omomorfismo di anelli. Allora valgono le seguenti:
    \begin{itemize}
        \item controimmagine di ideale è ideale;
        \item controimmagine di ideale primo è ideale primo.
    \end{itemize}
    Se $f$ è suriettivo valgono inoltre: 
    \begin{itemize}
        \item immagine di ideale è ideale;
        \item controimmagine di ideale massimale è ideale massimale;
        \item immagine di ideale massimale è ideale massimale.
    \end{itemize}
\end{theorem}
\begin{proof}
    esercizio.
\end{proof}

\subsection{Operazioni, ideali coprimi e radicale}
In seguito $I$ e $J$ sono due ideali qualsiasi dell'anello $A$.
\begin{definition}{somma di ideali}
    $I + J = \{ i + j : i \in I, j \in J \}$ è un ideale.
\end{definition}
\begin{definition}{intersezione di ideali}
    $I \cap J$ è un ideale.
\end{definition}
\begin{definition}{prodotto di ideali}
    indichiamo con $IJ$ l'ideale $(IJ)$ generato da $\{ij : i \in I, j \in J\}$.
\end{definition}
\begin{definition}{ideali coprimi}
    $I, J$ si dicono coprimi se $I + J = A$.
    
    Intuitivamente, due ideali sono coprimi se l'unico ideale che ``divide'' (i.e. contiene) entrambi è $(1) = A$. Alternativamente $I$ e $J$ sono ideali coprimi $\Leftrightarrow \exists x \in I, y \in J \ x + y = 1$.
    
    \dots ma attenzione! Dati $x, y \in A$ UFD con $MCD(x, y) = 1$, non necessariamente $(x)$ e $(y)$ sono coprimi, vedi $(2), (x) \in \Z[x]$.
\end{definition}
\begin{proposition}{prodotto e intersezione}
    $IJ \subseteq I \cap J$. Se inoltre $I$ e $J$ sono coprimi, vale l'uguaglianza. % è un sse? No: prendi un qualunque I = J ideale proprio.
\end{proposition}
\begin{proof}
    Il contenimento segue dalla proprietà di assorbimento comune sia a $I$ che a $J$. Se inoltre $\exists x \in I, y \in J \ x + y = 1$ si ha $\forall z \in I \cap J \ z = xz + yz \in IJ$, dunque il contenimento inverso e l'uguaglianza.
\end{proof}
\begin{definition}{radicale}
    indichiamo con $\sqrt{I}$ l'ideale $\{x \in A : \exists n \in \N \ x^n \in I\}$.
\end{definition}
Si noti la monotonia del radicale $I_1 \subseteq I_2 \implies \sqrt{I_1} \subseteq \sqrt{I_2}$ e che $\sqrt{\sqrt{I}} = \sqrt{I}$.
\begin{definition}{anello ridotto}
    $A$ si dice ridotto se $\sqrt{(0)} = (0)$.
\end{definition}
\begin{proposition}{radicale del prodotto}
    $\sqrt{IJ} = \sqrt{I \cap J} = \sqrt{I} \cap \sqrt{J}$
\end{proposition}
\begin{proof}
    Per monotonia $\sqrt{IJ} \subseteq \sqrt{I \cap J}$. D'altra parte, $x^n \in I \cap J \implies x^{2^n} = x^n x^n \in IJ$, dunque $\sqrt{IJ} = \sqrt{I \cap J}$. Ancora per monotonia si ha $\sqrt{IJ} \subseteq \sqrt{I} \cap \sqrt{J}$, inoltre dato $x$ tale che $x^n \in I$ e $x^m \in J$ si ha $x^{n+m} \in IJ$, quindi l'ultima uguaglianza.
\end{proof}
\begin{proposition}{radicale di un primo}
    se $P$ è ideale primo $\sqrt{P} = P$
\end{proposition}
\begin{proof}
    Per monotonia $P \subseteq \sqrt{P}$. Sia ora $x \in \sqrt{P}$ e $n = \min\{k \in \N : x^k \in P \}$. Se fosse $n > 1$, per primalità di $P$ si avrebbe $x \cdot x^{n-1} = x^n \in P \Rightarrow x \in P \lor x^{n-1} \in P$, contro la minimalità di $n$. Quindi necessariamente $n = 1$, cioè $x \in P$.
\end{proof}
\begin{proposition}{radicale di un ideale}
    Per ogni $I \tri A$ si ha
    \[
    \sqrt{I} = \bigcap_{\substack{I \subseteq P \tri A \\ P \text{ primo}}} P.
    \]
    In particolare $\sqrt{(0)} = \bigcap_{P \tri A \text{ primo}} P$.
\end{proposition}
\begin{proof}
    Notiamo innanzitutto che per Zorn esiste almeno un ideale massimale (dunque primo) che contiene $I$, quindi l'intersezione è ben definita.
    Il contenimento ``$\subseteq$'' segue allora dalla monotonia del radicale.
    Per il contenimento inverso, sia $a \in A \setminus \sqrt{I}$: vogliamo mostrare che esiste un ideale primo che contiene $I$ e a cui $a$ non appartiene. Detto $S = A \setminus \grp{a}$, sia $\mathscr{F} = \{J \tri A : I \subseteq J \subseteq S\}$. $\mathscr{F}$ è non vuoto ($I \in \mathscr{F}$) e induttivo. Per il lemma di Zorn esiste un elemento massimale $Q$: per concludere mostriamo che $Q$ è primo. Sia $x \notin Q$, allora:
    \begin{itemize}
        \item se $x \in S \setminus Q$, per massimalità di $Q$ in $S$ si ha $(Q,x) \nsubseteq S$, dunque esistono $k \in \N, q \in Q, b \in A$ tali che $q + bx = a^k$;
        \item se $x \in \grp{a}$, allora $\exists k \in \N \ x = a^k$, quindi vale ancora l'enunciato sopra scegliendo $q = 0, b = 1$.
    \end{itemize}
     Dati allora $x,y \in A \setminus Q$ valgono $q_1 + b_1 x = a^{k_1}$ e $q_2 + b_2 y = a^{k_2}$ per opportuni $q_i, b_i, k_i$. Dunque $a^{k_1 + k_2} = (q_1 + b_1 x)(q_1 + b_2 y) = (q_1 q_2 + q_1 b_2 y + q_1 b_1 x) + b_1 b_2 x y$, dove $q_1 q_2 + q_1 b_2 y + q_1 b_1 x \in Q$ per le proprietà di assorbimento e sottogruppo. Ma $a^{k_1 + k_2} \notin Q \Rightarrow b_1 b_2 x y \notin Q \Rightarrow xy \notin Q$. Cioè $Q$ primo.
\end{proof}

\subsection{Parti moltiplicative e campo dei quozienti}
\begin{definition}{parte moltiplicativa}
    un insieme $S \subseteq A$ è detto parte moltiplicativa se è un semigruppo moltiplicativo (ossia $\forall x,y\in S \ xy \in S$) che non contiene $0$ e contiene $1$.
    
    Da ora in poi $S$ sarà una parte moltiplicativa di $A$.
\end{definition}
\begin{definition}{localizzazione}
    si chiama localizzazione di $A$ rispetto a $S$ l'insieme $S^{-1}A := A \times S / \sim$ dove la relazione di equivalenza è definita da $(a, s) \sim (b, t) \iff \exists u \in S \ u(at - bs) = 0$. Indichiamo $(a, s) \in S^{-1}A$ con $\frac{a}{s}$. $S^{-1}A$ è un anello le cui operazioni sono definite come le operazioni sulle frazioni.
\end{definition}
\begin{proposition}{$A$ si immerge nella localizzazione}
    l'immersione $f: A \hookrightarrow S^{-1}A$ data da $f(a \mapsto \frac{a}{1})$ è un omomorfismo di anelli (la verifica è immediata).
\end{proposition}
\begin{proposition}{invertibili della localizzazione}
    $(S^{-1}A)^\times = \{ \frac{a}{s} \in S^{-1}A : \exists b \in A \ ab \in S \}$. Inoltre $\left(\frac{a}{s}\right)^{-1} = \frac{bs}{ab}$ con $b$ tale che $ab \in S$.
\end{proposition}
\begin{proof}
    Un elemento $\frac{a}{s}$ della localizzazione è invertibile sse $\exists \frac{b}{t} \in S^{-1}A$ tale che $\frac{a}{s} \cdot \frac{b}{t} = \frac{ab}{st} = \frac{1}{1}$, cioè $ab = st$, che accade se e soltanto se esiste in $S$ un multiplo di $a$.
\end{proof}
\begin{theorem}{ideali della localizzazione}
    gli ideali di $S^{-1}A$ sono tutti e soli gli insiemi della forma $S^{-1}I$ con $I$ ideale di $A$.
\end{theorem}
\begin{proof}
    Se $I$ è un ideale di $A$ allora $S^{-1}I = \{ \frac{x}{s} : x \in I\}$ è un ideale di $S^{-1}A$. La proprietà di assorbimento di $S^{-1}I$ segue da quella di $I$ e da $S$ parte moltiplicativa. Inoltre $S^{-1}I$ è un sottogruppo additivo, infatti:
    \begin{itemize}
        \item $0 \in I \Rightarrow \frac{0}{1} \in S$;
        \item $\frac{x}{s} \in S^{-1}I  \Rightarrow \frac{-x}{s} = - \frac{x}{s} \in S^{-1}I$ perché $x \in I \Rightarrow -x \in I$;
        \item $ \frac{x}{s}, \frac{y}{t} \in S^{-1}I  \Rightarrow \frac{x}{s}+\frac{y}{t} =\frac{xt+ys}{st} \in S^{-1}I$ perché $xt, ys \in I$ per la proprietà di assorbimento di $I$ e $st \in S$ perché $S$ parte moltiplicativa.
    \end{itemize}
    Dimostriamo ora che per ogni $J$ ideale di $S^{-1}A$ esiste un $I$ in $A$ tale che $J = S^{-1}I$. Consideriamo l'immersione $f: A \hookrightarrow S^{-1}A$ data da $f(a \mapsto \frac{a}{1})$. Sia $I = f^{-1}(J)$ la controimmagine di $J$ tramite l'omomorfismo $f$. Allora $S^{-1}I = \{ \frac{x}{s} : f(x) \in J,  \ s \in S\} = \{ \frac{x}{s} : \frac{x}{1} \in J, \ s \in S\}$. Ma $\forall x \in A, s \in S$ vale $\frac{x}{1} \in J \iff \frac{x}{s} \in J$ (si moltiplica rispettivamente per $\frac{s}{1}$ e $\frac{1}{s}$), dunque $S^{-1}I = J$.
    
    Dall'ultimo ragionamento segue che $f^{-1}(J) = f^{-1}(J \cap f(A))$, quindi moralmente ``$f^{-1}(J) = J \cap A$''.
    \end{proof}
\begin{theorem}{ideali primi della localizzazione}
    esiste una corrispondenza biunivoca tra gli ideali primi di $S^{-1}A$ e gli ideali primi di $A$ disgiunti da $S$.
\end{theorem}
\begin{proof}
    Dimostriamo innanzitutto che se $P$ è un ideale primo di $A$ disgiunto da $S$, allora $S^{-1}P$ è un ideale primo di $S^{-1}A$. Si nota intanto $S^{-1}P \neq S^{-1}A$, infatti $S^{-1}P = S^{-1}A \iff \frac{1}{1} \in S^{-1}P \iff S \cap P \neq \emptyset$. Inoltre $S^{-1}P$ è un ideale per il teorema precedente. Mostriamo che è primo: dato $\frac{x}{s} \frac{y}{t} = \frac{xy}{st} = \frac{p}{r} \in S^{-1}P$ si ha $xyr = pst \in P$ per assorbimento, dunque per primalità vale almeno una tra $x \in P$, $y \in P$ e $r \in P$. Sappiamo $r \notin P$ perché $r \in S$ e $S \cap P = \emptyset$, segue $x \in P \lor y \in P$ e quindi $\frac{x}{s} \in S^{-1}P \lor \frac{y}{t} \in S^{-1}P$.
    
    Consideriamo ora l'immersione $f : A \hookrightarrow S^{-1}A$ data da $f(a \mapsto \frac{a}{1})$.
    Siano $\mathcal{P} = \{P \tri A \text{ primo} : P \cap S = \emptyset\}$ e $\mathcal{Q} = \{Q \tri S^{-1}A : Q \text{ primo}\}$.
    Consideriamo le due funzioni $\alpha: \mathcal{P} \rightarrow \mathcal{Q}$ definita da $\alpha(P \mapsto S^{-1}P)$ e $\beta: \mathcal{Q} \rightarrow \mathcal{P}$ definita da $\beta(Q \mapsto f^{-1}(Q) = Q \cap A)$. Per quanto sopra, $\alpha$ è ben definita. $\beta$ è ben definita perché la controimmagine di un ideale primo è un ideale primo e la controimmagine di ideali propri di $S^{-1}A$ ha intersezione banale con $S$.
    $\alpha \circ \beta = id$ per quanto visto nel teorema precedente. Mostriamo anche $\beta \circ \alpha = id$, ossia $\forall P \in \mathcal{P} \ P = f^{-1}(S^{-1}P)$. Il contenimento $P \subseteq f^{-1}(S^{-1}P)$ segue da $x \in P \Rightarrow \frac{x}{1} \in S^{-1}P \Rightarrow x \in f^{-1}(S^{-1}P)$. Per quello inverso osserviamo che $x \in f^{-1}(S^{-1}P) \Rightarrow \exists s \in S \ \frac{x}{s} = \frac{p}{r} \in S^{-1}P$ con $p\in P, r \in S$, quindi $xr = ps \in P$ e, poiché $r \in S$ e $S \cap P = \emptyset$, necessariamente $x \in P$. Quindi $f^{-1}(S^{-1}P) \subseteq P$.
\end{proof}
\begin{proposition}{ideali primi e parti moltiplicative}
    dato $P$ ideale $P$ è primo $\iff$ $A \setminus P$ è una parte moltiplicativa.
\end{proposition}
\begin{proof}
    Sia $S = A \setminus P$. Da $0 \in P$ e $1 \notin P$ seguono $0 \notin S$ e $1 \in S$.
    $P$ è primo $\iff \forall x,y \in A \ (xy \in P \Rightarrow x \in P$ o $y \in P) \iff \forall x,y \in A \ (x \not \in P \text{ e } y \not \in P \Rightarrow xy \not \in P) \iff \forall x,y \in A \ (x \in S \text{ e } y \in S \Rightarrow xy \in S) \iff \forall x,y \in A \ (x \in S \text{ e } y \in S \Rightarrow xy \in S) \iff S$ è un semigruppo moltiplicativo, quindi sse $S$ è una parte moltiplicativa. 
\end{proof}
\begin{definition}{localizzazione}
    se $P$ è un ideale primo, $S = A \setminus P$ allora $A_P = S^{-1}A$ è detto il localizzato di $A$ a $P$; $A_P$ è un anello locale, ossia ha un solo ideale massimale.
\end{definition}
\begin{definition}{campo dei quozienti}
    Sia $A$ è un dominio e $S = A \setminus \{0\}$. Chiamiamo l'anello $K = S^{-1}A$ ``campo dei quozienti di A''. $K$ è un campo ed è minimo per inclusione tra i campi che contengono $A$.
\end{definition}

\subsection{UFD, PID, ED}

\begin{definition}{UFD}
    un dominio a fattorizzazione unica (o UFD: Unique Factorization Domain) è un dominio in cui per ciascun elemento non invertibile esiste una unica fattorizzazione come prodotto di elementi primi; la fattorizzazione si intende unica a meno di associati e ordine dei fattori.
\end{definition}
\begin{definition}{PID}
     un dominio a ideali principali (o PID: Principal Ideals Domain) è un dominio in cui tutti gli ideali sono principali.
\end{definition}
\begin{definition}{ED}
    un dominio euclideo (o ED: Euclidean Domain) è un dominio in cui 
    \begin{itemize}
        \item esiste una funzione grado $d : A \setminus \{0\} \rightarrow \N$ tale che $\forall a,b\in A\setminus \{0\}\ d(a) \leq d(ab)$;
        \item  $\forall a,b\in A, b\neq 0 \ \exists q,r \in A$ tali che $a = qb + r$ e $r = 0$ oppure $d(r) < d(b)$.
    \end{itemize}
    Si noti che la seconda condizione parla della divisione (appunto) euclidea.
\end{definition}
\begin{theorem}{caratterizzazione degli UFD}
    $A$ UFD $\iff$ $(i)$ ogni irriducibile è primo e $(ii)$ ogni catena discendente di divisibilità è stazionaria (equivalentemente, ogni catena ascendente di ideali principali è stazionaria. $(ii)$ è nota anche come Ascending Chain Condition on Principal ideals (ACCP)).

    Le condizioni sopra garantiscono rispettivamente l'unicità e l'esistenza della fattorizzazione in $A$.
\end{theorem}
\begin{proof}
    (*) Ad algebra 2. Riportiamo qui l'implicazione $(i) \land (ii) \implies A$ UFD. Se per assurdo esistesse un elemento $d \in A$ che non è prodotto di irriducibili, $d$ sarebbe necessariamente riducibile, dunque $d = a_1 x_1$ con $a_1$ e $x_1$ entrambi non unità e almeno uno dei quali, diciamo $x_1$, non è prodotto di irriducibili. Allo stesso modo $x_1 = a_2 x_2$. Proseguendo ricorsivamente si otterrebbe la catena di ideali principali $(d) \subsetneq (x_1) \subsetneq (x_2) \subsetneq \dots$: contro $(ii)$. Dunque ogni elemento si esprime come prodotto di irriducibili. Per quanto riguarda l'unicità, date $p_1 p_2 \dots p_n = q_1 q_2 \dots q_m$ due fattorizzazioni di uno stesso elemento si avrebbe per primalità $p_1$ associato a un $q_i$, diciamo $p_1 \sim q_1$, quindi per cancellazione $p_2 p_3 \dots p_n \sim q_2 q_3 \dots q_m$. Ma allora $n = m$ e a meno di riordinare gli indici $p_i \sim q_i$ per ogni $i = 1 \dots n$, cioè la fattorizzazione è unica a meno di associati.
\end{proof}
\begin{example}
    Usando la caratterizzazione si mostra che $A = \mathbb{K}[ \{ \sqrt[n]{x} : n \ge 1 \} ]$ non è UFD, infatti esiste la catena discendente di divisibilità $\{ x^{\frac{1}{2^n}} \}_{n \ge 1}$ per cui $x^{\frac1{2^{n+1}}} \mid x^{\frac1{2^n}}$.
\end{example}
\begin{theorem}{ideali primi in un PID}
    $A$ PID $\Rightarrow$ gli unici ideali primi sono $(0)$ e i massimali.
\end{theorem}
\begin{proof}
    $(0)$ è primo in ogni dominio e i massimali sono primi in ogni anello. Sia ora $P = (x), P \neq \{ 0 \}$ ideale primo. $x$ primo $\Rightarrow$ $x$ irriducibile, dunque $(x)$ massimale nella classe degli ideali principali, che in un PID significa massimale.
\end{proof}
\begin{proposition}{come costruire un grado}
    prima di tutto si individuano gli invertibili e si assegna ad essi il grado 1. Si procede induttivamente, individuando gli elementi tali che dividere per essi dia come possibili resti solo 0 o elementi a cui è stato già assegnato un grado $< k \in \N$ e si assegna ad essi il grado $k$.
\end{proposition}
\begin{proof}
    Ad algebra 2 vedremo i dettagli.
\end{proof}
\begin{theorem}{inclusioni}
    $ED \Rightarrow PID \Rightarrow UFD$.
\end{theorem}
\begin{proof}
    Per $PID \Rightarrow UFD$ usiamo la caratterizzazione degli $UFD$ vista prima. 
    \begin{itemize}
        \item[$(i)$] Sia $x$ irriducibile, allora $(x)$ massimale nella classe degli ideali principali di $A$ PID, dunque massimale, quindi $x$ primo.
        \item[$(ii)$] Sia $(a_1) \subseteq (a_2) \subseteq \dots$ una catena ascendente di ideali. Allora $I = \bigcup_{ns \in \N} a_n$ è un ideale. Poiché $A$ è PID, è principale $I = (x)$ con $x \in A$. Poiché $x$ appartiene all'unione degli $(a_n)$ esiste un $n_0$ per cui $x \in (a_{n_0})$. Ma allora $(x) \subseteq (a_{n_0}) \subseteq (x) \Rightarrow (a_{n_0}) = (x)$ e quindi la catena è stazionaria da $n_0$ in poi, infatti $\forall n \ge n_0 \ (x) = (a_{n_0}) \subseteq (a_n) \subseteq (x)$.
    \end{itemize}
    Per $ED \Rightarrow PID$ consideriamo un generico ideale $I$ di $A$ e dimostriamo che è generato da un qualsiasi suo elemento di grado minimo in $I$, sia esso $x$. Dato $y \in I$, per divisione euclidea esistono $q,r \in A$ tali che $y = qx+r$ con $r = 0 \lor d(r) < d(x)$, ma $r = y-qx \in I$ non può avere grado minore di $x$, dunque $r = 0$, ossia $y \in (x)$. Quindi $I \subseteq (x) \subseteq I$, cioè $I = (x)$.
\end{proof}
\begin{definition}{massimo comune divisore}
    Siano $a, b \in A$ non entrambi nulli. Un elemento $d \in A$ si dice ``massimo comune divisore'' $MCD(a, b)$ di $a$ e $b$ se per ogni $c$ vale l'implicazione $c \mid a \land c \mid b \implies c \mid d$. Si noti che l'MCD è definito a meno di associati.

    Altre formulazioni equivalenti si adattano alla struttura con cui stiamo lavorando, si dimostra che sono tutte consistenti.
    \begin{itemize}
        \item in un UFD, siano $p_1,\dots,p_s$ i primi che dividono almeno uno tra $a$ e $b$ e siano $\alpha_i, \beta_i \in \N$ gli esponenti dei $p_i$ nella fattorizzazione rispettivamente di $a$ e di $b$. Allora $MCD(a,b) = \prod_{i=1}^s p_i^{\min \{ \alpha_i,\beta_i \} }$;
        \item in un PID, $MCD(a,b)$ è l'elemento $d$ che soddisfa $(d) = (a,b)$;
        \item in un ED, $MCD(a,b)$ è il risultato dell'algoritmo di Euclide (che termina sempre) applicato ad $a$ e $b$.
    \end{itemize}
\end{definition}
\begin{proof}
    \underline{sono consistenti.} 
    Sia $A$ PID; per quanto visto prima $A$ è anche UFD. Consideriamo $d = MCD(a,b)$ secondo la definizione negli UFD. Dobbiamo dimostrare che vale anche $(d) = (a,b)$ e quindi $d$ coincide con quello trovato dalla definizione per i PID. Guardiamo la fattorizzazione degli elementi in $(a,b) = \{ax+by : x,y \in A\}$. Possiamo sicuramente isolare dai due addendi $\prod_{i=1}^s p_i^{\min\{\alpha_i,\beta_i\}}$, da cui segue che $(a,b) \subseteq (d)$. Per l'altro contenimento, osserviamo che $A$ PID $\Rightarrow \exists d' \in A \ (a,b) = (d') \subseteq (d) \Rightarrow d' \mid \prod_{i=1}^s p_i^{\min\{\alpha_i,\beta_i\}}$. Se però almeno uno degli esponenti fosse più piccolo, diciamo quello di $p_1$ avrei un assurdo perché esisterebbero $x,y \in A$ $d' = ax+by$ e basta allora guardare l'equazione modulo $p_1^{\min\{\alpha_1,\beta_1\}}$. Quindi $d' = d$.
    
    Sia ora $A$ ED; per quanto visto prima $A$ è anche PID. Sia $d$ tale che $(d) = (a,b)$. Se l'algoritmo di Euclide termina subito (cioè se $a = qb$), allora è chiaro $(b) = (a, b) = (d)$. Se invece $a = qb + r$ con $r \neq 0$, induciamo sul numero di passi dell'algoritmo. Per ogni $x$ vale $(x \mid b \land x \mid r) \iff (x \mid b \land x \mid a)$. Sia allora $\tilde d$ l'elemento trovato dall'algoritmo applicato a $b$ e $r$, per ipotesi induttiva $(\tilde d) = (b, r)$. Allora $\tilde d \mid a \land \tilde d \mid b$, da cui $\tilde d \mid d$, ma anche $d \mid b \land d \mid r$, da cui $d \mid \tilde d$, dunque $(d) = (\tilde d)$.
\end{proof}
\begin{theorem}{Bezout} 
    Dato $A$ PID $\forall a,b \in A \ \exists x,y \in A$ tali che $ax + by = MCD(a,b)$.
\end{theorem}
\begin{proof}
    segue dalla definizione di ideale generato e di massimo comune divisore nei PID.
\end{proof}
\begin{example2}{dominio non UFD}
    $\Z[\sqrt{-5}]$. Si usa che può essere dotato di una norma moltiplicativa (quella standard) e che $6 = 2 \cdot 3 = (1 + \sqrt{-5})(1 - \sqrt{-5})$, tutti primi.
\end{example2}
\begin{example2}{UFD non PID}
    $\Z[x]$ è UFD perché $\Z$ lo è, ma $(2,x)$ non è principale. O anche $\Q[x,y]$ è UFD per lo stesso motivo e $(x,y)$ non è principale. Si noti che su $\Z[x]$ non vale il teorema di Bezout: $1 = MCD(2, x)$ non si scrive come combinazione lineare di $2$ e $x$.
\end{example2}
\begin{example2}{PID non ED}
    $\Z[\frac{1 + \sqrt{-19}}{2}]$.
    La dimostrazione è molto lunga e a lezione viene solo data l'idea. Se interessati a una trattazione dettagliata potete seguire questo \href{https://www.jstor.org/stable/2322908}{link}.
\end{example2}

\begin{proposition}{(*) MCD ed estensioni}
    Siano $A \subset B$ due UFD e $(a, b) \in A$. Indichiamo con $d_A = MCD_A(a, b)$ il massimo comun divisore di $a$ e $b$ su $A$, con $d_B$ quello su $B$. Vale sempre $d_B \mid_B d_A$. Se inoltre su $A$ o su $B$ vale Bezout, allora anche $d_A \mid_B d_B$, ma l'ultima divisibilità è falsa in generale.
\end{proposition}
\begin{proof}
    Siano $a = d_A x$ e $b = d_A y$ con $x, y \in A$. Allora $MCD_B(a, b) = MCD_B(d_A x, d_A y) = d_A MCD_B(x, y)$, che è multiplo di $d_A$. Se in $A$ vale Bezout, esistono $\alpha, \beta \in A \ \alpha a + \beta b = d_A$. Da $d_B \mid_B a, b$ allora segue anche $d_B \mid_B d_A$, cioè $d_B$ e $d_A$ sono associati in $B$.

    Senza ipotesi aggiuntive l'ultima affermazione è falsa: consideriamo $\Z[a, b]$, $MCD(a, b) = 1$ e l'immersione $\Z[a, b] \hookrightarrow \Z[a, b, x]$ tramite $a, b \mapsto ax, bx$. $MCD_{\Z[a, b, x]}(ax, bx) = x$, che non è unità.

    \textbf{Domanda:} Vale in qualche senso anche l'inverso? È vero, per esempio, che dato $A$ UFD, se per ogni $B$ UFD estensione di $A$ vale $\forall (a, b) \in A \ d_B \mid_B d_A$, allora in $A$ vale Bezout? Se trovate una risposta, per favore fateci sapere.
\end{proof}

\subsection{Anelli di polinomi}

\begin{definition}{$A[x]$}
    $A[x]$ è l'anello dei polinomi a coefficienti in $A$.
\end{definition}

\begin{proposition}{ideali primi}
    se $P$ è un ideale primo di $A$ allora $P[x]$ è un ideale primo di $A[x]$. 
\end{proposition}
\begin{proof}
    basta notare che $\frac{A[x]}{P[x]} \cong \frac{A}{P}[x]$ e quindi $P$ primo $\iff \frac{A}{P}$ dominio $\Rightarrow \frac{A}{P}[x]$ dominio $\iff \frac{A[x]}{P[x]}$ dominio $\iff P[x]$ primo.
\end{proof}
\begin{proposition}{invertibili}
    $A[x]^\times = \{f(x) = \sum_{i= 0}^n a_ix^i \in A[x] : a_0 \in A^\times, \ a_1,\dots,a_n \in \sqrt{(0)} \}$.
\end{proposition}
\begin{proof}
    Dimostriamo i due contenimenti.
    
    \textbf{Lemma}: $\sqrt{(0)}[x] \subseteq \sqrt{(0)[x]}$
    \begin{proof}
        sia $f(x) \in \sqrt{(0)}[x]$. Poiché $f(x)$ ha finiti coefficienti tutti nilpotenti, esiste un esponente $M$ tale che tutti i coefficienti elevati a quel numero diano $0$. Allora sviluppando con il multinomio di Newton si ha che $f(x)$ elevato alla $\text{deg}(f)M$ fa il polinomio nullo, da cui segue $f(x) \in \sqrt{(0)[x]}$. 
    \end{proof}
    Sia $f(x) = \sum_{i= 0}^n a_ix^i \in A[x] $ tale che $a_0 \in A^\times, \ a_1,\dots,a_n \in \sqrt{(0)}$. Allora $f(x) = a_0 - xg(x)$ dove $g(x) \in \sqrt{(0)}[x] \subseteq \sqrt{(0)[x]}$. Quindi esiste un esponente $M \in \N$ tale che $g(x)^M = 0$. Scegliamo senza perdita di generalità $M$ dispari (se un esponente funziona, chiaramente funzionano tutti quelli maggiori o uguli a lui). Sia $h(x) = a_0^{-1}xg(x)$. Chiaramente anche $h(x)^M = 0$ Da $1 = 1- h(x)^M = (1-h(x))(1+h(x)+ \dots + h(x)^{M-1})$ segue che $f(x)a_0^{-1}(1+h(x)+ \dots + h(x)^{M-1}) = (a_0 +a_0h(x))a_0^{-1}(1+h(x)+ \dots + h(x)^{M-1})=(1+h(x))(1+h(x)+ \dots + h(x)^{M-1})=1$ e quindi $f(x)$ è invertibile. Ciò dimostra $A[x]^\times \supseteq \{f(x) = \sum_{i= 0}^n a_ix^i \in A[x] : a_0 \in A^\times, \ a_1,\dots,a_n \in \sqrt{(0)}$.
    
    Per l'altro contenimento consideriamo $r$ tale che $f(x)r(x) = 1$. Allora $f(0)r(0) = 1$ e quindi $a_0 \in A^\times$. Prendiamo ora un qualsiasi ideale primo $P$ di $A$.  $P[x]$ è primo per il lemma precedente e $\frac{A[x]}{P[x]} \cong \frac{A}{P}[x]$. Consideriamo l'uguaglianza $f(x)r(x) = 1$ in $\frac{A}{P}[x]$. Poichè quest'ultimo è un dominio, si ha che $\overline f(x) \overline r(x) = \overline 1 \Rightarrow \overline f(x) \in (\frac{A}{P}[x])^\times \Rightarrow \overline f(x) \in (\frac{A}{P})^\times$ e quindi $\overline f(x)$ è una costante. Da ciò segue che tutti i coefficienti diversi da $a_0$ sono in $P$, ossia $f(x) - a_0 \in P[x]$. Ma allora $f(x) - a_0 \in \bigcap_{P \tri A \text{ primo}} P[x] = \sqrt{(0)}[x]$. Da ciò segue $A[x]^\times \subseteq \{f(x) = \sum_{i= 0}^n a_ix^i \in A[x] : a_0 \in A^\times, \ a_1,\dots,a_n \in \sqrt{(0)}$
\end{proof}
\begin{definition}{contenuto}
    dato $A$ UFD, $f = \sum_{i= 0}^n a_ix^i \in A[x]$ chiamiamo contenuto di $f$ $c(f) = MCD(a_0,\dots, a_n)$.
\end{definition}
\begin{definition}{polinomio primitivo}
    dato $A$ UFD, $f  \in A[x]$ si dice primitivo se $c(f) = 1$.
\end{definition}
\begin{theorem}{lemma di Gauss}
    Sia $A$ UFD e $f,g \in A[x]$. Allora $c(fg) = c(f)c(g)$.
\end{theorem}
\begin{proof}
    Consideriamo prima di tutto il caso in cui sia $f$ che $g$ sono primitivi. Sia $f(x) = \sum_{i=0}^n a_i x^i$, $g(x) = \sum_{i=0}^m b_ix^i$. Allora $f(x)g(x) = \sum_{k=1}^{n+m} x^k\big( \sum_{i = 0}^k a_ib_{k-i} \big)$. Sia ora $p$ un qualsiasi primo, e siano $n_0,m_0$ i minimi interi tali che $a_{n_0}, b_{m_0}$ non siano multipli di $p$ (devono esistere altrimenti non si potrebbe avere $c(f)= c(g) = 1$). Allora si ha $\sum_{i = 0}^{n_0+m_0} a_ib_{k-i} \equiv a_{n_0}b_{m_0} \not \equiv 0 \pmod{p}$ e quindi $p \nmid c(fg)$. Quindi anche $c(fg)=1$ poiché non è diviso da nessun primo.
    
    Nel caso generale, consideriamo $f_1, g_1 \in A[x]$ primitivi e tali che $f = c(f)f_1, g = c(g)g_1$. Allora si ha $c(fg) = c(c(f)c(g)f_1g_1) = c(f)c(g)c(f_1g_1) = c(f)c(g)$ per quanto detto.
\end{proof}
\begin{corollary}{1}
    Sia $A$ UFD, $K$ il suo campo dei quozienti e $f,g \in A[x]$ con $g$ primitivo tali che $g(x) \mid f(x)$ in $K[x]$. Allora $g(x) \mid f(x)$ in $A[x]$.
\end{corollary}
\begin{proof}
    Sia $h(x) \in K[x]$ tale che $f(x) = g(x)h(x)$ e siano $a \in A, h_1 \in A[x]$ tali che $ah(x) = h_1(x)$ (basta considerare i coefficienti di $h$ come ridotti ai minimi termini e prendere il mcm dei denominatori). Allora si ha $af(x) = g(x)h_1(x)$. Applichiamo il lemma di Gauss e otteniamo che $ac(f) = c(g)c(h_1) = c(h_1)$, e quindi $a \mid c(h_1)$, da cui segue $h(x) = \frac{h_1(x)}{a} \in A[x]$, come voluto.
\end{proof}
\begin{corollary}{2}
    Sia $A$ UFD, $K$ il suo campo dei quozienti, $f \in A[x]$ e $g,h \in K[x]$ tali che $f(x) = g(x)h(x)$. Allora $\exists g_1, h_1 \in A[x]$ con $\text{deg}(g_1) = \text{deg}(g)$, $\text{deg}(h_1) = \text{deg}(h)$, e $f(x) = g_1(x)h_1(x)$.
\end{corollary}
\begin{proof}
    Come prima siano $a,b\in A, g_0,h_0 \in A[x]$ tali che $ag(x) = g_0(x)$. Sia $g_0(x) = c(g_0)g_1(x)$ con $g_1(x)$ primitivo. Allora $g_1(x) \mid f(x)$  in $K[x]$ e siamo nelle ipotesi del corollario 1, e quindi $g_1(x) \mid f(x)  in A[x]$. Da $af(x) = ag(x)h(x) = c(g_0)g_1(x)h(x)$ segue che allora possiamo definire $h_1(x) := a^{-1}c(g_0)h(x) \in A[x]$ e abbiamo la tesi.
\end{proof}
\begin{theorem}{equivalenza interessante}
    $A$ campo $\iff A[x]$ PID.
\end{theorem}
\begin{proof}
    Supponiamo $A[x]$ PID. Notiamo intanto che $A[x]$ dominio $\Rightarrow A$ dominio, essendo $A$ un sottoanello di $A[x]$. Consideriamo l'omomorfismo di valutazione in $0$ $\psi_0: A[x] \rightarrow A$. Chiaramente è suriettivo (basta guardare i polinomi costanti) e $\ker(\psi_0) = (x)$, da cui segue $A[x]/(x) \cong A$. Poiché l'immagine $A$ è un dominio, necessariamente l'ideale $(x)$ è primo, e quindi, essendo $\neq (0)$, per quanto visto sugli ideali primi nei PID è un ideale massimale. Allora per $A[x]/(x) \cong A$ vale che $A$ è un campo.
    
    Sia ora $A$ un campo. Allora $A[x]$ è un ED con grado dato dal grado del polinomio (dimostrazione vista ad aritmetica), e quindi in particolare è un PID. 
\end{proof}

\begin{theorem}{irriducibili in $A[x]$}
    Se $A$ è UFD gli irriducibili di $A[x]$ sono tutti e soli gli elementi $f$ tali che o $f \in A$ irriducibile oppure $\text{deg}(f) \geq 1$, $c(f) = 1$ e $f$ irriducibile in $K[x]$.
\end{theorem}
\begin{proof}
    Chiaramente $f(x) = g(x)h(x)$ con $g(x)h(x) \in A[x]$ $\Rightarrow \text{deg}(g),\text{deg}(h) \leq \text{deg}(f)$.
    Quindi se $\text{deg}(f) = 0$ per forza se si fattorizza $f$ in $A[x]$ la fattorizzazione deve contenere solo costanti, ossia elementi di $A$, e quindi è irriducibile se e solo se è irriducibile in $A$.
    
    Se invece $\text{deg}(f) \geq 1$ distinguiamo i casi. Se $c(f) \neq 1$ allora $f(x) = c(f)f_1(x)$ con $f_1(x)$ primitivo non costante. Nessuno dei due termini è invertibile e quindi $f$ non è irriducibile. Se invece $c(f) = 1$ notiamo intanto che $f$ irriducibile in $K[x] \Rightarrow f(x)$ irriducibile in $A[x]$ (una fattorizzazione in $A[x]$ è valida anche in $K[x]$). Dimostriamo ora che irriducibile in $A[x] \Rightarrow$ irriducibile in $K[x]$. Se $f$ riducibile in $K[x]$ esistono $g,h \in K[x]$ tali che $\ \leq \text{deg}(g)\text{deg}(h) < \text{deg}(f)$ e $f = gh$ e quindi , per il secondo corollario del lemma di Gauss esisterebbero $g_1,h_1 \in A[x]$ tali che $\text{deg}(g_1) = \text{deg}(g)$, $\text{deg}(h_1) = \text{deg}(h)$, e $f(x) = g_1(x)h_1(x)$, da cui $f$ è riducibile anche in $A[x]$. Quindi se $f$ è un polinomio primitivo non costante, è irriducibile in $A[x]$ se e solo se lo è in $K[x]$. 
\end{proof}
\begin{theorem}{polinomi UFD}
    $A$ UFD $\Rightarrow A[x]$ UFD.
\end{theorem}
\begin{proof}
    Sia $K$ il campo dei quozienti di $A$ e sia $f(x) \in A[x]$. Allora $f(x) \in K[x]$ che per quanto visto è un PID e quindi un UFD. Sia $f(x) = \prod_{i=1}^s g_i(x)^{\alpha_i}$ la fattorizzazione di $f(x)$ in $K[x]$ (wlog $g_i$ tutti non costanti). Dal corollario 2 del lemma di Gauss segue che esistono $h_1, \dots, h_s \in A[x]$ tali che $\text{deg}(g_i) = \text{deg}(h_i) \ \forall i = 1,\dots,s$ (si nota che dalla dimostrazione del lemma e del fatto che i $g_i$ erano irriducibili in $K[x]$ segue che anche gli $h_i$ sono irriducibili in $K[x]$) e $f(x) = \prod_{i=1}^s h_i(x)^{\alpha_i} = \prod_{i=1}^s (c(h_i)k_i(x))^{\alpha_i} = \prod_{i=1}^s c(h_i)^{\alpha_i} \prod_{i=1}^s k_i(x)^{\alpha_i}$ dove i $k_i$ sono polinomi primitivi non costanti. Poiché essi sono primitivi e irriducibili in $K[x]$ (per costruzione) sono anche irriducibili in $A[x]$. Il fatto che la fattorizzazione $\prod_{i=1}^s k_i(x)^{\alpha_i}$ sia l'unica possibile per il secondo termine segue dall'unicità della fattorizzazione in $K[x]$. Inoltre $\prod_{i=1}^s c(h_i)^{\alpha_i} \in A \Rightarrow $ possiede un'unica fattorizzazione in irriducibili poiché $A$ è UFD. Quindi i due termini che compongono $f(x)$ hanno ciascuno un fattorizzazione unica, e metterle insieme fornisce una e una sola fattorizzazione per $f(x)$ perché in un caso stiamo considerando solo irriducibili di $A[x]$ in $A$ e nell'altro caso solo irriducibili in $A[x] \setminus A$. 
\end{proof}
\begin{theorem}{criterio di Eisenstein}
    Sia $A$ UFD, $f(x) = \sum_{i=0}^n a_i x^i \in A[x]$ primitivo e $p\in A$ un primo tale che $(i)$ $p \nmid a_n$, $(ii)$ $p \mid a_i$ per $i=0,\dots,n-1$, $(iii)$ $p^2 \nmid a_0$. Allora $f(x)$ è irriducibile in $A[x]$ (e quindi anche in $K[x]$ per uno dei teoremi precedenti). 
\end{theorem}
\begin{proof}
    Chiaramente si deve avere $\text{deg}(f) = n\geq 1$. Supponiamo che esistano $g,h \in A[x]$ tali che $f(x) = g(x)h(x)$. Vediamo l'equazione di prima modulo $p$, considerando le immagini dei polinomi nell'anello $\frac{A}{(p)}[x] \cong \frac{A[x]}{(p)[x]}$. Per ipotesi: $a_nx^n \equiv \overline f(x) = \overline g(x) \overline h(x) \pmod{p}$. Poiché se il prodotto di due polinomi è un monomio allora sono entrambi monomi (basta guardare i termini di grado minimo e massimo nel prodotto e notare che i gradi devono coincider), l'unica possibilità è $\overline g(x)= bx^j, \overline h(x)= cx^k$ con $j+k = n$ e $b,c \not \equiv 0 \pmod{p}$ (poiché $bc \equiv a_n \not \equiv 0 \pmod{p}$). Allora si ha $g(x)= bx^j + pg_1(x), h(x)= cx^k + ph_1(x)$ dove $g_1,h_1 \in A[x]$. Notiamo ora che se $j,k \geq 1$ $a_0 f(0) = g(0)h(0) = pg_1(0)\cdot ph_1(0) = p^2 g_1(0)h_1(0)$ e si ha quindi un assurdo.
    Quindi almeno uno tra $j$ e $k$ è uguale a 0. Supponiamo senza perdita di generalità $j$. Allora si ha anche $k = n-j = n$. Guardiamo i gradi dei polinomi. $f(x) = g(x)h(x) \Rightarrow n = \text{deg}(f) = \text{deg}(g)+\text{deg}(h)$ Poiché $h = cx^n + ph_1(x)$ e $p \nmid c$, necessariamente $\text{deg}(h) \geq n \Rightarrow \text{deg}(h) = n, \text{deg}(g) = 0$, che è assurdo perché allora $g$ è una costante che divide $f$, che contraddice il fatto che $f$ sia primitivo.
\end{proof}

\subsection{L'anello $\Z[x]$}

\begin{proposition}{ideali primi e massimali}
    Gli ideali primi di $\Z[x]$ sono della forma: $(p), (p, \mu(x)), (\lambda(x))$ dove $p$ è un primo, $\mu(x), \lambda(x)$ sono polinomi rispettivamente irriducibili in $\mathbb{F}_p[x]$ e $\Z[x]$. Gli ideali massimali, invece, sono solo quelli della forma $(p,\mu(x))$ con $p, \mu(x)$ come prima. 
\end{proposition}
\begin{proof}
    Sia $M \subseteq \Z[x]$ un ideale massimale. Notiamo che massimale $\Rightarrow$ primo quindi caratterizziamo prima gli ideali primi, poi vediamo se possono essere massimali. Consideriamo l'immersione $\iota: \Z \rightarrow \Z[x]$. $\iota$ è un omomorfismo, quindi controimmagine di ideale primo è ideale primo: $\iota^{-1}(M) = M \cap \Z \subseteq \Z$ è ideale primo. Poiché $\Z$ è PID, gli ideali primi sono $(0)$ e i massimali, dove i massimali sono della forma $(p)$ con $p$ primo. Distinguiamo i casi.
    \begin{itemize}
        \item $M\cap \Z = (p)$: proiettiamo $M$ in $ \Fp [x]$ (usiamo che $\frac{\Z[x]}{(p)[x]} \cong \frac{\Z}{(p)}[x] = \Zp[x] = \Fp[x]$). Poiché la proiezione, che indichiamo con $\pi_p$, è un omomorfismo suriettivo, immagine di ideale primo è ideale primo. $\Fp [x]$ è PID, dunque vale una tra $\pi(M) = (0)$ e $\pi_p(M) = (\overline{\mu(x)}) $ con $\mu(x)$ irriducibile in $\Fp[x]$. Nel primo caso l'ideale è primo ma non massimale perché contenuto certamente in un ideale del secondo tipo, per esempio $(\overline{x})$. Nel secondo caso invece $M = \pi_p^{-1}(\pi_p(M)) = \pi_p^{-1}((\overline{\mu(x)})) = \mu[x] + \ker(\pi_p) = \mu[x]+(p) = (p,\mu(x))$. 
        \item $M\cap \Z = (0)$: prendiamo il campo dei quozienti di $\Z$, $\Q$. Dalla teoria sappiamo che $M$ ideale primo in $\Z[x]$ corrisponde a $S^{-1}M$ ideale primo in $\Q[x]$, dove $S=\Z\setminus \{0\}$. Quindi, poiché $\Q[x]$ è PID, $S^{-1}M = (\lambda(x))$ con $\lambda(x) \in \Q[x]$ irriducibile. Notiamo ora che per il lemma di Gauss $\forall p(x) \in \Z[x]$ primitivo si ha (indichiamo $(p(x))A[x]$ l'ideale generato da $p(x)$ in $A[x]$) $((p(x))\Q[x]) \cap \Z[x] = (p(x))\Z[x]$. Il lemma di Gauss ci dice infatti che se un polinomio a coefficienti interi è divisibile in $\Q$ per $p$, allora lo è anche in $\Z$: l'equazione riflette insiemisticamente questo enunciato.
        Scegliendo allora un rappresentante a coefficienti interi e primitivo $\lambda(x)$ si ha $M = ((\lambda(x))\Q[x]) \cap \Z[x] = (\lambda(x))\Z[x]$. Dimostriamo ora che gli ideali di questa forma non sono massimali. Sia $p$ un primo che non divide il coefficiente direttore di $\lambda(x)$. Allora $(\lambda(x)) \subsetneq (p, \lambda(x)) \subsetneq \Z[x]$, infatti $p \notin (\lambda(x))$ e $\frac{\Z[x]}{(p, \lambda(x))} \cong \frac{\Fp[x]}{(\overline{\lambda(x)})}$ non è l'anello banale poiché $\deg \overline{\lambda(x)} = \deg \lambda(x) > 0$. Dunque $(\lambda(x))$ non è massimale.
        % $(\lambda(x))$ è massimale $\iff \frac{\Z[x]}{(\lambda(x))}$ è campo. Poiché $\lambda(x)$ deve essere non costante, esiste un valore $x_0$ tale che $\lambda(x_0)\neq 0,\pm 1$. Esiste allora $q$ un primo che divide $x_0$. Se $\frac{\Z[x]}{(\lambda(x))}$ campo, allora esisterebbero $a(x),b(x) \in \Z[x]$ tali che $qa(x) + \lambda(x) b(x) = 1$. Ma valutando questa relazione in $x = x_0$ si avrebbe allora un assurdo guardando la divisibilità per $q$. 
    \end{itemize}
\end{proof}

\subsection{Esercizi}

\begin{exercise}
    Siano $A = \F_5[x]$, $I = (x^2 + 1)$ e $J = (x^3 - 1)$. Descrivere $I + J$, $IJ$ e $I \cap J$.
\end{exercise}
\begin{solution}
    $A$ è un PID, quindi i suoi ideali si indicano con un generatore. Eseguendo la divisione tra polinomi si trova $I + J = (I, J) = (x^2 + 1, x^3 - 1) = (1) = A$, cioè $I$ e $J$ coprimi. Per coprimalità allora anche $I \cap J = IJ = ((x^2+1)(x^3-1))$.
\end{solution}

\begin{exercise}
    Siano $A = \Q[x, y]$, $I = (x-1, y-1)$, $J = (1 - xy)$. Mostrare che $I$ è massimale, mentre $J$ non è massimale.
\end{exercise}
\begin{solution}
    Innanzitutto sia $I$ che $J$ sono ideali propri, infatti non contengono $1$. $I$ è l'ideale di tutti e soli i polinomi di $\Q[x, y]$ i cui coefficienti sommano a zero: ogni $f(x, y) = \sum_k {a_k x^{\alpha_k}y^{\beta_k}}$ si scrive come $f(x, y) = \sum_k {a_k (1 - (1 - x))^{\alpha_k}(1 - (1 - y))^{\beta_k}} = (\sum_k a_k) + r$ con $r \in I$, dunque $f \in I \iff \sum_k a_k = 0$. Ma allora $I$ è massimale, infatti dato $f(x, y) = \sum_k {a_k x^{\alpha_k}y^{\beta_k}} \notin I$ si ha $0 \neq \sum_k a_k \in (I, f)$, cioè $(I, f) = A$. $J$ non è massimale perché è strettamente contenuto in $I$ ideale proprio.
\end{solution}

\begin{exercise}
    Sia $A = \Q[x, y]$. Mostrare che $J = (1 - xy)$ è un ideale primo.
\end{exercise}
\begin{solution}
    $\Q$ UFD $\implies$ $\Q[x]$ UFD $\implies \Q[x][y] = \Q[x, y]$ UFD, in particolare ogni irriducibile di $Q[x, y]$ è anche primo. Mostriamo che $(1 - xy)$ è irriducibile, cioè che non è prodotto di fattori di grado 1. $(1 + \alpha x + \beta y)(1 + \gamma x + \delta y) = (1 - xy)$ implica $\alpha + \gamma = 0$ e $\alpha \gamma = 0$, da cui $\alpha = \gamma = 0$, analogo per $\beta = \delta = 0$, assurdo. Allora $1 - xy$ è irriducibile (quindi primo) e $J = (1 - xy)$ è un ideale primo.

    Possibile strada alternativa: trovare un omomorfismo da $\Q[x, y]$ in un dominio il cui kernel sia $J$. (Non garantisco porti da qualche parte. Qualcosa come $f(x, y) \mapsto f(x, x^{-1}) \in \Q(x)$ può funzionare?)
\end{solution}

\begin{exercise}
    Descrivere $\Q[x, y]/(x - y, x^3 + y^3 - x)$ come prodotto di campi.
\end{exercise}
\begin{solution}
    Dato un generico $f \in \Q[x, y]$ si ha $f(x, y) = \sum_k a_k x^{\alpha_k}y^{\beta_k} = \sum_k x^{\alpha_k} (x - (x - y))^{\beta_k} = f(x, x) + r(x, y)$ con $r \in (x - y)$. Consideriamo l'omomorfismo di anelli $\varphi : \Q[x, y] \to \Q[x]$ dato da $\varphi(f(x, y) \mapsto f(x, x))$, si verifica $\ker(\varphi) = (x - y)$, inoltre $\varphi$ è chiaramente surgettivo. Per il primo teorema di omomorfismo allora $\Q[x, y]/(x - y) \cong \Q[x]$. Dunque anche
    \[
        \frac{\Q[x, y]}{(x - y, x^3 + y^3 - x)} = \frac{\Q[x, y]}{(x - y, 2x^3 - x)} \cong \frac{\Q[x, y]/(x - y)}{(x - y, 2x^3 - x)/(x - y)} \cong \frac{\Q[x]}{(x(2x^2 - 1)}
    \]
    per il secondo teorema di omomorfismo. $(x)$ e $(2x^2 - 1)$ sono ideali coprimi di $\Q[x]$, dunque $(x) \cap (2x^2 - 1) = (x(2x^2 - 1))$, per il teorema cinese del resto e il primo teorema di omomorfismo allora $\Q[x]/(x(2x^2 - 1)) \cong \Q[x]/(x) \times \Q[x]/(2x^2 - 1) \cong \Q \times \Q[x]/(2x^2 - 1)$, che sono due campi poiché $2x^2 - 1$ è irriducibile in $\Q[x]$.
\end{solution}

\begin{exercise}
    Sia $S = \Z \setminus (2)$. Descrivere tutti gli ideali dell'anello $\Z_{(2)} := S^{-1}\Z$. Sia $i : \Z \hookrightarrow \Z_{(2)}$ l'immersione naturale. Dato un ideale $I$ di $\Z$, descrivere l'ideale $J$ generato da $i(I)$ e l'ideale $i^{-1}(J)$.
\end{exercise}
\begin{solution}
    $\Z_{(2)} = \{ \frac{a}{b} : a, b \in \Z, b \text{ dispari} \}$. Per il teorema sugli ideali primi della localizzazione $(2)$ è l'unico ideale primo di $\Z_{(2)}$. Gli ideali della localizzazione sono le localizzazioni degli ideali e la localizzazione di un ideale che contiene elementi di $S$ è l'intera localizzazione, dunque gli ideali propri non banali sono tutti e soli quelli nella forma $(2^n)$.
    In generale, dato $p$ primo dispari con $p \mid n$ si ha $\frac{n}{p} \in S^{-1}(n)$, dunque $S^{-1}(n) = S^{-1}(\frac{n}{p})$. Dato $I = (d 2^k) \tri \Z$ con $d$ dispari, allora $J = (i(I)) = (2^k)$ e $i^{-1}(J) = J \cap \Z = (2^k)$.
\end{solution}

\begin{exercise}
    Siano $A$ un anello commutativo e $K \subset A$ un campo, allora $A$ è un $K$-spazio vettoriale. Mostrare che se $\dim_K A < +\infty$ e $A$ è un dominio, allora $A$ è un campo. 
\end{exercise}
\begin{solution}
    Sia $r \in A \setminus \{0\}$ un generico elemento non zero. Se $A$ è un dominio, la moltiplicazione $x \mapsto rx$ è un'applicazione lineare iniettiva. In dimensione finita un endomorfismo iniettivo è anche suriettivo, dunque esiste $x \in A$ tale che $rx = xr = 1$, cioè ogni $r$ ammette un inverso (e un campo è per definizione un anello commutativo con gli inversi).
\end{solution}
\newpage
\section{Teoria dei campi}
Dove non diversamente specificato $\Omega$ è un campo qualsiasi con operazioni $+$ e $\cdot$ (il cui simbolo verrà omesso). L'elemento neutro della somma sarà indicato con $0$, quello del prodotto con $1$. $K,L,F,E$ saranno sempre sottocampi del campo ambiente $\Omega$. 

\subsection{Definizioni e richiami di Aritmetica}
\begin{definition}{estensione di campo}
    Si dice che $\Omega/K$ è un'estensione di campo (o semplicemente ``estensione'') di $K$ se $K$ è un sottocampo di $\Omega$. Si indica con $[\Omega : K] = \dim_K \Omega$ la dimensione di $\Omega$ come spazio vettoriale su $K$. L'estensione si dice finita se $[\Omega : K] < +\infty$.
\end{definition}
\begin{definition}{estensione semplice}
    dato $\alpha \in \Omega$ si indica con $K(\alpha)$ il sottocampo di $\Omega$ minimo per inclusione che contiene sia $K$ che $\alpha$.
\end{definition}
\begin{definition}{estensione generata}
    dato $S\subseteq \Omega$, si indica con $K(S) = \bigcap_{\substack{K \subseteq K' \subseteq \Omega \\ K' \text{campo}}} F$ l'estensione generata da S su K.
\end{definition}
\begin{definition}{composto di campi}
    dati $L,F$, si indica con $LF$ il minimo campo che li contiene entrambi, ossia $L(F) = F(L)$. Si dimostra
    \[
        F(L) = \left\{ \frac{p(\ell_1, \dots, \ell_n)}{q(\ell_1, \dots, \ell_n)} : \ell_i \in L, n \in \N, p,q \in F[x_1, \dots, x_n], q(\ell_1, \dots, \ell_n) \neq 0 \right\}.
    \]
\end{definition}

\begin{minipage}{0.7\textwidth}
    Sia $\alpha \in \Omega$. Indichiamo con $\varphi_{\alpha}$ l'omomorfismo di valutazione in $\alpha$, ossia $\varphi_{\alpha}: K[x] \rightarrow K[\alpha]$ tale che $\varphi_{\alpha}(f(x) \mapsto f(\alpha))$. Poiché $K[x]$ è un PID, $\ker(\varphi_\alpha)$ è principale, ossia $\exists \mu_{\alpha}(x) \in K[x]$ tale che $\ker(\varphi_\alpha) = (\mu_\alpha(x))$.
\end{minipage}\hfill
\begin{minipage}{0.3\textwidth}  
    \begin{center}
    \begin{tikzcd}
    K[x] \arrow{r}{\varphi_{\alpha}} \arrow{d}{\pi} & K(\alpha)\\
    \frac{K[x]}{\ker(\varphi_\alpha)} \arrow{ur}[swap]{\sim}
    \end{tikzcd}
    \end{center}
\end{minipage}\hfill

\begin{definition}{trascendente}
    se $\mu_{\alpha}(x)$ definito sopra è il polinomio nullo, $\alpha$ si dice trascendente su $K$.
    In questo caso $K(\alpha) \cong K[x]$ e $\alpha$ si comporta come un'indeterminata: un elemento che non si ``combina'' con nessun elemento diverso da se stessa.
    L'indeterminata $x$ è un elemento trascendente su qualsiasi campo.
\end{definition}
\begin{definition}{algebrico}
    se $\mu_{\alpha}(x)$ non è il polinomio nullo, $\alpha$ si dice algebrico su $K$.
\end{definition}
\begin{definition}{polinomio minimo}
    se $\alpha$ è algebrico, indichiamo con $\mu_{\alpha}(x)$ il generatore monico di $\ker(\varphi_\alpha)$. Esso coincide con l'unico polinomio $(i)$ monico, $(ii)$ irriducibile che $(iii)$ si annulla in $\alpha$.
    $\mu_{\alpha}(x)$ così definito è detto il polinomio minimo di $\alpha$ su $K$. Vale che $\deg \mu_\alpha(x) = \dim_K \frac{K[x]}{(\mu_{\alpha}(x))} = [K(\alpha) : K]$
\end{definition}
\begin{definition}{indipendenza algebrica}
    Un sottoinsieme $S \subset \Omega$ si dice algebricamente indipendente sul campo $K \subset \Omega$ se gli elementi di $S$ non soddisfano nessuna equazione polinomiale non banale a coefficienti in $K$. Dunque un singolo elemento $\alpha$ è algebricamente indipendente su $K$ sse è trascendente.
\end{definition}
\begin{definition}{estensione algebrica}
    $L/K$ si dice algebrica se ogni elemento di $L$ è algebrico su $K$.
\end{definition}
\begin{proposition}{estensione finita è algebrica}
    se $L/K$ è finita di grado $n$, dato $\alpha \in L$ gli elementi $1,\alpha,\dots,\alpha^n$ sono linearmente dipendenti, dunque esiste un polinomio di grado al più $n$ che si annulla in $\alpha$ (la combinazione lineare delle potenze di $\alpha$ fino a $n$ che dà come risultato $0$).
\end{proposition}

\subsection{Proprietà delle estensioni di campo}
\begin{minipage}{0.9\textwidth}
\begin{theorem}{torri di estensioni finite}
    dato il diagramma a lato, vale $L/K$ finita $\iff L/F$ e $F/K$ sono finite. Inoltre in questo caso $[L:K] = [L:F][F:K]$.
\end{theorem}
\end{minipage}\hfill
\begin{minipage}{0.1\textwidth}  
\begin{tikzcd}[every arrow/.append style={dash}]
&L\ar{d}\\
&F\ar{d}\\
&K
\end{tikzcd}  
\end{minipage}\hfill
\begin{proof}
    Se $L/K$ è finita, poiché $F \subseteq L$ anche $F/K$ è finita. Inoltre una qualsiasi $K$-base per $L/K$ è un $F$-insieme di generatori per $L/F$, quindi anche $L/F$ finita. Nell'altro verso, date $L/F$ e $F/K$ entrambe finite consideriamo due basi $\alpha_1,\dots, \alpha_n \in L$ e $\beta_1, \dots, \beta_m \in F$ tali che $L = F(\alpha_1,\dots, \alpha_n)$ e $F = K(\beta_1, \dots, \beta_m)$. Mostriamo che $\{\alpha_i \beta_j : 1 \le i \le n, 1 \le j \le m \}$ è una base di $L/K$.
    \begin{itemize}
        \item generano: sia $\gamma \in L$; per $\{\alpha_i\}$ base $\exists f_1,\dots,f_n \in F \ \gamma = \sum_{i=1}^n f_i \alpha_i$; analogamente per $\{\beta_i\}$ base $\forall f_i \ \exists k_{i,1},\dots,k_{i,m} \in K \ f_i =\sum_{j=1}^m k_{i,j} \beta_j$. Mettendo insieme le varie espressioni: $\gamma = \sum_{i=1}^n \alpha_i\sum_{j=1}^m k_{i,j} \beta_j = \sum_{i=1}^n\sum_{j=1}^m k_{i,j}\alpha_i\beta_j$ e quindi $\alpha_i\beta_j$ generano
        \item lineare indipendenza: se $0 = \sum_{i=1}^n\sum_{j=1}^m k_{i,j}\alpha_i\beta_j = \sum_{i=1}^n \alpha_i\sum_{j=1}^m k_{i,j} \beta_j$ si ha per lineare indipendenza degli $\alpha_i$ che $\forall i = 1,\dots, n \ \sum_{j=1}^m k_{i,j} \beta_j = 0$ e allora per lineare indipendenza dei $\beta_j \ \forall i=1,\dots,n \ \forall j = 1,\dots, m \ k_{i,j} = 0$.
    \end{itemize}
    Questo dimostra che l'estensione è finita e la moltiplicatività.
\end{proof}
\begin{minipage}{0.7\textwidth}
\begin{theorem}{shift di estensioni finite}
    dato il diagramma a lato, vale $L/K$ finita $\Rightarrow LF/F$ finita. Inoltre,  $[LF : F] \leq  [L:K]$.
\end{theorem}
\begin{proof}
    Se $\alpha_1,\dots, \alpha_n$ è una base per $L/K$, allora genera $LF/F$.
\end{proof}
\end{minipage}\hfill
\begin{minipage}{0.3\textwidth}  
\begin{tikzcd}[every arrow/.append style={dash}]
&&LF\ar{dl}\ar{dr}\\
&L \ar{dr} & & F \ar{dl}\\
&& K
\end{tikzcd}  
\end{minipage}\hfill
\begin{minipage}{0.7\textwidth}
\begin{theorem}{composto di estensioni finite}
    dato il diagramma a lato, vale $L/K$ finita  e $F/K$ finita $\Rightarrow LF/F$ finita. Inoltre, $[LF : K] \leq [L:K][L:F]$.
\end{theorem}
\begin{proof}
    Per shift, da $L/K$ finita, segue $LF/F$ finita e $[LF:F]\leq [L:K]$. Allora per torri da $LF/F$ e $F/K$ finite segue $LF/K$ finita e $[LF:K] = [LF:F][F:K] \leq [L:K][F:K]$.
\end{proof}
\end{minipage}\hfill
\begin{minipage}{0.3\textwidth}  
\begin{tikzcd}[every arrow/.append style={dash}]
&&LF\ar{dl}\ar{dd}\ar{dr}\\
&L \ar{dr} & & F \ar{dl}\\
&& K
\end{tikzcd}
\end{minipage}\hfill

\vspace{0.5cm}

\begin{proposition2}
    Un'estensione finitamente generata da elementi algebrici è algebrica.
\end{proposition2}
\begin{proof}
    Basta notare che gli elementi algebrici hanno grado finito sul campo base e quindi un'estensione generata da un numero finito di essi ha grado finito (al più il prodotto dei gradi). L'estensione è finita, dunque algebrica.
\end{proof}

\begin{minipage}{0.9\textwidth}  
\begin{theorem}{torri di estensioni algebriche}
    dato il diagramma a lato, vale $L/K$ algebrica $\iff L/F$ e $F/K$ sono algebriche.
\end{theorem}
\end{minipage}\hfill
\begin{minipage}{0.1\textwidth}  
\begin{tikzcd}[every arrow/.append style={dash}]
&L\ar{d}\\
&F\ar{d}\\
&K
\end{tikzcd}  
\end{minipage}\hfill
\begin{proof}
    Se $L/K$ è algebrica e $\alpha \in L$ allora $\alpha$ algebrico su $K$, quindi anche su $F$ perché almeno un polinomio in $F[x]$ (il polinomio minimo di $\alpha$ su $K$) si annulla in $\alpha$. Quindi $L/F$ algebrica. $F/K$ è algebrica poiché $F \subseteq L$ e $L/K$ è algebrica.
    
    Nel verso opposto, dato $\alpha \in L$ algebrico su $F$ sia $f(x) = \sum_{i=0}^n a_ix^i$ il suo polinomio minimo su $F$. Allora $\alpha$ è algebrico su $K(a_0,\dots, a_n) \subseteq F$, cioè $K(\alpha, a_0, \dots, a_n)/K(a_0, \dots, a_n)$ finita. Ma $K(a_0,\dots, a_n)$ è un'estensione di $K$ finitamente generata da elementi algebrici, quindi finita. Per torri allora anche $K(\alpha, a_0, \dots, a_n)/K$ è finita, quindi algebrica. In particolare $\alpha$ è algebrico su $K$.
\end{proof}
\begin{minipage}{0.7\textwidth}
\begin{theorem}{shift di estensioni algebriche}
    dato il diagramma a lato, vale $L/K$ algebrica $\Rightarrow LF/F$ algebrica.
\end{theorem}
\begin{proof}
    Dato $\ell \in L$, se $f(x) \in K[x]$ è il polinomio minimo di $\ell$ su $K$, esso è un polinomio di $F[x]$ che si annulla in $\ell$, quindi $\ell$ è algebrico su $F$. Ogni elemento $\alpha \in F(L)$ è rapporto di funzioni razionali di finiti elementi $\ell_1, \dots, \ell_n \in L$ a coefficienti in $F$, dunque $\alpha \in F(\ell_1, \dots, \ell_n)$, che è algebrica poiché gli $\ell_i$ sono algebrici, dunque $\alpha$ è algebrico su $F$ e $F(L) = LF$ è un'estensione algebrica di $F$.
\end{proof}
\end{minipage}\hfill
\begin{minipage}{0.3\textwidth}  
\begin{tikzcd}[every arrow/.append style={dash}]
&&LF\ar{dl}\ar{dr}\\
&L \ar{dr} & & F \ar{dl}\\
&& K
\end{tikzcd}  
\end{minipage}\hfill

\begin{minipage}{0.7\textwidth}
\begin{theorem}{composto di estensioni algebriche}
    dato il diagramma a lato, vale $L/K$ algebrica  e $F/K$ algebrica $\Rightarrow LF/K$ algebrica.
\end{theorem}
\begin{proof}
    Per shift da $L/K$ algebrica segue $LF/F$ algebrica. Per torri da $LF/F$ e $F/K$ algebriche segue $LF/K$ algebrica.
\end{proof}
\end{minipage}\hfill
\begin{minipage}{0.3\textwidth}  
\begin{tikzcd}[every arrow/.append style={dash}]
&&LF\ar{dl}\ar{dd}\ar{dr}\\
&L \ar{dr} & & F \ar{dl}\\
&& K
\end{tikzcd}  
\end{minipage}\hfill

\begin{definition}{campo algebricamente chiuso}
    $K$ si dice algebricamente chiuso se ogni polinomio non costante in $K[x]$ ha una radice in $K$. Si verifica $K$ algebricamente chiuso sse gli unici irriducibili di $K[x]$ sono i polinomi di grado 1 e sse ogni polinomio in $K[x]$ si spezza nel prodotto di fattori di grado 1.
\end{definition}
\begin{definition}{chiusura algebrica}
    Un campo algebricamente chiuso $\overline K$ si dice una chiusura algebrica di $K$ se $K \subseteq \overline K$ e $\overline K/K$ è algebrica.
\end{definition}
\begin{theorem}{esistenza e unicità della chiusura algebrica}
    Per ogni campo $K$ esiste una chiusura algebrica, unica a meno di isomorfismo (cioè date due chiusure algebriche esiste tra esse un isomorfismo che ristretto a $K$ sia l'identità).
\end{theorem}
\begin{proof}
    Ad algebra 2.
\end{proof}
\begin{theorem}{algebrici su un campo}
     Gli elementi algebrici su un campo $K$ formano a loro volta un campo.
\end{theorem}
\begin{proof}
    Sia $K' = \{ \alpha \in \overline K : \alpha \text{ algebrico su } K\}$. Presi $\alpha, \beta \in K'$, $K(\alpha, \beta)$ è finita, $\alpha + \beta, \alpha\beta, \frac{\alpha}{\beta}$ appartengono a $K(\alpha, \beta)$, quindi hanno grado finito su $K$, quindi sono algebrici e appartengono a $K'$, che quindi è un campo.
\end{proof}
\begin{theorem}{chiusura algebrica di $\Q$}
    $\overline{\Q} = \{ \alpha \in \C : \alpha \text{ algebrico su } \Q\}$
\end{theorem}
\begin{proof}
    $\overline{\Q}$ è un campo per il teorema precedente, $\overline{\Q}/\Q$ algebrica segue dalla definizione. Mostriamo che $\overline \Q$ è algebricamente chiuso. Siano $f(x) \in \overline{\Q}[x]$ e $\alpha$ una sua radice nella chiusura algebrica di $\overline{\Q}$. Per definzione $\overline{\Q}(\alpha)/\overline{\Q}$ è algebrica, quindi per torri anche $\overline{\Q}(\alpha) / \Q$ è algebrica, cioè $\alpha$ algebrico su $\Q$ e $\alpha \in \overline{\Q}$.
\end{proof}
\begin{definition}{campo di spezzamento}
    sia $\mathscr{F} = \{f_i : i \in I\}$ una famiglia di polinomi in $K[x]$. Il campo di spezzamento di $\mathscr{F}$ su $K$ è il minimo sottocampo di $\overline{K}$ che contiene tutte le radici di tutti i polinomi in $\mathscr{F}$.
\end{definition}

\subsection{Criterio della derivata e campi finiti}

\begin{proposition}{criterio della derivata}
    $f(x) \in K[x]$ ha radici multiple in $\overline{K}$ $\Leftrightarrow$ $(f(x),f'(x)) \neq 1$.
\end{proposition}
\begin{proof}
    ($\Leftarrow$) Il caso $f = 0$ non ha bisogno di ulteriore analisi. Se $f \neq 0$, $(f(x), f'(x))$ ha grado positivo, sia allora $\alpha \in \overline{K}$ una sua radice. Scriviamo $f(x) = (x - \alpha)g(x) \in \overline{K}[x]$ e deriviamo $f$ con la regola di Leibniz, valutando $f'$ in $\alpha$ si ha $0 = f'(\alpha) = g(\alpha) + (\alpha - \alpha)g'(x)$, da cui $g(\alpha) = 0$, cioè $g(x) = (x - \alpha)h(x)$ e quindi $f(x) = (x - \alpha)^2 h(x)$.
    
    ($\Rightarrow$) Sia $f(x) = (x - \alpha)^2g(x) \in \overline{K}[x]$, allora $f'(x) = 2(x - \alpha)g(x) + (x - \alpha)^2g'(x)$, in particolare $(x - \alpha) \mid f'(x)$. Poiché $f,f' \in K[x]$ e $f(\alpha) = f'(\alpha) = 0$, detto $\mu_\alpha(x) \in K[x]$ il polinomio minimo di $\alpha$ su $K$ si ha $\mu_\alpha(x) \mid f(x), f'(x)$ e dunque $\mu_\alpha \mid (f(x), f'(x))$ che quindi è nullo o ha grado positivo: in particolare $(f(x), f'(x)) \neq 1$.
\end{proof}

\begin{corollary}{derivata di irriducibili}
    Sia $f(x) \in K[x]$ irriducibile.  $f$ ha radici multiple $\Leftrightarrow$ $f'(x) = 0$.
\end{corollary}
\begin{proof}
    Notiamo $(f(x),f'(x)) \mid f(x)$ e $f(x)$ irriducibile, dunque $(f(x),f'(x)) \in \{1,f(x)\}$. Per il criterio della derivata allora $f$ ha radici multiple $\Leftrightarrow$ $(f(x),f'(x)) = f(x)$. Ma $\deg f'(x) \leq \deg f(x)$ e $f(x) \mid f'(x)$ implicano necessariamente $f'(x)=0$.
\end{proof}
\begin{proposition}{omomorfismo di Frobenius}
    Sia $K$ un campo a caratteristica $p$. Allora la mappa $\Phi: K \rightarrow K$ tale che $\Phi(a \mapsto a^p)$ è un omomorfismo iniettivo (detto \textit{di Frobenius}). Se $K$ è finito quindi $\Phi$ è un automorfismo.
\end{proposition}
\begin{proof}
    Notiamo che $\Phi(0)=0$, $\Phi(1)=1$ e $\Phi(a)\Phi(b) = \Phi(ab)$ perché un campo è commutativo. Manca solo $\Phi(a) + \Phi(b) = \Phi(a+b)$, ma sviluppando il binomio di Newton, $(a+b)^p = \sum_{k=0}^p \binom{p}{k}a^kb^{p-k} = a^p+b^p$, infatti $k\neq0,p \Rightarrow p \mid \binom{p}{k}$. (Quest'ultima identità viene talvolta chiamata ``binomio ingenuo''.) Chiaramente $\ker(\Phi) = \{0\}$, quindi l'omomorfismo è iniettivo. Se $K$ è finito, tanto basta a dire che $\Phi$ è un automorfismo.
\end{proof}
\begin{definition}{campo perfetto}
    $K$ si dice perfetto se ogni polinomio irriducibile $f(x) \in K[x]$ ha radici tutte distinte in $\overline{K}$.
\end{definition}
\begin{proposition}{due classi di campi perfetti}
    Se $K$ è un campo finito o di caratteristica 0, allora è perfetto.
\end{proposition}
\begin{proof}
    Sia $f\in K[x]$ irriducibile, $f(x) = \sum_{i=0}^n a_ix^i$, $f'(x) = \sum_{i = 1}^{n}{i a_i x^{i-1}}$. Sappiamo che $f$ ha radici multiple $\iff f'(x) = 0$, cioè se $\forall i = 1 \dots n$ vale  $i a_i = 0$.
    
    Se $K$ è a caratteristica 0, allora $(\forall i = 1 \dots n \ i a_i = 0) \implies f(x) = a_0$ costante. In questo caso $f$ non ha radici oppure è il polinomio nullo.
    
    Se $K$ è finito e a caratteristica $p$, allora $f'(x) = 0$ se e solo se per tutti gli indici $i = 1 \dots n$ non multipli di $p$ vale $a_i = 0$. Ma allora $f(x) = \sum_{i=0}^m a_{pi}x^{pi} = g(x^p)$ con $g(t) = \sum_{i=0}^{m} a_{pi}t^i \in K[x]$. Per l'omomorfismo di Frobenius $f(x) = g(x^p) = g(x)^p$: assurdo poiché $f$ è irriducibile.
\end{proof}

\begin{theorem}{(*) $K$ perfetto sse l'omomorfismo di Frobenius è surgettivo}
    Sia $K$ campo con $\char K = p$, allora $F$ è perfetto $\iff$ $\Phi: x \mapsto x^p$ è surgettivo.
\end{theorem}
\begin{proof}
    $(\impliedby)$ Bisogna mostrare che ogni $f \in K[x]$ con $f' = 0$ è riducibile. Un tale polinomio si scrive nella forma $f(x) = \sum_{k=0}^{n} a_k x^{pk}$. Usiamo l'ipotesi e siano $b_1, \dots, b_n$ tali che $b_k^p = a_k$. Allora $f(x) = (\sum_{k=0}^{n} b_k x^{k})^p$, dunque $f$ riducibile.
    $(\implies)$ Dato un generico $a \in K$ consideriamo il polinomio $f(x) = x^p - a \in K[x]$. Data $b \in \overline{K}$ con $b^p = a$ per il binomio ingenuo si ha $f(x) = (x - b)^p$. Poiché $K$ è perfetto, il polinomio minimo $\mu_b(x) \in K[x]$ di $b$ su $K$ ha tutte le radici distinte. Ma allora $\mu_b(x) ^ p \mid f(x)$, quindi necessariamente $\mu_b(x)$ ha grado 1, cioè $b \in K$. Dunque $\Phi : x \mapsto x^p$ è surgettivo.
\end{proof}

\begin{theorem}{esistenza e unicità di $\F_{p^n}$}
    Per ogni primo $p$ e intero $n \ge 1$, esiste un unico campo $F$ con $p^n$ elementi all'interno di una fissata chiusura algebrica di $\F_p$.
\end{theorem}
\begin{proof}
    Se $F$ esiste, allora $\#F^\times = p^n - 1$, dunque per Lagrange gli elementi di $F^\times$ sono radici in $\overline{\F_p}$ di $x^{p^n - 1} - 1$ e gli elementi di $F$ sono radici di $f(x) = x^{p^n} - x$. Per il criterio della derivata queste radici sono tutte distinte, infatti $f'(x) = p^n x^{p^n - 1} - 1 = -1$ in caratteristica $p$. Ma allora $F = \{ \alpha \in \overline{\F_p} : \alpha^{p^n} - \alpha = 0 \}$ è l'unico candidato $p^n$-campo (unicità). Si verifica che $F$ così definito contiene 0, 1, è chiuso per somma, prodotto, opposti e inversi (esercizio), dunque è un campo (esistenza).
\end{proof}

Dunque $\F_{p^n}$ è il campo di spezzamento su $\F_p$ di $x^{p^n} - x = 0$ e $\F_{p^n}^\times$ sono tutte e sole le $p^n-1$-esime radici dell'unità.

\begin{theorem}{Sottogruppi moltiplicativi di un campo}
    Sia $K$ campo e $G < K^\times$ un sottogruppo moltiplicativo. Se $G$ è finito, allora è ciclico.
\end{theorem}
\begin{proof}
    Sia $\#G = n$, $\forall g \in G \ g^n = 1$. Definiamo $f_d(x) = x^d - 1 \in K[x]$. Per Ruffini $f_d$ ha al più $d$ radici in $G$. Sia $G_d = \{ \alpha \in G : \alpha^d - 1 = 0 \}$, vale $\#G_d \le d$. Sia $k_d = \#\{ \alpha \in G : \ord g = d \}$. Se $d \nmid n$, allora $k_d = 0$, se invece $d \mid n$ e $k_d > 0$, dato $g \in G \ \ord g = d$ si ha $\grp{g} \subseteq G_d$, ma allora per cardinalità $\grp{g} = G_d$, dunque $k_d = \varphi(d)$. Si ha
    \[
        n = \#G = \sum_{d | n} k_d \le \sum_{d | n} \varphi(d) = n,
    \]
    che quindi sono tutte uguaglianze e $k_n = \varphi(n) \ge 1$, cioè $\exists g \in G \ \ord g = n$, vale a dire $G$ ciclico.
\end{proof}
\begin{corollary2}
    $\F_{p^n}^\times$ è ciclico. Inoltre $\F_{p^n} = \F_p(\alpha)$ per qualche $\alpha \in \overline{\F_p}$.
\end{corollary2}
\begin{proof}
    $\Fpn^\times = \grp{\alpha} \land \Fp(\alpha) \subseteq \Fpn \implies \Fpn = \Fp(\alpha)$.
\end{proof}
\begin{observation}{non vale l'implicazione inversa}
    $\Fpn = \Fp(\alpha) \nRightarrow \grp{\alpha} = \Fpn^\times$.
\end{observation}
\begin{proof}
    Funziona più o meno qualunque controesempio con $p^n - 1$ non primo. Consideriamo $\F_9 \cong \frac{\F_3[x]}{(x^2 + 1)} = \{ 0, 1, 2, x, x+1, x+2, 2x, 2x+1, 2x+2 \}$: ogni elemento $\alpha \in \F_9 \setminus \F_3$ ha necessariamente polinomio minimo di grado 2 su $\F_3$, dunque $\F_3(\alpha) = \F_9$. Tuttavia $\grp{x} = \{x, -1, x, 1\} \neq \F_9^\times$.
\end{proof}
\begin{corollary}{polinomi irriducibili su $\Fp$}
    Per ogni $p$ primo, $n \ge 1$ naturale esistono in $\Fp[x]$ polinomi irriducibili di grado $n$.
\end{corollary}
\begin{proof}
    Sia $\Fpn = \Fp(\alpha)$ per qualche $\alpha \in \overline{\Fp}$, allora $[\Fp(\alpha) : \Fp] = n = \deg \mu_\alpha(x)$ irriducibile.
\end{proof}
\begin{proposition}{inclusioni tra sottocampi}
    $\Fpm \subseteq \Fpn \iff m | n$.
\end{proposition}
\begin{proof}
    ($\Rightarrow$) Per torri $n = [\Fpn : \Fp] = [\Fpn : \Fpm][\Fpm : \Fp] = [\Fpn : \Fpm]m$. ($\Leftarrow$) Ricordiamo il prodotto notevole $(x^m)^\lambda - 1 = (x^m - 1)((x^m)^{\lambda - 1} + \dots + 1)$, in particolare per $x = p$ si ha $p^n - 1 = a(p^m - 1)$ per un opportuno $a$ intero. Allora $\forall \alpha \in \Fpm^\times \ \alpha^{p^m - 1} = 1$, ma allora anche $\alpha^{p^n - 1} = (\alpha^{p^m - 1}) ^ a = 1^a = 1$, cioè $\alpha \in \Fpn^\times$.
\end{proof}

% potenziale TODO: includere il reticolo di sottocampi (per esempio) di \F_{p^12}

\begin{observation}{radici di irriducibili}
    Sia $f(x) \in \Fp[x]$ irriducibile di grado $n$ e siano $\{\alpha_1, \dots, \alpha_n\} \subseteq \overline{\Fp}$ le sue radici (che sappiamo essere distinte). $f$ è il polinomio minimo degli $\alpha_i$ su $\Fp$, quindi $\Fpn = \Fp(\alpha_1) = \dots = \Fp(\alpha_n) \cong \frac{\Fp[x]}{f(x)}$.
    Si noti che l'estensione semplice con una radice di $f$ ha automaticamente incluso tutte le radici di tutti gli irriducibili di grado (esattamente) $n$.
\end{observation}

\begin{proposition}{Campo di spezzamento su $\Fq$ $(q = p^n)$}
    Sia $f \in \Fq[x] \ f(x) = f_1^{e_1}(x) \dots f_r^{e_r}(x)$ con gli $f_i(x)$ irriducibili e $\deg f_i = d_i$. Allora, detto $d = [d_1, \dots, d_r]$ l'mcm dei gradi, il campo di spezzamento di $f$ su $\Fq$ è $\F_{q^d}$
\end{proposition}
\begin{proof}
    Il campo di spezzamento (in seguito, cds) di $f$ su $\Fq$ è il composto dei cds degli $f_i$ su $\Fq$. Il cds di $f_i$ su $\Fq$ è $\F_{q^{d_i}}$. Sia allora $\F_{q^c}$ il cds $f$ su $\Fq$. Vale $\F_{q^{d_i}} \subset \F_{q^c}$ se e solo se $d_i | c$, dunque $d = [d_1, \dots, d_r] \mid c$. $\F_{q^d}$ è il più piccolo campo che contiene tutti gli $\F_{q^{d_i}}$, vale a dire il cds.
\end{proof}

\begin{theorem}{Campo di spezzamento su $\Fp$ di $x^n - 1$}
    Sia $n = p^a m$ con $(m, p) = 1$. Allora il campo di spezzamento di $x^n - 1$ su $\Fp$ coincide con il cds di $x^m - 1$ su $\Fp$ ed è uguale a $\F_{p^d}$ con $d = \ord_{\Zn^\times}p$.
\end{theorem}
\begin{proof}
    $x^{p^a m} - 1 = (x^m - 1)^{p^a}$ per il binomio ingenuo, dunque moltiplicare $m$ per una potenza di $p$ cambia solo la molteplicità delle radici di $x^m - 1$. Sia $G_n = \{ \alpha \in \overline{\Fp} : \alpha^n = 1 \} = G_m = \{ \alpha \in \overline{\Fp} : \alpha^m = 1 \}$. $\# G_m = m$ per il criterio della derivata, infatti $(x^m - 1, mx^{m-1}) = 1$. Sappiamo che il campo di spezzamento $\Fp(G_m)$ è un'estensione finita, quindi uguale a $\F_{p^d}$ per qualche $d$: ci chiediamo chi è $d$.

    \textbf{Lemma:} $G_m = \{ \alpha \in \overline{\Fp} : \alpha^m = 1 \} \le \F_{p^d}^\times = \{ \alpha \in \overline{\Fp} : \alpha^{p^d - 1} = 1 \}$ se e solo se $m \mid p^d - 1$.
    \begin{proof}
        $(\implies)$ Per Lagrange $m = \#G_m | \#\F_{p^d}^\times = p^d - 1$.
        $(\impliedby)$ Sia $p^d - 1 = m l$. Allora per ogni $\alpha \in G_m$ si ha $\alpha^{p^d - 1} = (\alpha^m)^l = 1^l = 1$, cioè $\alpha \in \F_{p^d}^\times$.
    \end{proof}
    
    Per il lemma, $d = \min \{ k : m | p^k - 1 \} = \min \{ k : p^k \equiv 1 \pmod{m} \} = \ord_{\Zn^\times}p$.
\end{proof}

\begin{theorem}{Automorfismi di $\Fpn$}
    L'omomorfismo di Frobenius genera tutti gli automorfismi di $\Fpn$, in particolare $\Aut(\Fpn) = \grp{\Phi: x \mapsto x^p} \cong \Zn$.
\end{theorem}
\begin{proof}
    Ogni automorfismo $\varphi \in \Aut(\Fpn)$ manda $1$ in sé, dunque fissa puntualmente $\Fp$. Sia $\alpha \in \Fpn$ tale che $\Fpn = \Fp(\alpha)$ e $\mu_\alpha \in \Fp[x]$ il suo polinomio minimo, che ricordiamo ha grado $n$. Ogni $\varphi \in \Aut(\Fpn)$ è univocamente determinata dall'immagine di $\alpha$. Poiché $\mu_\alpha(\alpha) = 0$ e $\varphi(\mu_\alpha(x)) = \mu_\alpha(\varphi(x))$, $\varphi(\alpha)$ è una radice di $\mu_\alpha$, dunque $\#\Aut(\Fpn) \le n$. Consideriamo $\{ \alpha, \alpha^p, \dots, \alpha^{p^n - 1} \}$ le immagini di $\alpha$ tramite $\Phi^0, \Phi^1, \dots, \Phi^{n - 1}$: se queste sono tutte distinte i $\Phi^k$ sono $n$ automorfismi distinti di $\Fpn$, dunque tutti gli automorfismi di $\Fpn$. Se per assurdo fosse $\alpha^{p^i} = \alpha^{p^j}$ con $0 \le i < j < n$, allora $\alpha^{p^j - p^i} - 1 = 0$, da cui $(\alpha^{p^{j - i} - 1} - 1)^{p^i} = 0$, quindi $\alpha$ sarebbe una radice $p^{j-i} - 1$-esima dell'unità e dunque $\Fp(\alpha) \subseteq \Fp^{j - i}$: assurdo poiché $j - i < n$ e $\alpha$ è un generatore di $\Fpn^\times$. Quindi tutte le immagini di $\alpha$ tramite $\Phi^0, \dots, \Phi^{n-1}$ sono tutte distinte e questi sono tutti e soli gli automorfismi di $\Fpn$.
\end{proof}

\subsection{Estensioni normali}

Con ``immersione'' intenderemo sempre un omomorfismo di anelli (quindi anche di campi) iniettivo.
    
Notiamo primariamente che un omomorfismo di campi diverso dall'omomorfismo banale è necessariamente iniettivo (basta notare che gli unici ideali, quindi possibili nuclei dell'omomorfismo, $(0)$ e il campo stesso). Da ora fino alla fine delle dispense escluderemo l'omomorfismo banale da tutti i ragionamenti, in modo che un omomorfismo di campi sia sempre un'immersione.

\begin{proposition}{immersioni con estensioni semplici}
    Dato $\alpha \in \overline{K}$, le immersioni $\varphi: K(\alpha) \rightarrow \overline{K}$ tali che $\varphi|_K = id$ sono tante quante le radici distinte di $\mu_{\alpha}(x)$ in $\overline{K}$.
\end{proposition}    
\begin{proof}
    Per il $1^{\circ}$ teorema di omomorfismo costruire una tale $\varphi$ equivale a costruire una $\tilde \varphi: K[x] \rightarrow \overline{K}$ per cui $\tilde \varphi |_K = id$ e $(\mu_{\alpha}(x)) \subseteq \ker(\tilde \varphi)$. Notiamo ora che se $x \overset{\tilde \varphi}{\mapsto} \beta$, allora $\tilde \varphi$ è l'omomorfismo di valutazione in $\beta$. Ma $\varphi(\mu_{\alpha}(x)) = \mu_{\alpha}(\varphi(x))$ e $\mu_{\alpha}(x) \in  \ker(\tilde \varphi) \iff \mu_{\alpha}(\beta) = 0$. Quindi gli omomorfismi che vanno bene sono tutti e soli quelli per cui $\beta$ è una radice di $\mu_{\alpha}(x)$.

    Si noti che per l'iniettività di $\varphi$ si ha necessariamente $\ker(\tilde \varphi) = (\mu_\alpha(x))$.
\end{proof}

\begin{proposition}{estensione a un'estensione semplice}
    Sia $K$ un campo perfetto, $\alpha \in \overline{K}$, $[K(\alpha):K] = n$. Allora ogni immersione $\varphi: K \hookrightarrow \overline{K}$ ammette $n$ estensioni distinte $\varphi_1,\dots,\varphi_n: K(\alpha) \hookrightarrow \overline{K}$ tali che $\varphi_i|_K = \varphi$.
\end{proposition}
\begin{proof}
    Già visto nella Proposizione 1 nel caso in cui $\varphi = id$: occorre solo generalizzare.
    Per il primo teorema di omomorfismo costruire tale immersione equivale a costruire una $\tilde \varphi: K[x] \rightarrow \overline{K}$ tale che $(\mu_{\alpha}(x)) \subseteq \ker(\tilde \varphi)$. Come sopra se $\tilde \varphi(x \mapsto \beta)$ allora $\tilde \varphi$ è l'omomorfismo di valutazione in $\beta$ e quindi $\mu_{\alpha}(x) \in \ker(\tilde \varphi) \iff \varphi\mu_{\alpha}(\beta) = 0$, dove $\varphi\mu_{\alpha}$ è il polinomio i cui coefficienti sono le immagini secondo $\varphi$ dei coefficienti di $\mu_{\alpha}(x)$. Usando che $\varphi(K) \cong K \Rightarrow \varphi K[x] \cong K[x] \Rightarrow \varphi$ preserva l'irriducibilità $\varphi\mu_{\alpha}(x)$ è irriducibile. Allora, essendo $K$ perfetto, ha $n$ radici distinte. Le possibili scelte per $\beta$ sono quindi tutte e sole le $n$ radici di $\varphi\mu_{\alpha}(x)$.
\end{proof}
\begin{proposition}{estensione a un'estensione finita}
    Sia $E/K$ finita, con $[E:K] = n$. Allora ogni immersione $\varphi: K \hookrightarrow \overline{K}$ ammette $n$ estensioni distinte a $E$.
\end{proposition}
\begin{proof}
    Notiamo intanto che $E/K$ finita $\Rightarrow E/K$ algebrica $\Rightarrow E \subseteq \overline{K}$ e quindi sarà tutto ben definito. Procediamo per induzione su $n$. Il passo base $n=1$ è ovvio.

    Consideriamo ora $\alpha \in E \setminus K$. Allora $[K(\alpha) : K] = m \geq 1$, $[E : K(\alpha)] = d = \frac{n}{m}$ (per torri). Se $d=1$ la tesi segue dalla proposizione precedente. Se invece $n >d>1$ usiamo la proposizione precedente per costruire $m$ estensioni di $\varphi$, $\varphi_1,\dots, \varphi_m$ a $K(\alpha)$. Per ipotesi induttiva, ciascun $\varphi_i:K(\alpha) \rightarrow \overline{K}$ ammette esattamente $d$ estensioni $\varphi_{i,1},\dots,\varphi_{i,d}$ a $E$ tali che $\varphi_{i,j}|_K = \varphi_i|_K = \varphi|_K$. Allora le $\varphi_{i,j}$ così costruite sono le estensioni cercate e sono tutte distinte.
    
    Dimostriamo che non ve ne sono altre. Sia $\psi: E \rightarrow \overline{K}$ un'estensione di $\varphi$. Allora $\psi|_{K(\alpha)}: K(\alpha) \rightarrow \overline{K}$ e quindi per la proposizione precedente, essendo le $\varphi_i$ le uniche estensioni deve valere $\psi|_{K(\alpha)} = \varphi_i$ per un qualche $i$. Ma allora $\psi$ estende un $\varphi_i$ e per ipotesi induttiva, essendo le $\varphi_{i,j}$ le uniche possibili estensioni, $\psi = \varphi_{i,j}$ per un qualche $j$. Questo conclude la dimostrazione. 
\end{proof}
\begin{proposition}{generalizzazione}
    In generale, data $E/K$ algebrica e $\psi : K \to \overline{K}$ esiste almeno un'estensione $\varphi : E \to \overline{K}$ tale che $\varphi|_K = \psi$. (La dimostrazione, omessa nel corso, è un'applicazione di Zorn.)
\end{proposition}
\begin{definition}{coniugati}
    Si dicono coniugati su $K$ di $\alpha \in \overline{K}$ le radici del polinomio minimo $\mu_{\alpha}(x) \in K[x]$ di $\alpha$ su $K$.
\end{definition}
\begin{proposition}{immersioni e coniugati}
    Data $E/K$ algebrica e $\alpha \in E$, le immersioni $\varphi: E \rightarrow \overline{K}$ tali che $\varphi|_K = id$ mandano necessariamente $\alpha$ in un suo coniugato su $K$.
\end{proposition}
\begin{proof}
 Basta osservare che, poiché $\varphi|_K = id$, $\varphi(\mu_{\alpha}(x)) =\mu_{\alpha}(\varphi(x)) $ e quindi, sostituendo $x = \alpha$, $0 = \mu_{\alpha}(\varphi(\alpha))$.
\end{proof}
\begin{definition}{estensione normale}
    $F/K$ algebrica si dice normale se $\forall \varphi: F \rightarrow \overline{K}$ tale che $\varphi|_K = id$ si ha $\varphi(F)=F$.
\end{definition}
\begin{proposition2}
    Le estensioni di grado $2$ sono normali.
\end{proposition2}
\begin{proof}
    Assumiamo per ora $\char K \neq 2$. Sia $F/K$ di grado 2 e $\alpha \in F \setminus K$. Allora $[K(\alpha):K] = 2$ e $F = K(\alpha)$. Sia quindi $p(x) = x^2 + ax+b$ il polinomio minimo di $\alpha$ su $K$. Allora i coniugati di $\alpha$ sono $\frac{-a\pm\sqrt{\Delta}}{2}$ e quindi $F = K(\alpha) = K(\sqrt{\Delta})$. I coniugati di $\sqrt{\Delta}$ sono $\pm\sqrt{\Delta}$, quindi da $\varphi|_K = id$ si ha $\varphi(K(\sqrt{\Delta})) = K(\pm\sqrt{\Delta}) = K(\sqrt{\Delta})$ come voluto.

    Con la caratterizzazione delle estensioni normali che segue possiamo abbandonare l'ipotesi $\char K \neq 2$, infatti il termine noto di un polinomio (in questo caso di secondo grado) è prodotto delle radici.
\end{proof}
\begin{theorem}{caratterizzazione delle estensioni normali}
    Sia $F/K$ algebrica. Sono equivalenti: 
    \begin{itemize}
        \item[(i)] $F/K$ normale;
        \item[(ii)] $\forall f \in K[x]$ irriducibile se $f$ ha una radice in $F$ allora ha tutte le radici in $F$;
        \item[(iii)] $F$ è campo di spezzamento di una famiglia di polinomi.
    \end{itemize}
\end{theorem}
\begin{proof}
    $(i) \Rightarrow (ii)$ Sia $f \in K[x]$ irriducibile, $\alpha$ una radice di $f$. Allora per irriducibilità $f(x) = u\mu_{\alpha}(x)$ con $u \in K^\times$. Quindi senza perdita di generalità supponiamo $f(x) = \mu_{\alpha}(x)$. Siano $\alpha_1, \dots, \alpha_n$ le radici di $f$. Consideriamo le $n$ immersioni $\varphi_i: K(\alpha) \rightarrow F$ tali che $\varphi_i(\alpha) = \alpha_i$ e $\varphi_i|_K = id$ (esistono per le proposizioni precedenti). Dai fatti sopra dimostrati sappiamo che ciascuno di questi $\varphi_i$ si estende a $F$ (in realtà lo abbiamo dimostrato per estensioni finite); chiamiamo $\tilde \varphi_i$ l'estensione. Da $F/K$ normale sappiamo $\tilde \varphi_i (F) = F$ $\forall i = 1,\dots,n$ e quindi $\tilde \varphi_i(\alpha) = \varphi_i(\alpha) = \alpha_i \in F$.
    
    $(ii) \Rightarrow (iii)$ Consideriamo come famiglia di polinomi l'insieme dei polinomi minimi di tutti gli $\alpha \in F$. Sia $F_0$ il suo campo di spezzamento. Chiaramente $F \subseteq F_0$. Ma per l'ipotesi, $(ii)$ poiché per costruzione ciascuno di questi polinomi ha almeno una radice in $F$, ciascuno di questi polinomi si fattorizza completamente in $F$, e quindi $F_0 \subseteq F$.
    
    $(iii) \Rightarrow (i)$ Sia $F$ campo di spezzamento di $\mathscr{F} = \{f_i(x) : i \in I\}$ con $f_i \in K[x]$ e  $\Lambda_i$ l'insieme contentente tutte le radici di $f_i$. Allora $F = K\big( \bigcup_{i \in I} \Lambda_i \big)$ e inoltre $\forall \varphi: F \rightarrow \overline{K}$ poiché le immersioni mandano coniugati in coniugati si ha $\forall i \in I \ \varphi(K(\Lambda_i)) = K(\Lambda_i)$ ($\varphi$ agisce sulle radici permutandole, perché è iniettiva e fissa le radici di $f_i$) e quindi $\varphi(F) = \varphi(K\big( \bigcup_{i \in I} \Lambda_i \big))= K\big( \bigcup_{i \in I} \Lambda_i \big) = F$.
\end{proof}
\begin{minipage}{0.9\textwidth}  
\begin{theorem}{torri di estensioni normali}
    dato il diagramma a lato, vale $L/K$ normale $\Rightarrow L/F$ normale.
\end{theorem}
\begin{proof}
    L'algebricità di $L/F$ segue dalle torri di estensioni algebriche. Sia $\varphi: L \rightarrow \overline{K}$ tale che $\varphi|_F = id$. Allora si ha $\varphi|_K = id$ perché $K \subseteq F$ e quindi per normalità di $L/K$ si ha $\varphi(L) = L$, come voluto.
\end{proof}
\end{minipage}\hfill
\begin{minipage}{0.1\textwidth}  
\begin{tikzcd}[every arrow/.append style={dash}]
&L\ar{d}\\
&F\ar{d}\\
&K
\end{tikzcd}  
\end{minipage}\hfill
\begin{minipage}{0.7\textwidth}
\begin{theorem}{shift di estensioni normali}
    dato il diagramma a lato, vale $L/K$ normale e $F/K$ algebrica $\Rightarrow LF/F$ normale.
\end{theorem}
\begin{proof}
    L'algebricità di $F/K$ serve per la buona definizione. L'algebricità di $LF/F$ segue dallo shift di estensioni algebriche con $L/K$. Per la normalità, consideriamo $\varphi: LF \rightarrow \overline{K}$ tale che $\varphi|_F = id$. Allora si ha che $\varphi|_K =  (\varphi|_F)|_K = id$. Consideriamo $\varphi|_L : L \rightarrow \overline{K}$. Vale $(\varphi|_L)|_K = id$ per quanto detto prima; allora per normalità di $L/K$, $\varphi(L) = \varphi|_L(L) = L$. Quindi, poiché $\varphi(F)= F$, $\varphi(LF) = LF$.
\end{proof}
\end{minipage}\hfill
\begin{minipage}{0.3\textwidth}  
\begin{tikzcd}[every arrow/.append style={dash}]
&&LF\ar{dl}\ar{dr}\\
&L \ar{dr} & & F \ar{dl}\\
&& K
\end{tikzcd}  
\end{minipage}\hfill
\vspace{0.5cm}
\begin{minipage}{0.7\textwidth}
\begin{theorem}{composto di estensioni normali}
    dato il diagramma a lato, vale $L/K$ normale e $F/K$ normale $\Rightarrow LF/K$ normale.
\end{theorem}
\begin{proof}
    L'algebricità segue dal composto di estensioni algebriche.  Per la normalità, consideriamo come prima $\varphi: LF \rightarrow \overline{K}$. Allora per normalità di $L/K$, $\varphi(L) = \varphi|_L(L) = L$. Analogamente $\varphi(F) = F$ e quindi $\varphi(LF) = LF$.
\end{proof}
\end{minipage}\hfill
\begin{minipage}{0.3\textwidth}  
\begin{tikzcd}[every arrow/.append style={dash}]
&&LF\ar{dl}\ar{dd}\ar{dr}\\
&L \ar{dr} & & F \ar{dl}\\
&& K
\end{tikzcd}  
\end{minipage}\hfill
\begin{minipage}{0.65\textwidth}
\begin{theorem}{implicazione di normalità}
    dato il diagramma a lato, vale $L/K$ normale e $F/K$ normale $\Rightarrow (L\cap F)/K$ normale.
\end{theorem}
\begin{proof}
    L'algebricità è ovvia ($K \subseteq L \cap F \subseteq L$ e $L/K$ algebrica $\Rightarrow (L\cap F)/K$ algebrica). Per la normalità, consideriamo $\varphi: L\cap F\rightarrow \overline{K}$ tale che $\varphi|_K = id$. Poiché $LF/(L\cap F)$ è algebrica, $\varphi$ può essere estesa a $\tilde \varphi : LF \rightarrow \overline{K}$. Allora, poiché $L/K$ è normale, si ha $\tilde \varphi(L)  = L$ e analogamente $\tilde \varphi(F) = F$. Quindi $\tilde \varphi(L\cap F)  = \varphi(L\cap F) = L\cap F$.
\end{proof}
\end{minipage}\hfill
\begin{minipage}{0.35\textwidth}  
\begin{tikzcd}[every arrow/.append style={dash}]
&&LF\ar{dl}\ar{dd}\ar{dr}\\
&L \ar{dr}\ar{ddr} & & F \ar{dl}\ar{ddl}\\
&&L\cap F\ar{d}\\
&& K
\end{tikzcd} 
\end{minipage}\hfill

\subsection{Corrispondenza di Galois}
\begin{definition}{estensione separabile}
    $L/K$ si dice separabile se $\forall \alpha \in L \ \mu_{\alpha}(x)$ ha derivata non nulla. Osserviamo che ciò è certamente vero se il campo $K$ è perfetto.
\end{definition}

\begin{definition}{estensione di Galois}
    $L/K$ si dice di Galois se è normale e separabile. Osserviamo che se il campo $K$ è finito o a caratteristica 0 (campi con cui lavoreremo per ora) per quanto detto allora ``di Galois'' e ``normale'' si equivalgono.
\end{definition}
\begin{definition}{gruppo di Galois}
    Se $L/K$ è di Galois $\Aut_K(L) = \{\varphi : L \rightarrow \overline K : \varphi|_K = id\}$ per quanto detto finora è un gruppo con la composizione. $\Aut_K(L)$ si indica anche con $\Gal(L/K)$ (gruppo di Galois di $L$ su $K$). Inoltre se $L/K$ è anche finita $\#\Aut_K(L) = [L:K]$.
\end{definition}

\begin{theorem}{Galois di un campo di spezzamento}
    Sia $f \in K[x]$ irriducibile di grado $n$, e sia $L$ il campo di spezzamento di $f$ su $K$. Allora $n \mid [L:K] \mid n!$ e inoltre $\Gal(L/K) \hookrightarrow S_n$.
\end{theorem}
\begin{proof}
    Sia $\alpha \in \overline{K}$ una radice di $f$. Allora per irriducibilità $f(x) = u \mu_{\alpha}(x)$ con $u \in K^\times$. Poiché allora $K \subseteq K(\alpha) \subseteq L$ e $[K(\alpha) : K] = n$, si ha $n | [L:K]$ per torri.

    Siano ora $\alpha_1,\dots,\alpha_n$ le radici di $f$. Per quanto detto finora $\forall \varphi \in \Gal(L/K)$  $\varphi(\{\alpha_1,\dots,\alpha_n\}) = \{\alpha_1,\dots,\alpha_n\}$ e quindi possiamo considerare l'azione di $\Gal(L/K)$ su $\{\alpha_1,\dots,\alpha_n\}$ data da $\varphi \mapsto \varphi_{\{\alpha_1,\dots,\alpha_n\}}$. Questa mappa è chiaramente un omomorfismo iniettivo e quindi $\Gal(L/K) \hookrightarrow S(\{\alpha_1,\dots,\alpha_n\}) \cong S_n$, da cui segue $[L:K] = \#\Gal(L/K) \mid \#S_n = n!$
\end{proof}
\begin{proposition}{irriducibilità e orbite}
    Sia $f \in K[x]$ di grado $n$, con radici $\alpha_1,\dots,\alpha_n$ e sia $L$ il campo di spezzamento di $f$ su $K$. $\Gal(L/K)$ agisce sulle radici di $f$ per permutazione e l'azione è transitiva $\iff$ $f$ è irriducibile.
\end{proposition}
\begin{proof}
    Consideriamo l'azione della dimostrazione precedente (è analoga anche se abbiamo tolto l'ipotesi di irriducibilità). In modo analogo a prima agisce per permutazione su $\{ \text{radici di } f(x)\}$. Sappiamo poi dalle proposizioni dimostrate sopra che con questa azione $\text{orb}(\alpha) = \{ \text{radici di } \mu_{\alpha}(x)\}$. Se $\alpha$ è radice di $f(x)$, allora si ha $\mu_{\alpha} \mid f$ e di conseguenza $f$ irriducibile $\iff$ $f(x) = u\mu_{\alpha}(x)$ con $u \in K^\times$ $\iff$ $\{ \text{radici di } f(x)\} =  \{ \text{radici di } \mu_{\alpha}(x)\} =\text{orb}(\alpha)$ ovvero l'azione è transitiva.
\end{proof}
\begin{theorem}{dell'elemento primitivo}
        Sia $L/K$ finita e separabile. Allora $L/K$ è semplice, ovvero $\exists \alpha \in L$ tale che $L = K(\alpha)$. 
\end{theorem}
\begin{proof}
    Se $K$ è un campo finito allora anche $L$ finito poiché $L/K$ finita. Sappiamo che $L^\times = L \setminus \{0\}$ è un gruppo moltiplicativo ciclico, dunque la tesi.

    Se $K$ è infinito notiamo che $L/K$ finita implica che è finitamente generata e quindi esistono $\alpha_1, \dots,\alpha_n$ tali che $L = K(\alpha_1, \dots,\alpha_n)$. Procediamo invece per induzione su $n$, trattando prima di tutto il caso $n=2$ e poi passando al caso generale.
    
    Sia $L = K(\alpha, \beta)$ e sia $d = [L:K]$. Allora per un teorema dimostrato precedentemente esistono esattamente $d$ immersioni $\varphi_1, \dots, \varphi_d : L \rightarrow \overline{K}$ tali che $\varphi_i |_K = id$. Consideriamo il polinomio
    \[
        F(x) = \prod_{1\leq i < j \leq d} ((\varphi_i(\alpha) + x\varphi_i(\beta)) - (\varphi_j(\alpha) + x\varphi_j(\beta))).
    \]
    $F$ è certamente non nullo perché prodotto di fattori non nulli: $(\varphi_i(\alpha) + x\varphi_i(\beta)) - (\varphi_j(\alpha) + x\varphi_j(\beta)) = 0 \iff \varphi_i(\alpha) + x\varphi_i(\beta) = \varphi_j(\alpha) + x\varphi_j(\beta) \iff \varphi_i(\alpha)=\varphi_j(\alpha) \text{ e } \varphi_i(\beta) = \varphi_j(\beta) \iff \varphi_i = \varphi_j$, ma vale sempre $i \neq j$.
    
    Poiché il campo $K$ è infinito, sicuramente esiste un $t \in K$ tale che $F(t) \neq 0$, e quindi i $\varphi_i(\alpha) + t\varphi_i(\beta)$ sono tutti distinti. Sia $\gamma = \alpha + t\beta \in L$ per questo fissato $t$. Per omomorfismo $\varphi_i(\gamma) = \varphi_i(\alpha) + t\varphi_i(\beta)$ e quindi si ha che $\varphi_1(\gamma),\dots, \varphi_n(\gamma)$ sono tutti distinti. Ciò implica che $[K(\gamma) : K] \geq d$ (sappiamo che il grado dell'estensione è il grado del polinomio minimo di $\gamma$ su $K$ e che le immagini di $\gamma $ secondo questi omomorfismi sono radici di tale polinomio minimo; ne abbiamo trovate $d$ distinte quindi il grado è almeno $d$). Ma da $K(\gamma) \subseteq L$ segue allora che $[K(\gamma) : K]=d$ e $L = K(\gamma)$. \\
A questo punto possiamo svolgere l'induzione. Per il passo base $n=1$ la tesi è ovvia. Per il passo induttivo, notiamo che per ipotesi induttiva $L = K(\alpha_1, \dots,\alpha_n) = K(\beta, \alpha_n)$ e per quanto detto nel caso $n=2$ esiste allora $\gamma \in L$ tale che $L = K(\beta, \alpha_n) = K(\gamma)$. 
\end{proof}
\begin{definition}{campo fissato da un sottogruppo}
    Sia $L/K$ di Galois, e $H \leq \Gal(L/K)$. Indichiamo con $L^H = \Fix(H) = \{ \alpha \in L : \sigma(\alpha) = \alpha \ \forall \sigma \in H\}$ il campo fissato da tutti gli elementi di $H$ (è chiaramente un campo e poiché $H$ è contenuto nel Galois si ha $K \subseteq L^H$).
\end{definition}
\begin{theorem}{corrispondenza di Galois}
    Sia $L/K$ di Galois finita. Allora c'è una corrispondenza tra i sottocampi di $L$ che contengono $K$ e i sottogruppi di $ \Gal(L/K)$, che associa il sottogruppo $H$ al campo fissato $L^H$ (e viceversa un campo al sottogruppo che lo fissa). Inoltre $H \tri \Gal(L/K) \iff L^H/K$ è normale e in tal caso $\Gal(L^H/K) \cong \frac{\Gal(L/K)}{\Gal(L/L^H)}$.
\end{theorem}
\begin{proof}
    Sia $\mathscr{E} = \{ F \text{ campo} : K \subseteq F \subseteq L\}$ e $\mathscr{G}_{L/K} = \{ H \leq \Gal(L/K)\}$. Per torri $L$ è un'estensione normale di tutti i campi in $\mathscr{E}$, che a loro volta sono estensioni di $K$. Quindi gli oggetti sono ben definiti.
    
    Definiamo $\alpha: \mathscr{E} \rightarrow \mathscr{G}$ $\alpha(F \mapsto \Gal(L/F))$ e $\beta: \mathscr{G} \rightarrow \mathscr{E}$ $\beta(H \mapsto L^H)$.

    \begin{lemma2}
        Sia $H \leq \Gal(L/K)$. Allora $M = L^H \iff H = \Gal(L/M)$.
    \end{lemma2}
    \begin{proof}
        Sia $M$ un campo con $K \subseteq M \subseteq L$. Per torri $L/K$ di Galois $\Rightarrow$ $L/M$ di Galois.
    
        Sia $M = L^H$. Chiaramente $H \subseteq \Gal(L/M)$. Allora per il teorema dell'elemento primitivo (stiamo lavorando con estensioni finite) si ha $L = M(\alpha)$ per un qualche $\alpha \in L$. Consideriamo $f(x) = \prod_{\sigma \in H} (x - \sigma(\alpha)) \in L[x]$. Notiamo che  $\forall \rho \in H$, indicando di nuovo con $ \rho f$ il polinomio applicando $\rho $ ai coefficienti di $f$, si ha $\rho f(x) = \prod_{\sigma \in H} (x - \rho\circ\sigma(\alpha)) =  \prod_{\tau \in H} (x - \tau(\alpha)) = f(x)$ e quindi $f(x) \in L^H[x] = M[x]$ (i suoi coefficienti restano invariati applicando $\rho \in H$). D'altra parte si ha $f(\alpha) = 0$ (basta prendere $\sigma = id$) e $\text{deg}(f) = \#H$, da cui segue $\#\Gal(L/M) = [L:M] = \text{deg}(\mu_{\alpha}) \leq \text{deg}(f(x)) = \#H$. Quindi $\#\Gal(L/M) = \#H$ (avevamo la disuguaglianza inversa per contenimento) e quindi $\Gal(L/M) = H$.
    
        Per l'altra freccia sia ora $H = \Gal(L/M)$. $M \subseteq L^H$ segue dalle definizioni. Supponiamo ora per assurdo che $M \neq L^H$. Allora per teoria precedente $\exists \varphi : L^H \rightarrow \overline{M}$ tale che $\varphi \neq id$ ma $\varphi|_M = id$. Possiamo estendere $\varphi$ a $L$ ottenendo $\tilde \varphi : L \rightarrow \overline{M}$ tale che di nuovo $\tilde \varphi \neq id$ ma $\tilde \varphi|_M = id$. Quindi $\tilde \varphi \in \Gal(L/M) = H$. Allora per definizione $\tilde \varphi|_{L^H} = \varphi  = id$ che è assurdo. 
    \end{proof}
    Usando il lemma appena dimostrato si ha da una parte $\alpha \circ \beta (H) = \Gal(L/L^H) = H$ e quindi $\beta \circ \alpha = id$ e dall'altra $\beta \circ \alpha (F) = L^{\Gal(L/F)} = F$ e quindi $\beta \circ \alpha = id$  (stiamo usando le due frecce della coimplicazione separatamente). Quindi $\alpha$ e $\beta$ sono entrambe bigettive, e sono una l'inversa dell'altra.
    
    \textbf{Lemma 2:} Sia $H \leq \Gal(L/K)$, $\sigma \in \Gal(L/K)$. Allora $L^{\sigma H \sigma^{-1}}=\sigma(L^H)$.
    \begin{proof}
         Basta notare che $\sigma(L^H) = \{ \sigma(\alpha) \in L : \forall \varphi \in H \ \varphi(\alpha) = \alpha\} =  \{ \beta \in L : \forall \varphi \in H \ \varphi(\sigma^{-1}(\beta)) = \sigma^{-1}(\beta)\} = \{ \beta \in L : \forall \varphi \in H \ \sigma \circ \varphi \circ \sigma^{-1}(\beta) = \beta \} = L^{\sigma H \sigma^{-1}}$.
    \end{proof}
    Usando il lemma appena dimostrato si ha: $H \tri \Gal(L/K) \iff \forall \sigma \in \Gal(L/K) \ \sigma H \sigma^{-1} = H$ Ma per il lemma 1 questo equivale a $ \forall \sigma \in \Gal(L/K) \ L^{\sigma H \sigma^{-1}}=L^H \iff  \forall \sigma \in \Gal(L/K) \ \sigma(L^H)=L^H \iff L^H/K$ è normale.
    
    Infine mostriamo l'isomorfismo di gruppi. Sia $res : \Gal(L/K) \rightarrow \Gal(L^H/K)$ la restrizione a $L^H$. $res$ è suriettivo perché $L/K$ algebrica $\Rightarrow$ $L/L^H$ algebrica e quindi ciascun omomorfismo in $\Gal(L^H/K)$ si estende a uno in $\Gal(L/K)$. \\ $\ker(res) = \{ \sigma \in \Gal(L/K) : \sigma|_{L^H} = id\} = \Gal(L/L^H) = H$ per il lemma 1. Allora per il $1^{\circ}$ teorema di omomorfismo si ha $\Gal(L^H/K) \cong \frac{\Gal(L/K)}{\Gal(L/L^H)}$.
\end{proof}

\begin{theorem}{Galois in campi finiti}
    $\Gal(\Fpn / \Fpm) = \grp{\Phi^m} \cong \Z/(n/m)\Z$, dove $\Phi : x \mapsto x^p$ è l'omomorfismo di Frobenius.
\end{theorem}
\begin{proof}
    Segue direttamente dalla caratterizzazione di $\Aut(\Fpn) = \grp{\Phi} \cong \Zn$, infatti $\#\Gal(\Fpn / \Fpm) = [\Fpn : \Fpm] = n / m$ e $\Gal(\Fpn / \Fpm) \le \Aut(\Fpn) \cong \Zn$, che ha un unico sottogruppo di ordine $n / m$. Si noti che effettivamente $\Phi^m : x \mapsto x^{p^m}$ ristretto a $\Fpm$ è l'identità.
\end{proof}

\subsection{Fatti sui gruppi di Galois}
%isomorfismi dei galois con le operazioni
\begin{minipage}{0.5\textwidth}  
\begin{tikzcd}[every arrow/.append style={dash}]
&&LF\ar{dl}\ar{dr}\\
&L \ar{dr} & & F \ar{dl}\\
&& K = L \cap F
\end{tikzcd}  
\end{minipage}\hfill
\begin{minipage}{0.5\textwidth}
\begin{theorem}{del traslato}
    Siano $K = L\cap F$, $F/K$ di Galois e $L/K$ algebrica. Allora $LF/L$ è di Galois e $\Gal(LF/L) \cong \Gal(F/K)$.
\end{theorem}
\end{minipage}\hfill
\begin{proof}
    $LF/L$ è di Galois per shift di estensioni normali.
    Consideriamo l'omomorfismo $res: \Gal(LF/L) \rightarrow \Gal(F/K)$ dato dalla restrizione a $F$. Esso è chiaramente un omomorfismo e $\ker(res) = \{ \varphi \in \Gal(LF/L) : \varphi|_F = id \} $. Ma $\forall \varphi \in \Gal(LF/L) \varphi|_L = id$ per definizione e quindi $\varphi \in \ker(res) \iff \varphi|_LF = id \iff \varphi = id$ essendo $LF$ il dominio di $\varphi$.
    
    Per dimostrare la suriettività, sia $I = \imm(res)$.
    Il suo campo fissato è $F^I = \{\alpha \in F : \forall \sigma \in I \ \sigma(\alpha) = \alpha \} = \{\alpha \in F : \forall \varphi \in \Gal(LF/L) \ \varphi|_F(\alpha) = \alpha \} = F \cap \{\alpha \in LF : \forall \varphi \in \Gal(LF/L) \ \varphi(\alpha) = \alpha \} = F \cap L = K$. Allora per corrispondenza di Galois $ \imm(res) = I = \Gal(F/K)$. 
\end{proof}
\begin{corollary}{moltiplicatività del composto}
    Siano come prima $K = L\cap F$, $F/K$ di Galois e $L/K$ algebrica. Allora $[LF:K] = [F:K][L:K]$.
\end{corollary}
\begin{proof}
    Per il teorema, $\Gal(LF/L) \cong \Gal(F/K) \Rightarrow [LF:L] = [F:K]$. Ma per torri si ha allora $[LF:L]=[LF:L][L:K] = [F:K][L:K]$. 
\end{proof}
\begin{theorem}{campi fissati e operazioni}
    Sia $L/K$ di Galois finita, e $H,S \leq \Gal(L/K)$. Allora valgono le seguenti:
    \begin{enumerate}[label=($\roman*$)]
        \item $L^H \subseteq L^S \iff H \supseteq S$
        \item $L^{H\cap S} = L^HL^S$
        \item $L^{\langle H, S \rangle} = L^H \cap L^S$
    \end{enumerate}
    \begin{minipage}{0.5\textwidth}  
    \begin{tikzcd}[every arrow/.append style={dash}]
    && L\ar{d} \\
    && L^HL^S\ar{dl}\ar{dr}\\
    &L^H \ar{dr} & & L^S \ar{dl}\\
    && L^H \cap L^S \ar{d}\\
    && K 
    \end{tikzcd}  
    \end{minipage}\hfill
    \begin{minipage}{0.5\textwidth}  
    \begin{tikzcd}[every arrow/.append style={dash}]
    && \{id\}\ar{d} \\
    && H \cap S \ar{dl}\ar{dr}\\
    &H \ar{dr} & & S \ar{dl}\\
    && \langle H, S \rangle \ar{d}\\
    && \Gal(L/K)
    \end{tikzcd}  
    \end{minipage}\hfill
\end{theorem}
\begin{proof}
 Per $(i)$ basta notare che $H \supseteq S \Rightarrow L^H \subseteq L^S$ segue direttamente dalla definizione, mentre sempre per definizione  $L^H \subseteq L^S \Rightarrow \Gal(L/L^H) \supseteq \Gal(L/L^S)$ e per uno dei lemmi visti a lezione $H = \Gal(L/L^H), S = \Gal(L/L^S)$. \\
Per $(ii)$ notiamo intanto che per $(i)$ $L^H \subseteq L^{H\cap S}$ e analogamente $L^S$, quindi $L^HL^S \subseteq L^{H\cap S}$. Per corrispondenza di Galois $L^HL^S = L^N$ per $N = \Gal(L/L^HL^S)$. Ma allora $N = \Gal(L/L^H) \cap \Gal(L/L^S) = H \cap S$. Infatti si ha $\Gal(L/L^HL^S) =  \{\varphi : L \rightarrow \overline{K} : \varphi \text{ omomorfismo e } \varphi_{L^HL^S} = id\} = \{\varphi : L \rightarrow \overline{K} : \varphi \text{ omomorfismo e }\varphi_{L^H} = id, \varphi_{L^H} = id\} = \Gal(L/L^H) \cap \Gal(L/L^S)$.\\
Per $(iii)$ come prima notiamo intanto che per $(i)$ $L^H \supseteq L^{\langle H, S \rangle}$ e analogamente $L^S$. Quindi $L^H \cap L^S \supseteq L^{\langle H, S \rangle}$. Notiamo ora che $\alpha \in L^H \cap L^S \Rightarrow \forall \nu \in H, \psi \in S, \ \nu(\alpha)=\alpha = \psi(\alpha) \Rightarrow \forall \varphi \in \langle H, S \rangle, \ \varphi(\alpha)=\alpha \Rightarrow \alpha \in L^{\langle H, S \rangle}$  quindi $L^H \cap L^S \subseteq L^{\langle H, S \rangle}$.
\end{proof}

\begin{theorem}{estensioni con radici quadrate}
    Sia $\char(K) \neq 2$. Allora dati $\alpha, \beta \in K$, $K(\sqrt{\alpha}) = K(\sqrt{\beta}) \iff \alpha\beta \text{ è un quadrato in } K \iff \frac{\alpha}{\beta} \text{ è un quadrato in } K$.
\end{theorem}
\begin{proof}
    $\alpha\beta \text{ è un quadrato in } K \iff \frac{\alpha}{\beta} \text{ è un quadrato in } K$ è ovvia (basta moltiplicare/dividere per $\beta^2$).
    
    $K(\sqrt{\alpha}) = K(\sqrt{\beta}) \iff \exists x,y \in K \ \sqrt{\alpha} = x+y \sqrt{\beta}$, i.e. $\sqrt{\alpha} - y \sqrt{\beta} = x$. Se $x=0$ si ha $\sqrt{\frac{\alpha}{\beta}} \in K$ e la tesi è ovvia. Se $y=0$ si ha $\sqrt{\alpha} \in K$, e allora $\alpha\beta \text{ quadrato in } K \iff  \beta$ è un quadrato in $K$, quindi $\iff K(\beta) = K = K(\alpha)$. Altrimenti elevando al quadrato
    $\alpha + y^2 \beta - 2y\sqrt{\alpha \beta} = x^2 \iff \sqrt{\alpha\beta} = \frac{\alpha + y^2\beta - x^2}{2y} \in K \iff \frac{\alpha}{\beta} \text{ è un quadrato in } K$. 
\end{proof}

\begin{theorem}{Galois di una biquadratica}
    Siano $p(x) = x^4 + ax^2 + b \in \Q[x]$ irriducibile e $L$ il suo campo di spezzamento. Sia $\Delta = a^2 - 4b$. Allora $G = \Gal(L / \Q)$ è isomorfo a:
    \begin{itemize}
        \item $\Z/2\Z \times \Z/2\Z$ se $b$ è un quadrato in $\Q$;
        \item $\Z/4\Z$ se $b\Delta$ è un quadrato in $\Q$;
        \item $D_4$ se né $b$ né $b\Delta$ sono quadrati in $\Q$.
    \end{itemize}
\end{theorem}
\begin{proof}
    Teniamo a mente $G \hookrightarrow S_4$.
    Ponendo $t = x^2$ le soluzioni di $p$ sono $t_{\pm} = \frac{-a \pm \sqrt{\Delta}}{2}$. Siano allora $x_1 = \sqrt{t_+}, x_2 = \sqrt{t_-}, x_3 = -x_1, x_4=-x_3$ le radici della biquadratica. Si ha $L = \Q(x_1,x_2,x_3,x_4)=\Q(x_1,x_3)$.

    \begin{minipage}{0.55\textwidth}  
    Consideriamo il diagramma a lato. Notiamo che $t_+ \in \Q(x_1)$ e quindi $\Q(\sqrt{\Delta}) \subseteq \Q(x_1)$ e analogamente $\Q(\sqrt{\Delta}) \subseteq \Q(x_2)$. $p$ non si spezza in fattori di grado due, quindi $\sqrt{\Delta} \notin \Q$, da cui $[\Q(\sqrt{\Delta}) : \Q] = 2$. Per irriducibilità di $p$ seguono anche $[\Q(x_1): \Q] = [\Q(x_2) :  \Q] = 4$. Per torri e shift $[\Q(x_1) : \Q(\sqrt{\Delta})] = 2$ e $[L : \Q(x_2)] \leq 2$, da cui $[L : \Q] = \in \{4, 8\}$.
    \end{minipage}\hfill
    \begin{minipage}{0.4\textwidth}  
    \begin{tikzcd}[every arrow/.append style={dash}]
    &L\arrow{dl}[swap]{\leq 2}\arrow{dr}{\leq 2}\\
    \Q(x_1)\arrow{dr}{2}\arrow{ddr}[swap]{4} &&\Q(x_2)\arrow{dl}[swap]{2}\arrow{ddl}{4}\\
    & \Q(\sqrt{\Delta}) \arrow{d}{2}\\
    & \Q
    \end{tikzcd}  
    \end{minipage}\hfill

    Notiamo $[L : \Q] = 4 \iff \Q(x_1) = \Q(x_2)$. $\Q(x_1) = \Q(\sqrt{\Delta}, \sqrt{t_+})$ e $\Q(x_2) = \Q(\sqrt{\Delta}, \sqrt{t_-})$: per un teorema precedente queste coincidono sse $\sqrt{t_+ t_-} \in \Q(\sqrt{\Delta})$. Ma $x_1x_2 = \sqrt{t_+ t_-} = \sqrt{b}$.
    
    \begin{minipage}{0.5\textwidth}
    Studiamo quindi quest'altro diagramma. Per quanto detto finora $[L : \Q] = 4 \iff [\Q(\sqrt{\Delta}, \sqrt{b}) : \Q] = 2$. Ma allora, poiché $[\Q(\sqrt{\Delta}) : \Q] = 2$ per torri $[\Q(\sqrt{\Delta}, \sqrt{b}) : \Q(\sqrt{\Delta})] = 1$, che si verifica quando 
    \begin{itemize}
        \item $b$ è quadrato in $\Q \Rightarrow [\Q(\sqrt{b}) : \Q] = 1$;
        \item $b$ non è un quadrato in $\Q$ e $\Q(\sqrt{\Delta}) = \Q(\sqrt{b})$, che accade sse $b\Delta$ è quadrato.
    \end{itemize}
    \end{minipage}\hfill
    \begin{minipage}{0.5\textwidth}  
    \begin{tikzcd}[every arrow/.append style={dash}]
    &\Q(\sqrt{\Delta}, \sqrt{b})\ar{dl}\ar{dr}\\
    \Q(\sqrt{\Delta}) \arrow{dr}{2} & & \Q(\sqrt{b}) \arrow{dl}{2 \text{ o } 1}\\
    & \Q
    \end{tikzcd} 
    \end{minipage}\hfill
    
    Studiamo le strutture dei gruppi nei due casi. Gli unici strutture possibili di un gruppo di ordine 4 sono $\Z/4\Z$ e $\Z/2\Z \times \Z/2\Z$. Una caratteristica che li distingue certamente è la presenza o meno di un elemento di ordine $4$. Ricordiamo che per irriducibilità di $p$ $G$ agisce transitivamente sull'insieme delle radici. Chiamiamo $f_i \in G$ l'elemento del Galois tale che $f(x_1) = x_i$ per $i = 1, \dots, 4$. È chiaro che $f_1 = id$ e $\ord(f_3) = 2$. Gli altri elementi sono o entrambi di ordine $4$ o entrambi di ordine $2$. Studiamo l'ordine di $f_2$. $f_2^2(x_1) = f_2(x_2) = f_2(\frac{\sqrt{b}}{x_1}) = \frac{f_2(\sqrt{b})}{x_2}$. 
    \begin{itemize}
        \item Se $b$ è un quadrato in $\Q$, allora $\frac{f_2(\sqrt{b})}{x_2} = \frac{\sqrt{b}}{x_2} = x_1$ e quindi $f_2$ ha ordine 2, da cui segue che $G \cong \Z/2\Z \times \Z/2\Z$. 
        \item Se invece $b$ non è un quadrato in $\Q$ allora $\sqrt{b} = \frac{c}{\sqrt{\Delta}}$ per qualche $c \in \Q$. Allora $\frac{f_2(\sqrt{b})}{x_2} = \frac{c}{f_2(\sqrt{\Delta}) x_2}$. Notando che $\sqrt{\Delta} = 2x_1^2 + a = -(2x_2^2 + a)$ si ha $f_2(\sqrt{\Delta}) = 2f_2(x_1)^2 + a = 2x_2^2 + a = -\sqrt{\Delta}$, quindi $f_2^2(x_1) = -x_1 = x_3$ e $f_2$ ha ordine $4$, da cui segue $G \cong \Z/4\Z$.
    \end{itemize}
    Se invece $[L : \Q] = 8$ allora $G$ è isomorfo a un $2$-Sylow di $S_4$, quindi $D_4$.
\end{proof}
\begin{theorem}{Galois delle radici dell'unità}
     Sia $\zeta_n = e^{\frac{i \pi}{n}}$ una radice $n$-esima dell'unità. Allora $\Q(\zeta_n)/\Q$ è di Galois e $\Gal(\Q(\zeta_n)/\Q) \cong (\Zn)^\times$.
\end{theorem}
\begin{proof}
    L'estensione $\Q(\zeta_n) = \Q(\zeta_n^k \mid k = 0, \dots, n-1)$ è il campo di spezzamento di $x^n-1$ su $\Q$, dunque un'estensione normale di $\Q$.
    Ogni elemento di $\Gal(\Q(\zeta_n)/\Q)$ è univocamente determinato dall'immagine di $\zeta_n$. Poiché gli omomorfismi iniettivi preservano gli ordini, $\psi(\zeta_n) = \zeta_n^k$ con $(n, k) = 1$, da cui $\#\Gal(\Q(\zeta_n)/\Q) \le \varphi(n)$.

    \begin{lemma2}
        Dato un primo $p$ che non divide $n$ e una radice primitiva $\zeta_n$, $\zeta_n$ e $\zeta_n^p$ sono coniugate.
    \end{lemma2}
    \begin{proof}
        Dato $p \nmid n$, per il criterio della derivata $x^n - 1$ ha radici distinte in $\overline{\Fp}$.
        
        Siano $f(x)$ il polinomio minimo di $\zeta_n$ e $g(x)$ il polinomio minimo di $\zeta_n^p$ su $\Q$. $f(x)$ e $g(x)$ sono divisori monici di $x^n - 1$ in $\Q[x]$, quindi per il lemma di Gauss sono a coefficienti interi. $g(x^p)$ si annulla in $\zeta_n$, quindi $g(x^p) = f(x)h(x)$ in $\Z[x]$ per il lemma di Gauss.
         
        Se per assurdo fosse $f(x) \neq g(x)$, allora $(f(x), g(x)) = 1$ e quindi $f(x)g(x) \mid x^n - 1$, cioè $x^n - 1 = f(x)g(x)l(x)$ in $\Z[x]$. Proiettiamo questa relazione $\mod p$: $\overline{x^n - 1} = \overline{f(x)} \overline{g(x)} \overline{l(x)}$. D'altra parte $\overline{f(x)} \overline{h(x)} = \overline{g(x^p)} = \overline{g(x)}^p$, quindi ogni radice di $\overline{f(x)}$ in $\overline{\Fp}$ è anche una radice di $\overline{g(x)}$. Data allora $\alpha \in \overline{\Fp}$ radice di $\overline{f(x)}$, essa sarebbe radice di $x^n - 1$ in $\overline{\Fp}$ con molteplicità almeno due: assurdo.
    \end{proof}
    Per induzione allora $\zeta_n$ coniugata a $\zeta_n^k$ per ogni $k$ coprimo con $n$. Dunque anche $\deg f(x) \geq \varphi(n)$, da cui l'uguaglianza.

    Consideriamo ora la funzione iniettiva $\Psi: \Gal(\Q(\zeta_n)/\Q) \rightarrow (\Zn)^\times$ definita da $\sigma(\zeta_n \mapsto \zeta_n^k) \mapsto k$.
    Per quanto appena visto $\Psi$ è anche surgettiva. Inoltre, se $\sigma_1(\zeta_n \mapsto \zeta_n^{k_1})$ e $\sigma_2(\zeta_n \mapsto \zeta_n^{k_2})$, allora $\sigma_2 \circ \sigma_1(\zeta_n \mapsto (\zeta_n^{k_1})^{k_2} = \zeta_n^{k_1k_2})$, quindi $\Psi(\sigma_2 \circ \sigma_1) = k_1k_2 = \Psi(\sigma_1)\Psi(\sigma_2)$, cioè $\Psi$ è l'isomorfismo cercato.
\end{proof}

Il polinomio minimo di una radice $n$-esima primitiva dell'unità è detto ``$n$-esimo polinomio ciclotomico'' e si indica con $\Phi_n(x)$. Dalla dimostrazione precedente deduciamo $\deg \Phi_n(x) = \varphi(n)$. Vale inoltre
\[
    x^n - 1 = \prod_{d \mid n}{\Phi_d(n)},
\]
infatti entrambi i membri sono polinomi monici, con le stesse radici e privi di radici doppie. Prendendo i gradi ridimostriamo la nota identità $n = \sum_{d \mid n}{\varphi(n)}$, già incontrata durante il corso di aritmetica.

\begin{theorem}{Galois con radici di primi}
    Siano $p, q$ primi e $f(x) = x^p -  q \in \Q[x]$. Detto $L$ il campo di spezzamento di
    $f$ vale
    \[
        \Gal(L/\Q) \cong \Zp \rtimes (\Zp)^\times
    \]
    dove $(\Zp)^\times$ agisce su $\Zp$ tramite moltiplicazione.
\end{theorem}
\begin{proof}
    Con le tecniche già viste in questo capitolo dimostriamo che $L = \Q(\zeta_p, \sqrt[p](q))$ ha grado $p(p-1)$ su $\Q$. I coniugati di $\zeta_p$ sono gli $\zeta_p^j$ con $j = 1 \dots p-1$, quelli di $\sqrt[p]{q}$ sono gli $\zeta_p^k$ con $k = 0\dots p-1$. Un elemento del Galois manda coniugati in coniugati, quindi i $p(p-1)$ elementi di $G = \Gal(L/\Q)$ sono tutti e soli quelli che mappano $\zeta_p,\sqrt[p]{q}$ in rispettivi coniugati. Rimane da trovare una funzione bigettiva $\Psi : \Zp \rtimes (\Zp)^\times \to G$ e verificare che sia un'omomorfismo: quella a cui probabilmente stai pensando funziona.
\end{proof}

\begin{exercise}
    Si completino i dettagli della dimostrazione precedente. Si dimostri poi che l'enunciato vale più in generale nel caso $f(x) = x^p - a$ con $a$ non potenza $p$-esima perfetta. Si dimostri in particolare che $x^p - a$ è irriducibile.

    \tiny{HINT: tramite cambio di variabile, è sufficiente che in $a$ appaia un primo $q$ con esponente congruo a $1$ $\mod p$ per ottenere un polinomio $q$-Eisenstein.}

    \tiny{HINT: sfruttare il fatto che ogni classe di resto nonzero $\mod p$ ammette un inverso e che $\Q(\sqrt[p]{a}^k) \subseteq \Q(\sqrt[p]{a})$}.
\end{exercise}

\begin{theorem}{(*) $\sqrt{\pm p} \in \Q(\zeta_p)$}
    Se $p$ è un primo diverso da 2, allora $\sqrt{\pm p} \in \Q(\zeta_p)$, dove il segno $\pm$ dipende da $p$ modulo 4: è $-$ se $p \equiv 3 \pmod{4}$, $+$ se $p \equiv 1 \pmod{4}$.
\end{theorem}
\begin{proof}
    (non fatta a lezione e non la dimostrazione canonica). Se il lettore ha seguito/sta seguendo il corso di Analisi numerica si sarà imbattuto nella matrice di Fourier, un tipo particolare di matrice di Vandermonde. Dati $(x_1. x_2, \dots, x_n)$ definiamo la Vandermonde (quadrata) come:
\begin{equation*}
V(x_0, x_1, x_2, \dots, x_m) = 
\begin{bmatrix}
1 & x_0 & x_0^2 & \dots & x_0^{n-1}\\
1 & x_1 & x_1^2 & \dots & x_1^{n-1}\\
1 & x_2 & x_2^2 & \dots & x_2^{n-1}\\
\vdots & \vdots & \vdots & \ddots & \vdots \\ 
1 & x_{n-1} & x_{n-1}^2 & \dots & x_{n-1}^{n-1}\\
\end{bmatrix}
\end{equation*}
E quella di Fourier con (detta $\zeta_n$ la radice $n$-esima dell'unità): $\Omega_n = V(1,\zeta_n,\zeta_n^2,\dots, \zeta_n^{n-1})$
%\begin{equation*} \Omega_n = V(1,\zeta_n,\zeta_n^2,\dots, \zeta_n^{n-1}) =  \begin{bmatrix} 1 & 1 & 1 & \dots & 1\\ 1 & \zeta_n & \zeta_n^2 & \dots & \zeta_n^{n-1}\\ 1 & \zeta_n^2 & \zeta_n^4 & \dots & \zeta_n^{2n-2)}\\ \vdots & \vdots & \vdots & \ddots & \vdots \\  1 & \zeta_n^{n-1} & \zeta_n^{2(n-1)} & \dots & \zeta_n^{(n-1)^2}\\ \end{bmatrix} \end{equation*}
Si richiamano alcune proprietà, seguendo le definizioni date sopra:
\begin{itemize}
    \item $\text{det}(V) = \prod_{1 \leq i < j \leq n} (x_j - x_i)$
    \item $\Omega_n^2 = n \cdot
    \begin{bmatrix}
1 & 0 & 0 & \dots & 0 & 0\\
0 & 0 & 0 & \dots & 0 & 1\\
0 & 0 & 0 & \dots & 1 & 0\\
\vdots & \vdots & \vdots  & \ddots & \vdots & \vdots \\ 
0 & 0 & 1 & \dots & 0 & 0  \\
0 & 1 & 0 & \dots & 0 & 0 
\end{bmatrix}  \quad \Rightarrow \quad \text{det}(\Omega_n)^2 = n^n \cdot (-1)^{\frac{n(n+1)}{2} - 1}$
\end{itemize}
Sostituendo $n$ con $p \geq 3$ primo, $\Omega_n = V(1,\zeta_p,\zeta_p^2,\dots, \zeta_p^{p-1})$, quindi
    \[
        \sqrt{p^p \cdot (-1)^{\frac{p(p+1)}{2} - 1}} =  \text{det}(\Omega_n) = \prod_{1 \leq i < j \leq n} (\zeta_p^j - \zeta_p^i).
    \]
Il lato destro dell'equazione, si ottiene da $\zeta_p$ con somme e prodotti, quindi $\sqrt{p^p \cdot (-1)^{\frac{p(p+1)}{2} - 1}} \in \Q(\zeta_p)$. Ma poiché $p$ dispari $\sqrt{p^p}$ è un intero moltiplicato per $\sqrt{p}$. Quindi $\sqrt{\pm p} \in \Q(\zeta_p)$. Il segno $\pm$ è dato da $\frac{p(p+1)}{2} - 1$: si verifica che è $-1$ se $p \equiv 3 \pmod{4}$, $1$ se $p \equiv 1 \pmod{4}$.


\end{proof}

\subsection{Teorema fondamentale dell'algebra}

% "Il primo teorema di omomorfismo?"
% ~Max
Non il primo teorema di omomorfismo, bensì ``$\C$ è algebricamente chiuso''.

\begin{proof}
    Sia $f(x) \in \C[x]$. Notiamo che $g(x) = f(x) \overline{f}(x) \in \R[x]$. Dimostriamo che il suo campo di spezzamento, che chiamiamo $K$, è $\R$ o $\C$.
    Il campo di spezzamento di $f$ è contenuto in quello di $g$, quindi se $g$ si spezza in $\C$, anche $f$ si spezza.

    \begin{minipage}{0.2\textwidth}  
    \begin{tikzcd}[every arrow/.append style={dash}]
    K\arrow{d}\\
    K^{P_2} = \R(\alpha) \arrow{d}{d = 1}\\
    \R
    \end{tikzcd}  
    \end{minipage}\hfill
    \begin{minipage}{0.75\textwidth}  
    Sia $G = \Gal(K/\R)$ (gruppo finito perché $K$ è campo di spezzamento di un unico polinomio) e sia $P_2$ un suo $2$-Sylow. Allora $d = [G : P_2]$ è un dispari. Per corrispondenza di Galois $K^{P_2}$ è un'estensione di $\R$ di grado $d$. Per il teorema dell'elemento primitivo essa è semplice, ossia $K^{P_2}=\R(\alpha)$ per un qualche $\alpha \in K$. 
    \end{minipage}\hfill 
    
    Detto $\mu(x)\in \R[x]$ il polinomio minimo di $\alpha$, si ha $\deg(\mu(x)) = d$ dispari e quindi per il teorema dei valori intermedi $\mu(x)$ ha almeno una radice in $\R$. Ma allora, poiché $\mu(x)$ è irriducibile, deve valere per forza $d=1$.
    
    Quindi $\#G = 2^n$ per un qualche $n \in \N$ e allora per Sylow esiste una catena di sottogruppi $\{id\}=G_0 \subset G_1 \subset \dots \subset G_n = G$ tale che ciascun gruppo ha indice $2$ nel successivo. Per corrispondenza di Galois esiste allora una catena di campi $K=K_0 \supset K_1 \supset \dots \supset K_n = \R$ tale che ciascuna estensione ha grado $2$ sulla successiva. Ma l'unica estensione di grado $2$ di $\R$ è $\C$, e $\C$ non ha estensioni di grado $2$. Quindi l'unica possibilità è $K = \R$ o $\C$ come voluto.
\end{proof}

\subsection{Esercizi}

\begin{exercise}
    Si contino i polinomi monici irriducibili di grado 10 in $\F_p[x]$.
\end{exercise}
\begin{solution}
    Lavoriamo in una chiusura algebrica $\overline{{\F_p}}$. Sia $f(x) \in \F_p[x]$ irriducibile di grado $n$ di radici $\alpha_1, \dots, \alpha_n$, allora $\F_{p^n} = \F_p(\alpha_i) \cong \frac{\F_p[x]}{(f(x))}$ per unicità di $\F_{p^{10}}$ in $\overline{\F_p}$. Identifichiamo ogni polinomio monico irriducibile con l'insieme delle sue 10 radici: per polinomi irriducibili distinti questi insiemi non si intersecano. $[\F_{p^{10}} : \F_p] = 10$, dunque ogni elemento $\alpha \in \F_{p^{10}}$ ha polinomio minimo di grado al più 10. Se $\deg \mu_\alpha(x) = d$, allora $\F_{p^d} = \F(\alpha) \subseteq \F_{p^{10}}$. Ma sappiamo che $\F_{p^m} \subseteq \F_{p^n} \iff m | n$, dunque i possibili $d$ sono 1,2,5,10. Gli elementi con polinomio minimo di grado 1 o 2 sono tutti e soli gli elementi di $\F_{p^2}$, così come gli elementi con polinomio minimo di grado 1 o 5 sono tutti e soli gli elementi d $\F_{p^5}$. Per il principio di inclusione-esclusione esistono $p^{10} - p^5 - p^2 + p$ elementi con polinomio minimo di grado 10, che corrispondono a $\frac{p^{10} - p^5 - p^2 + p}{10}$ polinomi monici irriducibili di grado 10.
\end{solution}

\begin{exercise}
    Determinare il campo di spezzamento di $f_7(x) = x^7 - 1$ su $\F_5$ e su $\F_{11}$.
\end{exercise}
\begin{solution}
    Si tratta di applicare il teorema sul cds di $x^n - 1$ su $\Fp$.
    $\#(\Z/7\Z^\times) = 6$, dunque $d_5 = \ord_{\Z/7\Z^\times} 5$ e $d_{11} = \ord_{\Z/7\Z^\times} 11$ sono entrambi divisori di 6.

    Si trova $d_5 = 6$ e $d_{11} = 3$, dunque il cds di $x^7 - 1$ su $\F_5$ e su $\F_{11}$ sono rispettivamente $\F_{5^6}$ e $\F_{{11}^3}$.
\end{solution}

\begin{exercise}
    Determinare la forma della fattorizzazione di $x^8 - 1$ su $\Fp$.
\end{exercise}
\begin{solution}
    Se $p = 2$, allora $x^8 - 1 = (x - 1)^8$. Assumiamo ora $p \neq 2$, quindi $(8, p) = 1$. Per quanto visto il campo di spezzamento di $x^8 - 1$ su $\Fp$ è $\F_{p^d}$ con $d = \ord_{{\Z/8\Z}^\times}p$. Ricordiamo $(\Z/8\Z)^\times \cong \Z/2\Z \times \Z/2\Z$, dunque $d \in \{ 1, 2 \}$.
    Ma allora $x^8 - 1$ si spezza come prodotto di fattori di grado uno e due su ogni $\Fp$ e c'è almeno un fattore di grado due se e solo se $d = 2$.
    Indipendentemente dal campo, $x^8 - 1 = (x^4 + 1)(x^2 + 1)(x + 1)(x - 1)$ \dots quindi $x^4 + 1$ è un irriducibile di $\Z[x]$ che per ogni $p$ \textit{non} è irriducibile su $\Fp$!

    Da aritmetica sappiamo che dato $f(x) \in \Z[x]$, se, detta $\bar f(x)$ la proiezione di $f$ modulo $p$, $\bar f(x)$ è irriducibile su $\Fp[x]$, allora $f$ è irriducibile su $\Z[x]$. L'esercizio ci dice che non vale l'implicazione inversa, cioè esistono irriducibili di $\Z[x]$ che per ogni $p$ sono riducibili su $\Fp[x]$ e $x^4 + 1$ è uno di questi.
\end{solution}

\begin{exercise}
    Sia $K$ un campo di caratteristica diversa da 2 e siano $a, b \in K^\times$. Dimostrare che $K(\sqrt{a}) = K(\sqrt{b})$ se e solo se $a/b = c^2$ è un quadrato in $K$.
\end{exercise}
\begin{solution}
    $(\impliedby)$ $a = b c^2$ implica  $\sqrt{a} = c \sqrt{b}$ e $\sqrt{b} = \sqrt{a}/c$, da cui $K(\sqrt{a}) \subseteq K(\sqrt{b}) \subseteq K(\sqrt{a})$, quindi l'uguaglianza.
    $(\implies)$ $\sqrt{b} \in K(\sqrt{a})$ significa $\sqrt{b} = x + y \sqrt{a}$, dunque $x = \sqrt{b} - y\sqrt{a}$ per opportuni $x, y \in K$, elevando al quadrato, isolando le radici e dividendo per $b$ si trova $\frac{x^2 - b - y^2 a}{2yb} = \frac{\sqrt{a}}{\sqrt{b}}$, da cui $a/b$ quadrato in $K$.
\end{solution}

\begin{exercise}
    Siano $p, q \in \Z$ primi distinti. Si dimostri che $\sqrt{p} \in \Q(\sqrt p + \sqrt q)$ e si calcoli il polinomio minimo di $\sqrt p + \sqrt q$. Si dimostri inoltre che $\Q(\sqrt p + \sqrt q)$ è un'estensione normale di $\Q$.
\end{exercise}
\begin{solution}
    Ogni automorfismo di un campo contenente $\Q$ fissa per definizione $1$, dunque fissa puntualmente tutto $\Q$. Per l'esercizio precedente $\sqrt p \notin \Q(\sqrt q)$, per torri e shift $[\Q(\sqrt p, \sqrt q) : \Q] = 4$, quindi esistono quattro immersioni di $\Q(\sqrt p, \sqrt q)$ in $\overline \Q$: tutte e sole le possibili scelte per $\sqrt p \mapsto \pm \sqrt p$ e $\sqrt q \mapsto \pm \sqrt q$. Le immagini di $\sqrt p + \sqrt q$ tramite queste immersioni sono tutte distinte, dunque $[\Q(\sqrt p + \sqrt q) : \Q] \ge 4$, ma poiché $\Q(\sqrt p + \sqrt q) \subseteq \Q(\sqrt p, \sqrt q)$, i due campi coincidono, in particolare quindi $\sqrt p \in \Q(\sqrt p + \sqrt q)$. Poiché $\sqrt p + \sqrt q$ ha grado $4$, i $\pm \sqrt p \pm \sqrt q$ sono tutti e soli i coniugati di $\sqrt p + \sqrt q$, quindi $\mu_{\sqrt p + \sqrt q}(x) = (x - \sqrt p - \sqrt q) \dots (x + \sqrt p + \sqrt q)$. $\Q(\sqrt p + \sqrt q)/\Q$ è normale poiché è il campo di spezzamento di $\mu_{\sqrt p + \sqrt q}(x)$ su $\Q$.
\end{solution}

\begin{exercise}
    Siano $K$ campo e $f, g \in K[x]$ irriducibili con $\deg f = n$, $\deg g = m$, $(n, m) = 1$. Si dimostri che per ogni radice $\alpha$ di $f$ $g$ è irriducibile su $K(\alpha)$.
\end{exercise}
\begin{solution}
    Sia $\beta \in \overline{K}$ una radice di $g$. $[K(\alpha) : K] = n$ e $[K(\beta) : K] = m$, per shift allora $[K(\alpha, \beta) : K(\alpha)] \le m$, per torri allora $[K(\alpha, \beta) : K] \le nm$. Per torri $n, m \mid [K(\alpha, \beta) : K]$, ma allora $[K(\alpha, \beta) : K] = nm$, da cui per torri $[K(\alpha, \beta) : K(\alpha)] = m$. Quindi il polinomio minimo di $\beta$ su $K(\alpha)$ ha grado $m$, cioè è proprio $g$, che quindi è irriducibile.
\end{solution}

\begin{exercise}
    Determinare il polinomio minimo di $\alpha^2$ su un campo $K$ in funzione del polinomio minimo di $\alpha$ su $K$. Sia ora $\alpha = 2 + \sqrt{5 + \sqrt{-5}}$. Determinare il polinomio minimo di $\alpha^2$ su $\Q$.
\end{exercise}
\begin{solution}
    Consideriamo $K \subseteq K(\alpha^2) \subseteq K(\alpha)$. Per torri $n = \deg \mu_\alpha = [K(\alpha) : K] = [K(\alpha) : K(\alpha^2)][K(\alpha^2) : K]$ con $[K(\alpha) : K(\alpha^2)] \le 2$. Scriviamo $\mu_\alpha(x) = p(x^2) + x d(x^2)$. Se $d \equiv 0$, allora $n$ è pari e $p$ è un polinomio di grado $n / 2$ che si annulla in $\alpha^2$, quindi $p = \mu_{\alpha^2}$ e $[K(\alpha) : K(\alpha^2)] = 2$. D'altro canto, se $[K(\alpha) : K(\alpha^2)] = 2$, allora $\deg \mu_\alpha \le 2 \deg \mu_{\alpha^2}$, ma allora $\mu_\alpha(x) = \mu_{\alpha^2}(x^2)$. Quindi $d \equiv 0 \iff [K(\alpha) : K(\alpha^2)] = 2$. Se invece $K(\alpha) = K(\alpha^2)$, allora $\deg \mu_\alpha = \deg \mu_{\alpha^2}$. Consideriamo il polinomio $s(x^2) = (p(x^2) + xd(x^2))(p(x^2) - xd(x)) = p^2(x) - x^2 d^2(x)$, che ha grado $2n$ nell'indeterminata $x$ e grado $n$ nell'indeterminata $x^2$. Abbiamo $s(\alpha^2) = 0$, dunque $s = \mu_{\alpha^2}$.

    Cerchiamo ora $\mu_{\alpha^2}$ con $\alpha = 2 + \sqrt{5 + \sqrt{-5}}$. Per quanto sopra ci basta trovare $\mu_\alpha$ su $\Q$. Consideriamo $\beta = \alpha - 2 = \sqrt{5 + \sqrt{-5}}$.
    Mostriamo che $\Q(\alpha) = \Q(\beta)$ ha grado $4$ su $\Q$. Consideriamo $\Q \subset \Q(\sqrt{-5}) \subset \Q(\beta)$. $[Q(\beta) : \Q(\sqrt{-5})] = 2$ poiché $(a + b \sqrt{-5})^2 = 5 + \sqrt{-5}$ non ha soluzioni razionali (verifica), quindi per torri $[\Q(\beta) : \Q] = 4$.
    $\deg \mu_\alpha = \deg \mu_\beta$, da cui $\mu_\alpha(x) = \mu_\beta(x - 2)$. Il polinomio di quarto grado $(x^2 - 5)^2 - 5 = x^4 - 10x^2 + 20$ si annulla in $\beta$, di cui dunque è il polinomio minimo. Si verifica che $\mu_\alpha(x) = \mu_\beta(x - 2) = p(x^2) + xd(x^2)$ ha termini di grado dispari, dunque il polinomio minimo di $\alpha^2$ è $p^2(x) - x^2d^2(x)$.
\end{solution}

\begin{exercise}
    Sia $p$ un numero primo e siano $\alpha, \beta \in \overline{\Fp}$. Poniamo $m = [\Fp(\alpha) : \Fp]$ e $n = [\Fp(\beta) : \Fp]$. Dimostrare che, se $(m, n) = 1$, allora $[\Fp(\alpha + \beta) : \Fp] = mn$.
\end{exercise}
\begin{solution}
    $(m, n) = 1 \implies p \nmid m \lor p \nmid n$. (wlog) $p \nmid m$.
    Per un esercizio svolto in precedenza, $[\Fp(\alpha, \beta) : \Fp] = \#\Gal(\Fp(\alpha, \beta) / \Fp) = mn$. Siano $\alpha = \alpha_1, \dots, \alpha_m$, $\beta = \beta_1, \dots, \beta_n$ i coniugati rispettivamente di $\alpha$ e di $\beta$. Per cardinalità, $\Gal(\Fp(\alpha, \beta) / \Fp) = \{ \psi(\alpha \mapsto \alpha_i, \beta \mapsto \beta_j) : i = 1, \dots, m, j = 1, \dots, n \}$. Le immagini di $\alpha + \beta$ tramite gli elementi del Galois sono quindi gli elementi nella forma $\alpha_i + \beta_j$: se queste sono $mn$ elementi distinti, allora $\deg \mu_{\alpha + \beta} = mn$, da cui la tesi. Supponiamo
     $\alpha_i + \beta_j = \alpha_k + \beta_l$, allora $\alpha_i - \alpha_k = \beta_l - \beta_j \in \Fp(\alpha) \cap \Fp(\beta) = \Fp$ per coprimalità. Mostriamo che $\alpha_i - \alpha_k = 0$, da cui $i = k$ e $j = l$, quindi la tesi. Supponiamo per assurdo $\alpha$ coniugato a $\alpha + c$ con $c \in \Fp^\times$, cioè $\alpha + c$ radice di $\mu_\alpha(x)$. Allora $\mu_\alpha(x + c)$ è un polinomio monico irriducibile di grado $m$ che si annulla in $\alpha$, dunque $\mu_\alpha(x) = \mu_\alpha(x + c)$. Reiterando otteniamo che per ogni radice $\gamma$ di $\mu_\alpha(x)$, gli elementi $\gamma, \gamma + c, \gamma + (p-1)c$ sono radici di $\mu_\alpha(x)$, da cui $p \mid \deg \mu_\alpha(x)$: assurdo poiché $p \nmid n = \deg \mu_\alpha(x)$. Necessariamente allora $\alpha_i - \alpha_k = 0$, da cui la tesi.
\end{solution}

Ripetendo il ragionamento della soluzione si dimostrano i seguenti fatti:
\begin{enumerate}
    \item Siano $K$ un campo con $\char K = 0$ e $\alpha, \beta$ coniugati su $K$. Allora $\alpha - \beta \notin K$;
    \item Siano $K$ un campo perfetto con $\char K = p$ e $\alpha, \beta$ coniugati di grado $n$ su $K$. Se $(n, p) = 1$, allora $\alpha - \beta \notin K$.
\end{enumerate}


\end{document}

\subsection{Costruzioni con riga e compasso}

\newpage
\begin{minipage}{0.5\textwidth}  
\begin{tikzcd}[every arrow/.append style={dash}]
&&LF\ar{dl}\ar{dr}\\
&L \ar{dr} & & F \ar{dl}\\
&& K
\end{tikzcd}  
\end{minipage}\hfill
\begin{minipage}{0.5\textwidth}
\textbf{Teorema - composto di estensioni finite:} dato il diagramma a lato, vale $L/K$ finita  e $F/K$ finita $\Rightarrow LF/F$ finita.
\end{minipage}\hfill
 \begin{proof}
 TODO
\end{proof}


\begin{tikzcd}[every arrow/.append style={dash}]
&&D_4\ar[d]\ar[dl]\ar[dr]\\
&\langle r^2,f \rangle \ar[d]\ar[dl]\ar[dr]&\langle r\rangle\ar[d] &\langle r^2,rf\rangle\ar[d]\ar[dl]\ar[dr]\\
\langle f\rangle\ar[rrd]&\langle r^2,f\rangle\ar[rd]&\langle r^2\rangle\ar[d]&\langle rf\rangle\ar[ld]&\langle r^3f\rangle\ar[lld]\\
&&\langle e\rangle
\end{tikzcd}

\begin{minipage}{0.2\textwidth}  
\begin{tikzcd}
G \arrow{r}{\varphi} \arrow{d}{\pi_N} & G'\\
G/N \arrow{ur}[swap]{f}
\end{tikzcd}
\end{minipage}\hfill
\begin{minipage}{0.8\textwidth}  
\end{minipage}\hfill


% ------------------------------------------------------------------------------
% Reference and Cited Works
% ------------------------------------------------------------------------------

%\bibliographystyle{IEEEtran}
%\bibliography{References.bib}

% ------------------------------------------------------------------------------


